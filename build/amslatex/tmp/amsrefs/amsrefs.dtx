%%% ====================================================================
%%% @LaTeX-doc-source-file{
%%%   filename  = "amsrefs.dtx",
%%%   version   = "2.0",
%%%   date      = "2004/06/30",
%%%   time      = "11:24:39 EDT",
%%%   author    = "Michael J Downes and David M. Jones",
%%%   address   = "American Mathematical Society,
%%%                Publications Technical Group,
%%%                PO Box 6248,
%%%                Providence, RI 02940,
%%%                USA",
%%%   email     = "tech-support@ams.org",
%%%   URL       = "http://www.ams.org/",
%%%   abstract  = "A LaTeX package that permits bibliography style
%%%                to be controlled completely from the LaTeX side
%%%                instead of being determined chiefly by the BibTeX
%%%                style file.",
%%%   checksum  = "09127 8030 25325 242081",
%%%   docstring = "The checksum field, produced by Robert Solovay's
%%%                checksum utility, gives CRC-16 checksum, lines,
%%%                words, and characters.",
%%%  }
%%% ====================================================================
%%%
% \iffalse
%<*driver>
\NeedsTeXFormat{LaTeX2e}
\documentclass[draft,oneside]{amsdtx}

\setcounter{secnumdepth}{3}
\setcounter{tocdepth}{3}

\DoNotIndex{\bib,\BibSpec,\@apply,\citesel,\cite@cj,\cite}
\DoNotIndex{\IfOption,\add@toks@,\@emptytoks,\@temptokenb}

\makeatletter
\DeclareRobustCommand{\fld}{\category@index{field}}
\DeclareRobustCommand{\btype}{\category@index{entry type}}
\DeclareRobustCommand{\attr}{\category@index{attribute}}
\makeatother

%\OnlyDescription

\usepackage{amsrefs}
\usepackage{multicol}

\providecommand{\lat}[1]{\protect\LaTeX{}}

\DeclareTextSymbol{\lbracechar}{OT1}{123}
\DeclareTextSymbolDefault{\lbracechar}{OT1}
\DeclareTextSymbol{\rbracechar}{OT1}{125}
\DeclareTextSymbolDefault{\rbracechar}{OT1}

\newcommand{\rdag}{{\normalfont\dag}}

\providecommand{\embrace}[1]{%
  \begingroup \ntt\lbracechar#1\rbracechar\endgroup}

% One mandatory argument: cnm; two: cnmm; one optional and one
%    mandatory: cnom; throwing in a star: cnsom; and so on.
\providecommand{\cnbreak}{\penalty999\hskip0pt\relax}
\providecommand{\cnm}[2]{\cn{#1}\cnbreak\embrace{#2}}
\providecommand{\cnmm}[3]{\cn{#1}\cnbreak\embrace{#2}\cnbreak\embrace{#3}}
\providecommand{\cnom}[1]{\cn{#1}\cnbreak\ommitude}
\providecommand{\ommitude}[2][]{{\ntt[#1]}\cnbreak\embrace{#2}}
\providecommand{\cnmsm}[4]{%
  \cn{#1}\cnbreak\embrace{#2}\cnbreak *\embrace{#4}}

\newcommand{\rpack}[1]{\pkg{amsrefs} package}

\newcommand{\MacroArgs}{%
    \par
    \medskip \noindent\emph{Arguments}: \par}

\newcommand{\macroarg}[2]{
    \texttt{\arg{#1} <- }\emph{#2}\unkern\texttt{.}\par}

\newenvironment{bug}{%
    \quote
        \emph{Known bug:}}
    {\endquote}

% non-stupid hline: support for adding some space below
\newcommand{\nshline}{\hline\omit\\}

\newcommand{\SeveralOptions}[1]{%
  \vbox{\def\\{\unskip,\egroup\hbox\bgroup\ignorespaces}%
    \hbox{\strut#1}}}

% Make this a self-contained document without separate .bbl file.
\newcommand{\PrintBibliography}{%
  \begin{bibsection}\begin{biblist}\raggedright
  \bib{Jones2004}{article}{
    author={Jones, David M.},
    title={User's Guide to the \pkg{amsrefs} Package},
    note={distributed with the \pkg{amsrefs} code}
  }
  \bib{SOS99}{book}{
    author={Swanson, Ellen},
    author={O'Sean, Arlene},
    author={Schleyer, Antoinette},
    title={Mathematics into Type},
    edition={updated},
    year={1999},
    publisher={American Mathematical Society},
  }
  \end{biblist}\end{bibsection}
}

\makeatletter

\def\nametest#1{%
    \par
    \begingroup
        \let\EmptyNameWarning\relax
        \vdef\@tempa{#1}%
        {\tt author=\embrace\@tempa}:
        \name@split#1,,,\@nil
        G=\@showname{given}
        S=\@showname{surname}
        J=\@showname{jr}%
    \endgroup
}

\def\@showname#1{%
    {\ntt\lbracechar}%
    {\tt\@xp\macrotext\csname bib'#1\endcsname}%
    {\ntt\rbracechar}
}

\def\initialtest#1{%
    \par
    \begingroup
        \vdef\@tempa{#1}%
        {\tt author=\embrace\@tempa}:
        \name@split#1,,,\@nil
        I={\ntt\lbracechar}\csname bib'initials\endcsname{\ntt\rbracechar}%
    \endgroup
}

\makeatother

\CodelineIndex

\begin{document}
\title{The \pkg{amsrefs} package}
\author{Michael Downes and David M. Jones\\American Mathematical Society}
\date{Version \fileversion, \filedate}
\hDocInput{amsrefs.dtx}
\PrintIndex
\end{document}
%</driver>
% \fi
%
% \MakeShortVerb{\|}
%
% \maketitle
%
%   \tableofcontents
%
% \section{Introduction}
%
%    The \rpack/ is a \lat/ package for bibliographies that provides an
%    archival data format similar to the format of \bibtex/ database
%    files, but adapted to make direct processing by \lat/ easier. The
%    package can be used either in conjunction with \bibtex/ or as a
%    replacement for \bibtex/.
%
%    This document is written for anyone who wants to implement a new
%    bibliography style for \pkg{amsrefs} or who is just curious about
%    how the package is implemented.
%    The reader should be familiar with the contents of the ``User's
%    Guide to the \pkg{amsrefs} Package''~\cite{Jones2004}
%    (\fn{amsrdoc.tex}).
%
%    For the publisher or implementor, the chief advantages of the
%    \rpack/ are as follows:
%    \begin{description}
%
%    \item[Preservation of structure]
%
%    The internal structural information of the bibliography entries is not
%    lost when they are imported from the database file into the \lat/
%    document. This takes on its greatest significance when archiving
%    documents in \lat/ form or transmitting them to another user (such as a
%    publisher).
%
%    \item[Deferred formatting]
%
%    This means that the style of the bibliography can be readily changed
%    without reimporting everything from the original database(s).
%
%    \item[Setup requires only \lat/ knowledge]
%
%    All bibliography setup can be done in \lat/; learning another
%    programming language (such as the one used in \bibtex/ \fn{bst}
%    files) is unnecessary.
%
%    \end{description}
%
% \section{Package options}
%
%    In addition to the options documented in the user's guide, there
%    are a few additional options that were omitted either because
%    they are obsolete or deprecated options included only for
%    backwards compatability or because they are still considered
%    experimental and not yet ready for widespread use.
%
%    \begin{description}
%
%    \item[?]
%
%    Informational option. This causes \pkg{amsrefs} to
%    display a pointer to the User's Guide on the terminal an in the
%    log file.  (In previous versions, it displayed much more
%    material, including a summary of package options.)
%
%    \item[traditional-quotes, logical-quotes]
%
%    With the \textit{traditional quotes} option
%    (default), quotation marks produced by \cn{bibquotes}
%    (\secref{misc-cmds}) fall outside of other punctuation, ``like
%    this,'' whereas with the \textit{logical quotes} option the order
%    is reversed, ``like this''.
%
%    \item[beta, jpa]
%
%    Obsolete; these applied only to the beta version of the
%    \pkg{amsrefs} package.
%
% \end{description}
%
% \section{More about the \cn{bib} command}\label{bib-command}
%
% \subsection{Field names for the \cn{bib} command}
%
%    In addition to the fields discussed in the user's guide, the
%    following fields are used internally:
%    \begin{description}
%
%   \item[fulljournal]
%
%    Used internally by \cn{DefineJournal}.
%
%    \item[name]
%
%    Used internally by the \btype{name} bibliography type and
%    \cn{DefineName}.
%
%    \item[transition]
%
%    A dummy field used inside \cs{BibSpec}s when we want to force an
%    action unconditionally.
%
%    \end{description}
%
%    The following fields are included for backwards compatibility:
%    \begin{description}
%
%    \item[institution, school]
%
%    These are provided as aliases for \fld{organization} for
%    compatibility with \BibTeX.
%
%    \item[place]
%
%    A synonym for \fld{address}.  In earlier versions of
%    \pkg{amsrefs}, \fld{place} was preferred and \fld{address} was
%    considered as an alias for \fld{place}.  However, this seemed
%    like a gratuitous incompatibility with \BibTeX\ to me, so I have
%    reinstated \fld{address} as the primary field and \fld{place} is
%    now an undocumented alias.
%
%    \end{description}
%
%    The following fields are reserved for future use:
%    \begin{description}
%
%    \item[doi] Digital Object Identifier
%
%    \item[setup]
%
%    This is a special field that can be used to give arbitrary
%    commands to be executed at the beginning of the current \cn{bib}
%    entry, after all the fields have been read. The idea is that one
%    can alter the formatting of an individual entry through this
%    field, to handle special cases.
%
%    This is fully implemented, but I've been unable to think of any
%    good examples of its use; so, I've decided to suppress it until
%    such an example comes to light.
%
%    \item[url] Universal Resource Locator.
%
%    \end{description}
%
% \subsection{Bibliography entry types}
%
%    The following additional entry types (or, really, pseudo-entry
%    types) are used internally by \pkg{amsrefs}:
%    \begin{description}
%
%    \item[collection.article]
%
%    \item[proceedings.article]
%
%    \item[partial]
%
%    \item[conference]
%
%    \item[innerbook]
%
%    \item[name]
%
%    \item[nameLE]
%
%    \item[nameBE]
%
%    \item[nameinverted]
%
%    \item[publisher]
%
%    \end{description}
%
%    The following are currently undocumented aliases for various of
%    the standard types:
%    \begin{description}
%
%    \item[miscellaneous]
%
%    \item[periodical]
%
%    \end{description}
%
% \section{Customizing the bibliography style}\label{other-styles}
%
%    If you use the \rpack/ as is, the bibliography style you get is the
%    kind of style customarily seen in AMS publications. The recommended
%    way to get a different bibliography style is to write a \lat/
%    package which loads the \rpack/ with \cn{RequirePackage} and
%    then makes the desired changes by using suitable \cn{BibSpec}
%    commands as explained below. Thus, the general form of the custom
%    package will be
%    \begin{verbatim}
%    \ProvidesPackage{xyzbib}[2002/11/06 v1.28]
%
%    \RequirePackage{amsrefs}\relax
%
%    \BibSpec{article}{
%      ...
%    }
%
%    \BibSpec{book}{
%      ...
%    }
%\end{verbatim}
%
%    The interior formatting within entries is specified by \cn{BibSpec}
%    commands, one for each entry type. To illustrate, let's look at an
%    example style specification for entries of type \btype{article}:
%    \begin{verbatim}
%    \BibSpec{article}{%
%      +{}{\PrintAuthors}  {author}
%      +{,}{ \textit}      {title}
%      +{,}{ }             {journal}
%      +{}{ \textbf}       {volume}
%      +{}{ \parenthesize} {date}
%      +{,}{ }             {pages}
%      +{,}{ }             {note}
%      +{.}{}              {transition}
%      +{}{ }              {review}
%    }
%\end{verbatim}
%
%    It should be pretty obvious that each line specifies the formatting
%    for a particular field. After reading the data for a particular
%    \cn{bib} command, \lat/ steps through the style specification and
%    for each
%    field listed, prints the field with the given formatting \emph{if
%    and only if the field has a nonempty value}. The \verb'+' character
%    at the beginning of each field specification must be followed by three
%    arguments: the punctuation to be added if the field is nonempty;
%    space and/or other material to be added after the punctuation; and
%    the field name. It is permissible for the second part to end with a
%    command that takes an argument, such as \cn{textbf}, in which case
%    it will receive the field's value as its argument. By defining a
%    suitable command and using it here you can place material after the
%    field contents as well as before; \cn{parenthesize} is an example
%    of this.
%
%    The reason that the punctuation and the following space are specified
%    separately is that between them there is a crucial boundary for line
%    breaks. If you put a \cn{linebreak} command at the end of a field value,
%    the break point will actually be carried onward to a suitable point after
%    the next bit of punctuation (whose actual value may vary depending on
%    which of the following fields is the first to turn up with a nonempty
%    value).
%
%    The meaning of the \cn{parenthesize} command, supplied by
%    \pkg{amsrefs}, should be obvious. The meaning of the \cn{PrintAuthors}
%    command is a different story. But I don't think it is all that hard to
%    understand. If we have two or more author names which were given
%    separately,
%    and we need to combine them into a conventional name list using commas
%    and the word \qq{and}, then it would be nice if we had a command which
%    could take a list of names and Do The Right Thing. And that is just
%    what \cn{PrintAuthors} is.
%
%    The \pkg{rkeyval} package allows keys to be defined as additive: if the
%    key occurs more than once, each successive value will be concatenated to
%    the previous value, along with a prefix. The setup done by
%    \pkg{amsrefs} for the \fld{author} field is
%    \begin{verbatim}
%    \DefineAdditiveKey{bib}{author}{\name}
%\end{verbatim}
%    This means that if two names are given, as in
%    \begin{verbatim}
%    author={Bertram, A.},
%    author={Wentworth, R.},
%\end{verbatim}
%    then the final value of the \fld{author} field seen when \lat/
%    processes the style specification will be
%    \begin{verbatim}
%    \name{Bertram, A.}\name{Wentworth, R.}
%\end{verbatim}
%
%    The \fld{transition} field in our \cn{BibSpec} example is a dummy
%    field to be
%    used when punctuation or other material must be added at a certain point
%    in the bibliography without regard to the emptiness or non-emptiness of
%    the fields after it. The \fld{transition} field always tests as non-empty
%    but has no printed content. So when you use it you always get the
%    indicated punctuation and space at the indicated point in the list of
%    fields. If it were the last thing in this \cn{BibSpec} example, it could
%    serve just to put in the final period that is always wanted. But in AMS
%    bibliographies, if a \emph{Mathematical Reviews} reference is given, it is
%    conventionally printed \emph{after} the final period. Using the
%    \fld{transition} field as shown here ensures that the final period will
%    be always printed, even when the \fld{review} field is empty.
%
% \section{Miscellaneous commands provided by the \pkg{amsrefs}
%    package}\label{misc-cmds}
%
%    Most of the following commands are helper commands for use in
%    \cn{BibSpec} statements. The others are intended for use in bibliography
%    data.
%    \begin{description}
%
%    \item[\cn{parenthesize}]\SpecialUsageIndex{\parenthesize}
%    This command adds parentheses around its
%      argument. It is useful in \cn{BibSpec} statements because there is no
%      special provision for adding material after the field value.
%
%    \item[\cn{bibquotes}]\SpecialUsageIndex{\bibquotes}
%    This command is much like \cn{parenthesize} but
%      it adds quotes around its argument and it has one other important
%      difference: there are special arrangements to print the closing quote
%      \emph{after} a following comma or similar punctuation (unless the
%      \pkg{amsrefs} package is invoked with the \opt{logical-quotes} option,
%      in which case \cn{bibquotes} puts the closing quote immediately after
%      the quoted material).
%
%    \item[\cn{voltext}]\SpecialUsageIndex{\voltext}
%    This is used to format volume numbers.  By default, it precedes
%    the volume number by ``vol.''
%
%    \item[\cn{issuetext}]\SpecialUsageIndex{\issuetext}
%    This is used to format issue numbers.  By default, it precedes
%    the volume number by ``no.''
%
%    \item[\cn{editiontext}]\SpecialUsageIndex{\editiontext}
%    This command produces ``ed.''\ following an
%      edition number. See \cn{PrintEdition} for more information.
%
%    \item[\cn{DashPages}]\SpecialUsageIndex{\DashPages}
%    This command is similar in spirit to
%    \cn{voltext} but more complicated in its implementation. It takes
%    one argument which is expected to contain one or more page numbers
%    or a range of page numbers. The argument is printed with a prefix
%    of \qq{p.} if it seems to be a single page number, otherwise with a
%    prefix of \qq{pp.}.
%
%    \item[\cn{tsup}, \cn{tsub}, \cn{tprime}]\SpecialUsageIndex{\tsup}
%    \SpecialUsageIndex{\tsub}\SpecialUsageIndex{\prime}
%    These are for text subscripts
%      and superscripts, with \cn{tprime} producing a superscript prime
%      symbol. Unlike the standard \cn{textsuperscript} and
%      \cn{textsubscript} functions provided by \lat/, these do not use
%      math mode at all.\footnote{There is one drawback: If you don't
%      want to get the prime symbol for \cn{tprime} from the \fnt{cmsy}
%      font, you will need to redefine \cn{tprime} in some suitable
%      way.}
%
%    \item[\cn{nopunct}]\SpecialUsageIndex{\nopunct}
%    This command causes following punctuation to be omitted if it is
%    added with the internal function \cs{@addpunct}.
%
%    \item[\cn{PrintPrimary}]\SpecialUsageIndex{\PrintPrimary}
%    This is a relatively complicated function that determines the
%    ``primary'' contributors for an entry and formats them, or
%    replaces them by \cn{sameauthors} if appropriate.  It should be
%    used when an entry type might have editors or translators instead
%    of authors.  It prefers authors over editors and editors over
%    translators and generates a warning if there are no primary
%    contributors.
%
%    \item[\cn{PrintAuthors}]\SpecialUsageIndex{\PrintAuthors}
%    This is used to format the list of authors as the primary
%    contributors for an entry type.
%
%    \item[\cn{PrintEditorsA}]\SpecialUsageIndex{\PrintEditorsA}
%    This is similar to \cn{PrintAuthors} but adds |(ed.)| or |(eds.)|
%    following the editors.
%
%    \item[\cn{PrintEditorsB}]\SpecialUsageIndex{\PrintEditorsB}
%    This is similar to \cn{PrintEditorsA} but puts parentheses around
%    the entire list of editors.  It's used by, for example, the
%    \btype{article} type to print the editors of a
%    \btype{proceedings} or \btype{collection}.
%
%    \item[\cn{PrintEditorsC}]\SpecialUsageIndex{\PrintEditorsC}
%    Similar to \cn{PrintEditorsA} but precedes the editors by
%    |Edited by|.  It's used when the editors should be treated as
%    subsidiary contributors, rather than the primary contributor.
%
%    \item[\cn{PrintTranslatorsA}]\SpecialUsageIndex{\PrintTranslatorsA}
%    This is similar to \cn{PrintEditorsA} but adds |(trans.)|
%    following the translators.
%
%    \item[\cn{PrintTranslatorsB}]\SpecialUsageIndex{\PrintTranslatorsB}
%    This is similar to \cn{PrintEditorsB}.  It's not currently used,
%    but is provided for symmetry.
%
%    \item[\cn{PrintTranslatorsC}]\SpecialUsageIndex{\PrintTranslatorsC}
%    Similar to \cn{PrintEditorsC} but precedes the translators by
%    |Translated by|.
%
%    \item[\cn{sameauthors}]\SpecialUsageIndex{\sameauthors}
%    This is a function of one argument. If you use
%      the default set of \cn{BibSpec}s from the \pkg{amsrefs},
%      \cn{sameauthors} is applied to the author name for a given \cn{bib}
%      command if it matches exactly the author name of the preceding
%      \cn{bib} command. Change the definition of \cn{sameauthors} if you
%      don't want to get a bysame dash.
%
%    \item[\cn{bysame}]\SpecialUsageIndex{\bysame}
%    This is a horizontal rule of length 3~em. The default
%      definition of \cn{sameauthors} prints \cn{bysame} instead of the
%      author names.
%
%    \item[\cn{Plural}, \cn{SingularPlural}]\SpecialUsageIndex{\Plural}
%    \SpecialUsageIndex{\SingularPlural}
%    These are helper functions
%      that allow you to conditionally
%      print singular or plural forms such as \verb'(ed.)' or \verb'(eds.)'
%      depending on the number of names in the current name list.
%      The definition of \cn{PrintEditorsA} reads, in part,
%    \begin{verbatim}
%    ... (ed\Plural{s}.) ...
%\end{verbatim}
%
%    \item[\cn{PrintReviews}]\SpecialUsageIndex{\PrintReviews}
%    This is similar to \cn{AuthorList} but is used for
%      printing (possibly multiple) MR numbers given in the \fld{review}
%      field.
%
%    \item[\cn{BibField}]\SpecialUsageIndex{\BibField}
%    This is for more complicated programming tasks such
%    as may be necessary for some \cn{BibSpec}s.  It takes one
%    argument, a field name, and yields the contents of that field for
%    the current \cn{bib} entry.
%
%    \item[\cn{IfEmptyBibField}]\SpecialUsageIndex{\IfEmptyBibField}
%    If one writes
%    \begin{verbatim}
%    \IfEmptyBibField{isbn}{A}{B}
%\end{verbatim}
%    then the commands in A will be executed if the \fld{isbn} field is empty,
%    otherwise the commands in B.
%
%    \item[\cn{PrintEdition}]\SpecialUsageIndex{\PrintEdition}
%    If a bibliography entry has
%    \begin{verbatim}
%      edition={2}
%\end{verbatim}
%    and the \cn{BibSpec} used \cn{PrintEdition} to handle this field, then the
%    edition information will be printed as \qq{2nd ed.}\mdash that is, the
%    number is converted to cardinal form and \qq{ed.} is added (taken from
%    \cn{editiontext}).
%
%    \item[\cn{CardinalNumeric}]\SpecialUsageIndex{\CardinalNumeric}
%    This provides the conversion to cardinal
%      number form used by \cn{PrintEdition}.
%
%    \item[\cn{PrintDate}, \cn{PrintYear}]\SpecialUsageIndex{\PrintDate}
%    \SpecialUsageIndex{\PrintYear}
%    These functions convert a date in
%      canonical form (ISO 8601) to the form required by the current
%      bibliography style.  You can get your preferred date form by
%      redefining these functions or by changing your \cn{BibSpec} statements
%      to use another function of your own devising. The original definition
%      of \cn{PrintDate} adds parentheses (as for the year of a journal article
%      in normal AMS style), whereas the \cn{PrintYear} function simply
%      prints the year without any additional material (as for a book's year
%      of publication in normal AMS style).
%
%    \item[\cn{mdash}, \cn{ndash}]\SpecialUsageIndex{\mdash}
%    \SpecialUsageIndex{\ndash}
%      These are short forms for \cn{textemdash}
%      and \cn{textendash}, recommended instead of the more usual \verb'---'
%      and \verb'--' notation. From the \pkg{textcmds} package.
%
%    \item[et cetera \dots] [mjd,2002-01-03] See the \fn{.dtx} files for
%    further possibilities that I have not managed to get properly
%    documented yet!
%    \end{description}
%
% \StopEventually{\PrintBibliography}
% \iffalse
% \clearpage
% \typeout{@@@ That was the last page of the OnlyDescription}
% \fi
%
%%%%%%%%%%%%%%%%%%%%%%%%%%%%%%%%%%%%%%%%%%%%%%%%%%%%%%%%%%%%%%%%%%%%%%%%
% \section{Implementation}
%
% \subsection{Overview}
%
%    It will be a while yet before we get to any actual code. First we
%    need to understand what the code needs to accomplish in order to
%    provide the user interface described above in a way that is as
%    compatible as possible with existing \lat/ mechanisms.
%
% \subsubsection{Normal \lat/ processing of cites}
%
% \paragraph{First \lat/ pass}
%
%    Various commands are written to the \fn{.aux} file that are
%    mostly used by \bibtex/.
% \begin{enumerate}
% \item A \cnm{cite}{moo} command writes one line to the \fn{.aux} file:
%    \cnm{citation}{moo}. This indicates to \bibtex/ that it should
%    include `moo' in the list of cited items to be searched for.
%    The \cn{cite} command also checks to see if \cs{b@moo} contains
%    the corresponding citation label, but since this is the first
%    pass, the label won't be known yet, so \lat/ emits an `Undefined
%    citation' warning and prints a placeholder (i.e., \textbf{???})
%    instead of the citation label.
%
% \item A \cnm{bibliographystyle}{har} command writes one line to the
%    \fn{.aux} file: \cnm{bibstyle}{har}. This indicates to \bibtex/ that it
%    should use \fn{har.bst} to determine the style for sorting and
%    formatting the bibliography items.
%
% \item A \cnm{bibliography}{hij,klm,...} command writes one line to the
%    \fn{.aux} file: \cnm{bibdata}{hij,klm,...}. This indicates to \bibtex/
%    that it should look in \fn{hij.bib}, \fn{klm.bib}, \dots\ for
%    bibliographic data. The \cn{bibliography} also tries to input
%    the \fn{.bbl} file, but on the first pass it won't exist
%    yet.
%
% \end{enumerate}
%
%    On the first pass all \cn{cite}'s normally are reported as
%    undefined because the \fn{.bbl} file has not yet been created.
%
% \paragraph{\bibtex/ pass}
%
%    For a document named \fn{xyz.tex}, the command \verb'bibtex xyz' is
%    used to invoke \bibtex/. It looks in \fn{xyz.aux} to find the
%    citation information written there by \lat/. For each \cn{citation} line,
%    \bibtex/ searches for a corresponding entry in the specified
%    \fn{.bib} files and formats it. The entire list is then sorted in
%    whatever way dictated by the bibliography style, and written out to
%    the file \fn{xyz.bbl}. This normally produces entries that look
%    something like:
%    \begin{verbatim}
%    \bibitem{BGL} P. Busch, M. Grabowski and P. J. Lahti:
%    {\it Operational Quantum Physics.}
%    Springer Verlag, New York (1995).
%\end{verbatim}
%
% \paragraph{Second \lat/ pass}
%
%    Now the \fn{.bbl} file exists and contains some \cn{bibitem}
%    commands. At \cnm{begin}{document}, \lat/ reads the \fn{.aux} file,
%    hoping to find some \cn{bibcite} commands, but it will not find
%    them until the next time around. \cn{citation}, \cn{bibstyle}, and
%    \cn{bibdata} commands in the \fn{.aux} file are simply ignored by \lat/.
%    Then \lat/ proceeds to typeset the body of the document.
% \begin{enumerate}
% \item Instances of \cn{cite} still print question marks.
%
% \item The \cn{bibliography} command causes \lat/ to input \fn{xyz.bbl}
%    and typeset its contents.
%
% \item A \cnm{bibitem}{moo} command writes one line to the
%    \fn{.aux} file: \cnmm{bibcite}{moo}{9}, where 9 is the current item
%    number.
%
% \item A \cnom{bibitem}[Moody]{moo} command writes one line to the
%    \fn{.aux} file: \cnmm{bibcite}{moo}{Moody}, using the supplied label
%    instead of a number.
%
% \end{enumerate}
%
% \paragraph{Third \lat/ pass}
%
%    Now the \fn{.aux} file contains some \cn{bibcite} commands. Once again,
%    \lat/ reads the \fn{.aux} file when it reaches \cnm{begin}{document}.
% \begin{enumerate}
%
% \item A \cnmm{bibcite}{moo}{Moody} causes \lat/ to define \cs{b@moo}
%    with `Moody' as the replacement text.
%
% \item If two \cn{bibcite} commands have the same citation key, \lat/
%    gives a warning message. This happens at \cnm{begin}{document},
%    during the reading of the \fn{.aux} file.
%
% \item Instances of \cn{cite} in the body of the document will print
%    the appropriate labels obtained from the \fn{.aux} file.
%
% \item If there are any \cn{cite} commands for which the \fn{.aux} file did
%    not have a \cn{bibcite} command, \lat/ will give an `Undefined
%    citation' warning. This often happens if the \fn{.aux} file is incomplete
%    due to a \TeX{} error on the preceding pass.
% \end{enumerate}
%
% \subsection{How cites are processed by \pkg{amsrefs}}
%
%    In order to support its additional features (e.g., author-year
%    citations and the \opt{backrefs} option), the \rpack/ stores additional
%    information for each cite in the macro \cs{b@whatever}. Instead of
%    simply using the defined or undefined status of this macro to
%    trigger the standard warnings, we add some boolean flags to allow
%    us to discriminate more finely what the current situation is.
% \begin{itemize}
% \item Each time an item is cited in the body of the document, a
%    \texttt{backref} entry is added to the info of that item. The
%    \texttt{backref} info
%    is the current page and section location. Section location is a bit
%    hard to get right without better support from the document class.
%    So we provide a hook to allow it to work better when the support is
%    there.
%
% \item When a cite occurs, if the info is undefined then a warning is
%    issued and the info structure is created. A \cs{citation} command
%    and a \cs{citedest} command (providing backref info) are written to
%    the \fn{.aux} file. Because the backref info includes page number, it has
%    to be a non-immediate write.
%
%    An undefined info structure would normally happen only on a first
%    pass when no \fn{.aux} file exists, or when a new cite is added.
%    I.e., when the corresponding \cs{citation} command is not yet
%    present in the \fn{.aux} file.
%
% \item When a citation command occurs in the \fn{.aux} file, it initializes
%    the info structure if necessary, setting the ``bib-info-present''
%    flag to 0.
%
% \item When a \cs{citedest} command occurs in the \fn{.aux} file, it
%    initializes the info structure if necessary---but this shouldn't
%    happen: if the corresponding \cs{citation} command did not already
%    get processed, then something is wrong. So normally, the
%    \cs{citedest} command merely needs to add its backref info to the
%    existing info structure.
%
% \item When a \cs{bibcite} command occurs in the \fn{.aux} file, it will
%    normally find that \cs{b@whatever} is already defined, if the
%    bibliography occurs after all the \cn{cite} commands. What it must
%    do is fill in the appropriate blank slots in the info structure
%    set up by a previous \cs{citation} command.
%
% \item The \fn{.aux} file is actually processed two times, once at the
%    beginning of the document and once at the end. In the latter case,
%    \cs{bibcite} should give a warning if the backref-list is empty,
%    since that means there were no \cn{cite} commands for the given
%    key.
%
% \item When processing the bibliography: The \cn{bib} command needs to
%    check if it is using a key that is already used by another \cn{bib}
%    command.
% \end{itemize}
%
%    We therefore have
%    \begin{verbatim}
%    \b@xyz -> \citesel 00{label}{year}{backref-list}
%\end{verbatim}
%    where the first~0 is replaced by~1 if there has already been another
%    citation for the same key earlier in the document (some citation
%    styles use abbreviated forms for all instances after the first),
%    and the second~0 is replaced by~1 if the same key was already
%    used by an earlier \cn{bib} command.
%
%    Because the backref-list often includes page number information, it
%    cannot be built on the fly as we go along; instead we have to write
%    the information to the \fn{.aux} file and read it in at the beginning of
%    the next run.
%
%    If there was no \cs{bibcite} in the \fn{.aux} file for a given key, then
%    the info is
%    \begin{verbatim}
%    \b@xyz -> \citesel 00{}{}{backref-list}
%\end{verbatim}
%    If there was neither \cs{citation} nor \cs{bibcite} in the \fn{.aux} file
%    for a given key, then the \cn{cite} command should find that
%    \ncn{b@xyz} is undefined.
%
%    If the author-year option is in effect, the \qq{label} contains
%    the author last names instead of a label:
%    \begin{verbatim}
%    \b@xyz -> \citesel 00{\name{Smith}\name{Jones}}{...}{...}
%\end{verbatim}
%    Full name information is included in the data because some citation
%    styles give full names at the first citation and abbreviated forms
%    for subsequent instances.
%
% \subsection{Data structures}
%
%    The result of scanning the key/value pairs of a \cn{bib} command is
%    an assignment statement for \cs{rsk@toks}. (Cf.\@ the \pkg{rkeyval}
%    package.) For example, consider the entry
%    \begin{verbatim}
%    \bib{miller83}{article}{
%      author={Miller, G.},
%      title={Eine Bemerkung zur Darstellung von Polynomen \"{u}ber
%        Verb\"{a}nden}*{language={german}},
%      journal={J. Math. Sent.},
%      volume={10},
%      year={1983},
%      pages={26\ndash 30},
%    }
%\end{verbatim}
%    The scanned result is to assign
%    \begin{verbatim}
%    \global\rsk@toks{%
%       \set:bib'author{Miller, G.}{}%
%       \set:bib'title{Eine Bemerkung zur Darstellung von Polynomen
%           \"{u}ber Verb\"{a}nden}{language={german}}%
%       \set:bib'journal{J. Math. Sent.}{}%
%       \set:bib'volume{10}{}%
%       \set:bib'year{1983}{}%
%       \set:bib'pages{26\ndash 30}{}%
%    }
%\end{verbatim}
%    The code in the last arg of \cn{RestrictedSetKeys} then invokes
%    \cs{bib@exec} to do something with the value of \cs{rsk@toks}.
%    \begin{verbatim}
%    \bib@exec{miller83}{\the\rsk@toks}{\setbib@article}{}
%\end{verbatim}
%
% \subsection{Preliminaries}
%
%    \begin{macrocode}
%<*pkg>
%    \end{macrocode}
%
%    Standard declaration of package name and date.
%    \begin{macrocode}
\NeedsTeXFormat{LaTeX2e}[1995/12/01]
%    \end{macrocode}
%
%    \begin{macro}{\amsrefs@warning@nl}
%    \begin{macrocode}
\def\amsrefs@warning@nl{\PackageWarningNoLine{amsrefs}}
%    \end{macrocode}
%    \end{macro}
%
%    Backward handling for beta version.
%    \begin{macrocode}
\@ifpackagewith{amsrefs}{beta}{%
    \amsrefs@warning@nl{The beta option is deprecated^^J%
    and will be removed in a future release of amsrefs}
    \expandafter\edef\csname opt@amsrbeta.sty\endcsname
        {\@ptionlist{amsrefs.sty}}%
    \def\@currname{amsrbeta}%
    \expandafter\let\csname amsrbeta.sty-h@@k\endcsname\@empty
    \def\@tempa{%%
%% This is file `amsrefs.sty',
%% generated with the docstrip utility.
%%
%% The original source files were:
%%
%% amsrefs.dtx  (with options: `pkg')
%% 
%%% ====================================================================
%%% @LaTeX-doc-source-file{
%%%   filename  = "amsrefs.dtx",
%%%   version   = "1.01",
%%%   date      = "2000/04/07",
%%%   time      = "14:13:41 EDT",
%%%   author    = "Michael J Downes",
%%%   address   = "American Mathematical Society,
%%%                Publications Technical Group,
%%%                PO Box 6248,
%%%                Providence, RI 02940,
%%%                USA",
%%%   email     = "tech-support@ams.org",
%%%   URL       = "http://www.ams.org/",
%%%   abstract  = "A LaTeX package that permits bibliography style
%%%                to be controlled completely from the LaTeX side
%%%                instead of being determined chiefly by the BibTeX
%%%                style file.",
%%%   checksum  = "13215 2240 8648 73862",
%%%   docstring = "The checksum field, produced by Robert Solovay's
%%%                checksum utility, gives CRC-16 checksum, lines,
%%%                words, and characters.",
%%%  }
%%% ====================================================================
\NeedsTeXFormat{LaTeX2e}[1995/12/01]
%\ProvidesPackage{amsrefs}[2000/04/07 v1.03]
\ProvidesPackage{amsrbeta}[2000/04/07 v1.03]
\begingroup
\edef\c{\number\catcode\string`\`}
\catcode\string`\`=12
\edef\c{%
  \endgroup
  \noexpand\AtEndOfPackage{%
    \catcode\number`\==\number\catcode`\=
    \catcode\number`\:=\number\catcode`\:
    \catcode\number`\`=\c\relax
  }%
  \catcode\number`\`=12
}
\c \catcode`\'12 \catcode`\=12 \catcode`\:12
\let\@xp=\expandafter
\let\@nx=\noexpand
\long\def\@ifempty#1{\@xifempty#1@@..\@nil}
\long\def\@xifempty#1#2@#3#4#5\@nil{%
  \ifx#3#4\@xp\@firstoftwo\else\@xp\@secondoftwo\fi}
\long\def\@ifnotempty#1{\@ifempty{#1}{}}
\@ifundefined{@emptytoks}{\csname newtoks\endcsname\@emptytoks}{}
\def\addto@macro#1#2{\@xp\def\@xp#1\@xp{#1#2}}
\let\star@char=*
\def\star@#1#2{%
  \def\star@a##1{#1}\def\star@b{#1{#2}}%
  \futurelet\@let@token\star@test
}
\def\star@test{%
  \ifx\@let@token\star@char \let\star@b\star@a\fi
  \star@b
}
\RequirePackage{ifoption}[2000/02/15]
\DeclareExclusiveOptions{*sorted,*citation-order}
\DeclareExclusiveOptions{alphabetic,author-year,numeric}
\DeclareBooleanOption{short-journals}
\DeclareBooleanOption{short-months}
\DeclareBooleanOption{initials}
\DeclareExclusiveOptions{traditional-quotes,logical-quotes}
\DeclareExclusiveOptions{sorted-cites,non-sorted-cites}
\DeclareExclusiveOptions{non-compressed-cites,compressed-cites}
\DeclareBooleanOption{jpa}
\DeclareBooleanOption{backrefs}
\def\amsrefs@verbose{01}
\DeclareOption{?}{\def\amsrefs@verbose{00}}
\DeclareOption{beta}{}
\ExecuteOptions{numeric,*sorted,traditional-quotes,%
  sorted-cites,compressed-cites}
\ProcessOptions\relax
\ProcessExclusiveOptions
\IfOption{beta}{}{%
  \PackageWarningNoLine{amsrefs}{%
    Option `beta' should be used with this beta release}%
}
%\RequirePackage{rkeyval}[2000/02/02]
\RequirePackage{rkvbeta}[2000/02/02]
\RequirePackage{inicap}[2000/02/02]
\newenvironment{biblist}[1][]{%
  \normalfont \footnotesize \labelsep .5em\relax
  \list{\@biblabel{\thebib}}{%
    \resetbiblist{00}%
    \usecounter{bib}}%
    \sloppy
  \interlinepenalty\@m  \clubpenalty\@M \widowpenalty\clubpenalty
  \frenchspacing
  #1\relax
}{%
  \def\@noitemerr{\@latex@warning{Empty bibliography list}}%
  \endlist
}
\def\@biblabel#1{[#1]}
\newcommand{\resetbiblist}[1]{%
  \settowidth\labelwidth{\@biblabel{#1}}%
  \leftmargin\labelwidth \advance\leftmargin\labelsep
}
\def\bib@start#1{%
  \begingroup
  \let\@elt\relax % work around a LaTeX output routine bug
  \item\leavevmode
}
\def\bib@end{\relax
  \par\endgroup
}
\providecommand{\name}{}
\DefineAdditiveKey{bib}{author}{\name}
\DefineAdditiveKey{bib}{editor}{\name}
\DefineSimpleKey{bib}{title}
\DefineSimpleKey{bib}{subtitle}
\DefineSimpleKey{bib}{booktitle}
\DefineSimpleKey{bib}{series}
\DefineSimpleKey{bib}{conference}
\DefineSimpleKey{bib}{publisher}
\DefineSimpleKey{bib}{organization}
\DefineSimpleKey{bib}{institution}
\DefineSimpleKey{bib}{address}
\DefineSimpleKey{bib}{place}
\DefineSimpleKey{bib}{year}
\DefineSimpleKey{bib}{date}
\DefineSimpleKey{bib}{journal}
\DefineSimpleKey{bib}{volume}
\DefineSimpleKey{bib}{number}
\DefineSimpleKey{bib}{pages}
\DefineSimpleKey{bib}{part}
\DefineAdditiveKey{bib}{isbn}{\isbn}
\DefineSimpleKey{bib}{issn}
\DefineSimpleKey{bib}{doi}
\DefineAdditiveKey{bib}{review}{\review}
\DefineSimpleKey{bib}{status}
\DefineSimpleKey{bib}{eprint}
\DefineSimpleKey{bib}{preprint}
\DefineSimpleKey{bib}{language}
\DefineSimpleKey{bib}{note}
\DefineSimpleKey{bib}{xid}
\DefineSimpleKey{bib}{label}
\DefineSimpleKey{bib}{type}
\DefineSimpleKey{bib}{edition}
\DefineSimpleKey{bib}{setup}
\DefineDummyKey{bib}{transition}
\newcommand{\bib}[3][]{%
  \begingroup
  \toks@{#1}%
  \def\@tempa{#2}%
  \edef\@tempb{%
    \@nx\@bibdef\@xp\@nx\csname setbib@#3\endcsname{\the\toks@}{#3}%
      {\@xp\strip@prefix\meaning\@tempa}%
  }%
  \@tempb
}
\let\biblio@list\@empty
\newcounter{bib}
\def\amsrefs@nocomma{\PackageWarning{amsrefs}{%
  Missing comma: some text might print incorrectly\MessageBreak}}
\def\@bibdef#1#2#3#4{%
  \ifx\relax#1%
    \PackageError{amsrefs}{Undefined entry type: #3}\@ehc
    \let#1\setbib@misc
  \fi
  \RestrictedSetKeys[\amsrefs@nocomma]{bib}%
    {\bib@exec{#4}{\the\rsk@toks}{#1}{#2}\endgroup}%
}
\begingroup \catcode`\'=11
\gdef\bib@field@patches{%
  \let\bib@author\bib'author
  \let\bib@editor\bib'editor
  \let\bib@date\bib'date
  \ifx\bib@date\@empty \let\bib@date\bib'year \fi
  \bib@parsedate
  \let\bib@language\bib'language
  \ifx\bib@language\@empty \let\bib@language\biblanguagedefault \fi
  \ifx\bib'place\@empty \let\bib'place\bib'address\fi
  \bib'setup
}
\endgroup
\def\bib@bibcite#1#2#3{%
  \@cons\biblio@list{{#1}}%
  \if\numeric@refs
    \stepcounter{bib}%
    \protected@edef\@currentlabel{\@nx\@nx\@nx\cite@x{\thebib}{}}%
  \else
    #2% execute definitions locally
    \bib@field@patches
    \def\@currentlabel{\@nx\cite@x{\bib@year}{\bib@author}}%
    \let\name\relax
  \fi
  \protected@edef\@tempa{\protect\bibcite{#1}{\@currentlabel}}%
  \@tempa
}
\def\bib@print#1#2#3#4{%
  \alpha@label{#4}%
  \bib@start{#1}%
  #2\bib@field@patches
  \bib@cite{#1}\kern\@ne sp\relax
  #3%
  \bib@end
}
\let\bib@exec\bib@bibcite
\AtBeginDocument{\let\bib@exec\bib@print}
\IfOption{author-year}{%
  \def\cite@label{%
    \ifx\bib@author\@empty \bib@editor \else \bib@author \fi
  }%
}{%
  \def\cite@label{\@currentlabel}%
}

\IfOption{alphabetic}{%
  \def\alpha@label{\def\thebib}%
}{%
  \let\alpha@label\@gobble
}
\def\bib@cite@check#1#2#3#4#5#6\@nil#7{%
  \ifodd0#2\relax
    \PackageWarning{amsrefs}{%
      Duplicate use of \protect\bib\space key `#7'\MessageBreak}%
  \else
    \@xp\gdef\csname b@#7\endcsname{\citeselect #11{#3}{#4}{#5}}%
  \fi
}
\def\bib@cite#1{%
  \protected@write\@auxout {\def\name{\protect\name}}{%
    \protect\bibcite{#1}{%
      \protect\citeselect 00{\cite@label}{\bib@year}{}}%
  }%
  \@ifundefined{b@#1}{%
    \protected@edef\@tempa{%
      \gdef\@xp\@nx\csname b@#1\endcsname{%
        \protect\citeselect  01{\cite@label}{\bib@year}%
      }%
    }
    \@tempa
  }{%
    \begingroup
    \let\citeselect\bib@cite@check
    \csname b@#1\endcsname 00xxx\@nil{#1}%
    \endgroup
  }%
}
\let\@@testdef\@testdef
\def\@testdef #1{%
  \if b\@car#1x\@nil \@xp\test@bibcite
  \else \@xp\@@testdef
  \fi
  {#1}%
}
\def\test@bibcite#1#2#3{%
  \@xp\@xp\@xp\test@bibcite@a\csname b@#2\endcsname #3\@nil
}
\def\test@bibcite@a#1#2#3#4\citeselect#5#6#7\@nil{%
  \def\reserved@a{#4}\def\reserved@b{#7}%
  \ifx\reserved@a\reserved@b \else\@tempswatrue \fi
}
\def\printbibliography{%
  \bibsection
  \begin{biblist}%
  \let\@elt\bib@print
  \biblio@list
  \end{biblist}%
}
\renewenvironment{thebibliography}[1]{%
  \bibsection
  \biblist[\resetbiblist{#1}]%
}{%
  \endbiblist
}
\providecommand{\bibname}{Bibliography}
\@ifundefined{chapter}{%
  \newcommand{\bibsection}{\section*{\bibname}}%
}{%
  \newcommand{\bibsection}{\chapter*{\bibname}}%
}
\def\SwapBreak#1{%
  \@tempcnta\@MM
  \ifhmode
    \unskip
    \ifnum\lastpenalty=\z@
    \else \@tempcnta\lastpenalty \unpenalty
    \fi
  \fi
  \toks@{#1}%
  \edef\@tempa{%
    \@nx\deferredquoteslogical
    \ifnum\lastkern=\@ne
    \else
      \ifnum\spacefactor=\sfcode\@xp\@xp\@xp`\@xp\@car\string#1)\@nil
      \else \the\toks@
      \fi
    \fi
    \@nx\deferredquotes
    \ifnum\@tempcnta=\@MM \else \penalty\number\@tempcnta\space\fi
    \ifnum\lastkern=\@ne \ignorespaces \fi
  }%
  \@tempa
}
\newcount\series@index
\def\PrintSeries#1#2#3#4#5#6#7#8{%
  \begingroup
  \def\series@add@a{#2}%
  \def\series@add@b{\SwapBreak{#3}#4}%
  \def\series@add@c{\SwapBreak{#5}#6}%
  \def\series@add@d{\SwapBreak{#7}#8}%
  \PrintSeries@a{#1}%
}
\newcommand{\SingularPlural}[2]{#1}
\newcommand{\Plural}{\SingularPlural{}}
\def\PrintSeries@a#1#2#3#4{%
  \series@index\z@ \def#1{\advance\series@index\@ne \@gobble}%
  #3\relax
  \ifnum\series@index=\@ne \let\SingularPlural\@firstoftwo
  \else \let\SingularPlural\@secondoftwo
  \fi
  \chardef\series@total=\series@index \series@index=\z@
  \let#1\series@add
  #2#3#4\relax
  \endgroup
}
\def\series@add#1{%
  \advance\series@index\@ne
  \let\@tempa\relax
  \ifcase\series@index
  \or % material before name 1
    \let\@tempa\series@add@a
  \or % material before name 2
    \ifnum\series@total<\thr@@ \let\@tempa\series@add@b
    \else \let\@tempa\series@add@c
    \fi
  \else % material before name 3, 4, 5, ...
    \ifnum\series@index<\series@total \let\@tempa\series@add@c
    \else \let\@tempa\series@add@d
    \fi
  \fi
  \@tempa{#1}%
}
\def\bib@append#1#2#3{%
  \ifx\@empty#3\relax
  \else
    \ifx\relax#2\errmessage{#3=\relax}\fi
    \@temptokena{\SwapBreak{#1}#2}\toks@\@xp{#3}%
    \edef\@tempa{\the\@temptokena{\the\toks@}}%
    \@xp\@tempa
  \fi
  \relax
}
\let\endbracket\@empty
\let\bracket@stack\@empty
\def\push@bracket#1{%
  \xdef\bracket@stack{#1\bracket@stack}%
}
\def\pop@bracket{\iffalse{\fi
  \@xp\pop@bracket@a\bracket@stack \@empty}}
\def\pop@bracket@a#1{\leavevmode\/\upn{#1}\xdef\bracket@stack{\iffalse}\fi}
\providecommand{\upn}[1]{#1}
\newcommand{\parenthesize}[1]{%
  \leavevmode\push@bracket)\upn{(}#1\pop@bracket
}
\let\deferredquotes\@empty
\IfOption{logical-quotes}{%
  \def\deferredquoteslogical{\deferredquotes}%
}{%
  \let\deferredquoteslogical\relax
}
\newcommand{\bibquotes}[1]{%
  \textquotedblleft#1%
  \gdef\deferredquotes{%
    \global\let\deferredquotes\@empty \textquotedblright
  }%
}
\providecommand{\mdash}{\textemdash}
\providecommand{\ndash}{\textendash}
\providecommand{\MRhref}[1]{}
\@ifundefined{@addpunct}{%
  \def\@addpunct#1{%
    \relax\ifhmode\unskip\ifnum\spacefactor>\@m \else#1\fi\fi
  }
  \def\frenchspacing{\sfcode`\.1006\sfcode`\?1005\sfcode`\!1004%
    \sfcode`\:1003\sfcode`\;1002\sfcode`\,1001 }
}{}
\providecommand{\nopunct}{\spacefactor 1005 }
\def\bibspec@scan#1#2#3#4{%
  \toks@\@xp{\the\toks@\bib@append{#1}{#2}}%
  \edef\@tempa{%
    \toks@{\the\toks@ \@xp\@nx\csname bib'#3\endcsname}%
  }%
  \@tempa
  \ifx\@empty#4\@xp\@gobble % end the recursion
  \else \ifx +#4\else\bibspec@scan@error\fi
  \fi
  \bibspec@scan
}
\newcommand{\bibspec}[2]{%
  \toks@\@emptytoks
  \@ifnextchar{+}{\@xp\bibspec@scan\@gobble}{\bibspec@scan}#2%
    \@empty\@empty\@empty\@empty
  \@xp\edef\csname setbib@#1\endcsname{\the\toks@}%
}
\IfOption{initials}{%
  \def\set@forenames#1#2 #3{%
    \ignorespaces#1. \ifx\@empty#3\@xp\@gobbletwo\fi \set@forenames#3%
  }
}{%
  \def\set@forenames{\ignorespaces}
}
\def\uninvertName#1{\rnscan@a#1,,,\@nil}
\def\rnscan@a#1,#2,#3,#4\@nil{%
  \set@forenames#2 \@empty#1\unskip
  \@ifnotempty{#3}{, \ignorespaces #3}}
\def\UninvertedNames{%
  \PrintSeries{\name}%
    {\uninvertName}%
    {}{ and \uninvertName}%
    {,}{ \uninvertName}%
    {,}{ and \uninvertName}%
}
\newcommand{\aulist}[1]{\UninvertedNames{}{#1}{}}
\providecommand{\bysame}{\leavevmode\hbox to3em{\hrulefill}\thinspace}
\newcommand{\sameauthors}[1]{\bysame}
\def\PreviousAuthors{\relax}
\newcommand{\PrintAuthors}{%
  \ifx\PreviousAuthors\bib@author \@xp\sameauthors
  \else
    \global\let\PreviousAuthors\bib@author
    \@xp\aulist
  \fi
}
\newcommand{\PrintEditorsA}[1]{%
  \UninvertedNames{}{#1}{ (ed\Plural{s}.)}%
}
\newcommand{\PrintEditorsB}[1]{%
  \UninvertedNames{(}{#1}{\SwapBreak{,} ed\Plural{s}.)}%
}
\newcommand{\ReviewList}[1]{%
  \PrintSeries{\review}{}{,}{ }{,}{ }{,}{ }{}{#1}{}%
}
\newcommand{\voltext}{vol.~}
\newcommand{\pptext}[1]{%
  p\pp@scan@a#1@\ndash p@\ndash{\pp@scan#1@-p@-{}\@nil}\@nil.~#1%
}
\def\pp@scan#1-#2@-#3#4\@nil{#3}
\def\pp@scan@a#1\ndash#2@\ndash#3#4\@nil{#3}
\newcommand{\eprintpages}[1]{%
  #1\IfEmptyBibField{eprint}{}{\IfEmptyBibField{journal}{ pp.}{}}%
}
\def\ISSNlookup#1{%
  \@ifnotempty{#1}{%
    \ISSN@lu#1\@nil{#1}%
  }%
}
\def\ISSN@lu#1#2\@nil#3{%
  \def\@tempa{#3}%
  \ifcat A\@nx#1%
    \lowercase{\if#1}#1%
        \def\@tempb##1<#3>##2##3\@nil{\def\@tempc{##2}}%
      \@xp\@tempb\ISSN@list<#3>{}\@nil
      \ifx\@tempc\@empty \arf@unknown{journal}{#3}%
      \else \let\@tempa\@tempc
      \fi
    \fi
  \fi
  \@tempa
}
\def\arf@unknown#1#2{%
  \toks@{#2}%
  \PackageWarning{amsrefs}{Unknown #1 abbreviation '\the\toks@'}%
}
\def\PublisherLookup#1{\Publ@LU#1?\@nil{#1}}
\def\Publ@LU#1#2\@nil#3{%
  \ifcat A\@nx#1%
    \lowercase{\if#1}#1%
      \lowercase{\uppercase{\csname PN}.#3}\endcsname
    \else
      #3%
    \fi
  \fi
}
\def\biblanguageEnglish{English}
\let\biblanguagedefault\biblanguageEnglish
\edef\amsrefs@tmp{\catcode\number`\'=\number\catcode`\'\relax}
\catcode`\'=11
\newcommand{\EnglishInitialCaps}[1]{%
  \ifx\@empty\bib'language \let\bib'language\biblanguagedefault\fi
  \ifx\bib'language\biblanguageEnglish \inicap{#1}\fi
}
\amsrefs@tmp
\def\PrintThesisType#1{%
  \thesis@type#1?\@nil{#1}%
}
\def\thesis@type#1#2\@nil#3{%
  \ifx p#1Ph.D. Thesis%
  \else\ifx m#1Master's Thesis%
  \else #3\fi\fi
}
\newcommand{\BibField}[1]{\csname bib'#1\endcsname}
\newcommand{\IfEmptyBibField}{\rkvIfEmpty{bib}}
\newcommand{\PrintDOI}[1]{%
  DOI #1%
  \IfEmptyBibField{volume}{, (to appear in print)}{}%
}
\def\bib@parsedate{%
  \@xp\bib@parsedate@a\bib@date ---\@nil
}
\let\bib@year\@empty
\let\bib@month\@empty
\let\bib@day\@empty
\begingroup \catcode`\'=11
\gdef\bib@parsedate@a#1-#2-#3-#4\@nil{%
  \def\bib@year{#1}\def\bib@month{#2}\def\bib@day{#3}%
  \ifx\@empty\bib@month \let\bib'date\bib@year
  \else \def\bib'date{#1-#2-#3}%
  \fi
}
\endgroup
\newcommand{\PrintDatePV}[1]{%
  \IfEmptyBibField{doi}{\let\@tempa\PrintDate}{%
    \IfEmptyBibField{volume}{\let\@tempa\PrintDatePosted}{%
      \let\@tempa\PrintDate}}%
  \@tempa{#1}%
}
\newcommand{\PrintDate}[1]{(\print@date)}
\def\print@date{%
  \ifx\bib@month\@empty \else\print@month@day \fi
  \bib@year
}
\def\print@month@day{%
  \ifcase 0\bib@month\ignorespaces
  \or January\or February\or March\or April\or May\or June\or
      July\or August\or September\or October\or November\or December\or
      Winter\or Spring\or Summer\or Fall\else Unknown Month%
  \fi
  \ifx\@empty\bib@day \else \space\number 0\bib@day,\fi
  \space
}
\newcommand{\PrintYear}[1]{\bib@year}
\newcommand{\PrintDatePosted}[1]{\unskip,  posted on \print@date}
\newcommand{\CardinalNumeric}[1]{%
  \number#1\relax
  \if\ifnum#1<14 \ifnum#1>\thr@@ T\else F\fi\else F\fi T%
    th%
  \else
    \@xp\keep@last@digit\@xp#1\number#1\relax
    \ifcase#1th\or st\or nd\or rd\else th\fi
  \fi
}
\def\keep@last@digit#1#2{%
  \ifx\relax#2\@xp\@gobbletwo
  \else #1=#2\relax
  \fi
  \keep@last@digit#1%
}
\newcommand{\PrintEdition}[1]{%
  \afterassignment\print@edition
  \count@ 0#1\relax\@nil
}
\newcommand{\editiontext}{ed.}
\def\print@edition#1#2\@nil{%
  \ifx\relax#1\relax
    \ifnum\count@>\z@ \CardinalNumeric\count@
    \else ??th%
    \fi
    \ \editiontext
  \else \ifnum\count@>\z@\number\count@\fi #1#2\relax
  \fi
}
\newcommand{\intersentencespace}{{\spacefactor3000} }
\newcommand{\eprint}[1]{#1}
\bibspec{article}{%
  +{}{\PrintAuthors}       {author}
  +{,}{ \textit}      {title}
  +{:}{ \textit}      {subtitle}
  +{.}{ }             {part}
  +{,}{ \ISSNlookup}  {journal}
  +{} { \textbf}      {volume}
  +{} { \PrintDatePV} {date}
  +{,}{ \PrintDOI}    {doi}
  +{,}{ \eprint}      {eprint}
  +{} { \parenthesize}{status}
  +{,}{ \eprintpages} {pages}
  +{,}{ }             {note}
  +{.}{}              {transition}
  +{} {\intersentencespace \ReviewList}   {review}
}
\bibspec{book}{%
  +{}{\PrintAuthors}{author}
  +{}{\PrintEditorsA}{editor}
  +{,}{ \textit}{title}
  +{:}{ \textit}{subtitle}
  +{,}{ }{type}
  +{,}{ \EnglishInitialCaps}{booktitle}
  +{,}{ \PrintEdition}{edition}
  +{,}{ }{series}
  +{,}{ vol.~}{volume}
  +{,}{ part~}{part}
  +{,}{ \PublisherLookup}{publisher}
  +{,}{ }{organization}
  +{,}{ }{place}
  +{,}{ }{date}
  +{,}{ ISBN }{isbn}
  +{}{ \parenthesize}{status}
  +{,}{ }{note}
  +{.}{}{transition}
  +{}{\intersentencespace \ReviewList}{review}
}
\bibspec{incollection}{%
  +{}{\PrintAuthors}{author}
  +{,}{ \textit}{title}
  +{:}{ \textit}{subtitle}
  +{,}{ }{type}
  +{,}{ \EnglishInitialCaps}{booktitle}
  +{,}{ \PrintEdition}{edition}
  +{}{ \PrintEditorsB}{editor}
  +{,}{ Proc.\ }{conference}
  +{,}{ }{series}
  +{,}{ vol.~}{volume}
  +{,}{ part~}{part}
  +{,}{ \PublisherLookup}{publisher}
  +{,}{ }{organization}
  +{,}{ }{place}
  +{,}{ }{date}
  +{,}{ \pptext}{pages}
  +{,}{ ISBN }{isbn}
  +{}{ \parenthesize}{status}
  +{,}{ }{note}
  +{.}{}{transition}
  +{}{\intersentencespace \ReviewList}{review}
}
\let\setbib@inproceedings\setbib@incollection
\let\setbib@booktitle\setbib@incollection
\let\setbib@misc\setbib@book
\let\setbib@unpublished\setbib@book
\bibspec{techreport}{%
  +{}{\PrintAuthors}{author}
  +{}{\PrintEditorsA}{editor}
  +{,}{ \textit}{title}
  +{:}{ \textit}{subtitle}
  +{,}{ }{type}
  +{,}{ \EnglishInitialCaps}{booktitle}
  +{,}{ Technical Report }{number}
  +{,}{ }{series}
  +{,}{ vol.~}{volume}
  +{,}{ part~}{part}
  +{,}{ \PublisherLookup}{publisher}
  +{,}{ }{organization}
  +{,}{ }{institution}
  +{,}{ }{place}
  +{,}{ }{date}
  +{,}{ ISBN }{isbn}
  +{,}{ \eprint}      {eprint}
  +{}{ \parenthesize}{status}
  +{,}{ }{note}
  +{.}{}{transition}
  +{}{\intersentencespace \ReviewList}{review}
}
\bibspec{thesis}{%
  +{}{\PrintAuthors}{author}
  +{}{\PrintEditorsA}{editor}
  +{,}{ \textit}{title}
  +{:}{ \textit}{subtitle}
  +{,}{ \PrintThesisType}{type}
  +{,}{ part~}{part}
  +{,}{ }{organization}
  +{,}{ }{institution}
  +{,}{ }{place}
  +{,}{ }{date}
  +{}{ \parenthesize}{status}
  +{,}{ }{note}
  +{.}{}{transition}
  +{}{\intersentencespace \ReviewList}{review}
}
\DeclareRobustCommand{\tsup}[1]{%
  \leavevmode\raise.9ex\hbox{\fontsize\sf@size\z@\selectfont #1}%
}
\DeclareRobustCommand{\tsub}[1]{%
  \leavevmode\lower.6ex\hbox{\fontsize\sf@size\z@\selectfont #1}%
}
\DeclareRobustCommand{\tprime}{\tsup{\usefont{OMS}{cmsy}{m}{n}\char48 }}
\def\citeleft{[}
\def\citeright{]}
\def\citemid{,\penalty9999 \space}
\def\citepunct{,\penalty9999 \hskip.13em plus.1em minus.05em\relax}
\let\citeform\@iden
\def\@citeleft{\citeleft}
\def\citeselect@author#1#2#3#4#5{\UninvertedNames{#3}}
\def\citeselect@authoryear#1#2#3#4#5{\UninvertedNames{#3}, #4}
\def\citeselect@object#1#2#3#4#5{\UninvertedNames{#3} \citeleft#4}
\def\citeselect@year#1#2#3#4#5{#4}
\def\citeselect@key#1#2#3#4#5{#3}
\let\citeselect\citeselect@key
\DeclareRobustCommand{\cite}[1]{\star@{\cite@a\citeselect{#1}}{}}
\def\cite@multiple#1,#2#3\@nil{\ifx\relax#2F\else T\fi T}
\def\MultipleCiteKeyWarning#1{%
  \PackageWarning{amsrefs}{%
    Use of \string\cites\space is recommended\MessageBreak
    instead of \string\cite\space for multiple cites '#1'}%
  \global\let\MultipleCiteKeyWarning\@gobble
}
\def\cite@a#1#2{%
  \begingroup
  \def\@tempa{#2}%
  \toks@{\endgroup \cite@b#1}%
  \edef\@tempa{\the\toks@{\@xp\strip@prefix\meaning\@tempa}}%
  \@tempa
}
\def\cite@b#1#2#3{%
  \if\cite@multiple#2,\relax\@nil
    \MultipleCiteKeyWarning{#2}%
    \cites{#2}%
  \else
    \cite@c#1{#2}{#3}%
  \fi
}
\@ifundefined{@extra@b@citeb}{\let\@extra@b@citeb\@empty}{}
\def\cite@lookup#1{\csname b@#1\@extra@b@citeb\endcsname}
\def\undefined@cite#1{%
  \PackageWarning{amsrefs}{Undefined citation '#1'}%
  \texttt{#1}}
\newcommand{\citedest}[2]{}
\def\cite@c#1#2#3{%
  \begingroup
  \let\citeselect#1\relax
  \@citeleft
  \@xp\@xp\@xp\ifx\cite@lookup{#2}\relax \undefined@cite{#2}%
  \else \cite@lookup{#2}%
  \fi
  \if@filesw\immediate\write\@auxout{\string\citation{#2}}\fi
  \write\@auxout{\string\citedest{#2}{\thepage}}%
  \@ifnotempty{#3}{\citemid #3}%
  \citeright
  \endgroup
}
\newcommand{\cites}[1]{}
\def\cites#1{\begingroup
  \edef\@tempa{\zap@space#1 \@empty}%
  \edef\@tempa{\@xp\strip@prefix\meaning\@tempa}%
  \edef\@tempa{\endgroup
    \@nx\citelist@a{}{sorted}{%
      \@xp\cites@a\@tempa,{@\csname fi\@car\endcsname},\@nil
    }%
  }%
  \@tempa
}
\def\cites@a#1,{%
  \ifx @#1@\else \@nx\cite@c\@nx\citeselect{#1}{}\fi \cites@a
}
\newcommand{\citelist}{\@ifnextchar <{\citelist@a}{\citelist@a <{sorted}}}%
\def\NonNumericCiteWarning{%
  \PackageWarning{amsrefs}{%
    Unable to confirm that cite keys are numeric: not sorting%
  }%
}
\def\citelist@a#1#2#3{%
  \leavevmode
  \citeleft\nopunct % suppress first \citepunct
  \begingroup
  \def\citeleft{\@addpunct{\citepunct}}\let\citeright\ignorespaces
  \ifx\cite@sorted\relax #3\relax
  \else
    \@xp\ifx\csname cite@#2\endcsname\cite@sorted
      \ifx\citeselect\citeselect@key
        \cite@sorted@s #3\cite@sorted@e
      \else \NonNumericCiteWarning#3\relax
      \fi
    \else #3\relax
    \fi
  \fi
  \endgroup
  \citeright
}
\let\cite@das\@empty
\def\cite@sorted@s{%
  \begingroup \let\cite@print\cite@c
    \let\cite@c\cite@compress
    \begingroup \let\cite@c\cite@sorted \ignorespaces}
\def\cite@sorted@e{%
  \relax \@xp\endgroup\the\cite@toks \last@cite \endgroup
}
\newtoks\cite@toks
\def\cite@sorted#1#2#3{%
  \@tempcnta 0\cite@lookup{#2}\relax % highest number so far
  \cite@toks{\cite@c#1{#2}{#3}}%
  \edef\sort@cite@temp{\the\cite@toks}%
  \let\cite@c\sort@cite@b
  \ignorespaces
}
\def\sort@cite@b#1#2#3{%
  \@tempcntb 0\cite@lookup{#2}\relax
  \ifnum\@tempcntb>\@tempcnta
    \cite@toks\@xp{\the\cite@toks \cite@c#1{#2}{#3}}%
  \else
    \let\cite@c\sort@cite@c \cite@toks\@emptytoks
    \def\@tempb{\cite@toks\@xp{\the\cite@toks\cite@c#1{#2}{#3}}}%
    \sort@cite@temp \@tempb \let\cite@c\sort@cite@b
  \fi
  \edef\sort@cite@temp{\the\cite@toks}%
  \ignorespaces
}
\def\sort@cite@c#1#2#3{%
  \ifnum\@tempcntb<0\cite@lookup{#2}%
    \@tempb \let\@tempb\@empty
  \fi
  \cite@toks\@xp{\the\cite@toks \cite@c#1{#2}{#3}}%
  \ignorespaces
}
\let\last@cite\@empty
\def\cite@compress#1#2#3{%
  \cite@print#1{#2}{#3}\@tempcnta 0\cite@lookup{#2}\relax
  \let\cite@c\cite@compress@a
}
\def\cite@compress@a#1#2#3{%
  \advance\@tempcnta\@ne \@tempcntb 0\cite@lookup{#2}\relax
  \ifnum\@tempcnta=\@tempcntb
    \def\cite@dash{\textendash \nopunct \let\cite@dash\@empty}%
    \def\last@cite{\cite@print#1{#2}{#3}}%
    \let\cite@c\cite@compress@b
  \else
    \cite@print#1{#2}{#3}\@tempcnta\@tempcntb
  \fi
}
\def\cite@compress@b#1#2#3{%
  \advance\@tempcnta\@ne \@tempcntb 0\cite@lookup{#2}\relax
  \ifnum\@tempcnta=\@tempcntb
    \cite@dash
    \def\last@cite{\cite@print#1{#2}{#3}}%
  \else
    \last@cite \let\last@cite\@empty
    \cite@print#1{#2}{#3}\@tempcnta\@tempcntb
    \let\cite@c\cite@compress@a
  \fi
}
\IfOption{unsorted-cites}{\let\cite@sorted\relax}{}
\DeclareRobustCommand{\ycite}[1]{%
  \star@{\cite@a\citeselect@year{#1}}{}%
}
\DeclareRobustCommand{\ycites}{\cites}% not done yet
\DeclareRobustCommand{\ocite}[1]{%
  \star@{\cite@a\citeselect@object{#1}}{}%
}
\DeclareRobustCommand{\ocites}{\cites}% not done yet
\IfOption{author-year}{%
  \let\citeselect\citeselect@authoryear
  \def\citeleft{(}\def\citeright{)}%
  \def\@citeleft{\ifx\citeselect\citeselect@object\else\citeleft\fi}%
  \let\@biblabel\@gobble
}{%
  \let\ycite\cite \let\ocite\cite
  \let\ycites\cites \let\ocites\cites
}
\let\ISSN@list=\@empty
\newcommand{\DefineJournal}[1]{\lowercase{\DefJrn@a{#1}}}
\def\DefJrn@a#1#2#3#4{%
  \def\@tempa##1<#1>##2##3\@nil{%
    \ifx\@empty##3\@empty
    \else
      \PackageWarning{amsrefs}{Redefining journal '#1'->'#2'}%
    \fi
    \toks@{##1<#1>}\@temptokena{##3}%
    \xdef\ISSN@list{%
      \the\toks@\@xp\@nx\csname SN.#2\endcsname\the\@temptokena
    }%
  }%
  \edef\@tempb{%
    \@nx\@xp\@nx\@tempa\@nx\ISSN@list
      <#1>\@xp\@nx\csname SN.#2\endcsname\@nx\@nil
  }%
  \@tempb
  \global\@namedef{SN.#2}{\SerialISSNName{#1}{#3}{#4}}%
  \if\amsrefs@verbose
    \toks@{Journal #1 -> #2 / #3^^J = #4}\typeout{\the\toks@}%
  \fi
}
\newcommand{\SerialName}[3]{#2}
\let\SerialISSNName\SerialName
\newcommand{\DefinePublisher}[1]{%
  \lowercase{\uppercase{\expandafter\DefPub@a\csname PN}.#1}\endcsname{#1}%
}
\def\DefPub@a#1#2#3#4#5{%
  \ifx\relax#1\else
    \PackageWarning{amsrefs}{Redefining publisher '#2'}%
  \fi
  \gdef#1{\PublisherNameAddress{#3}{#4}{#5}}%
  \if\amsrefs@verbose
    \toks@{Publisher #2 -> #3 / #5^^J = #4}\typeout{\the\toks@}%
  \fi
}
\newcommand{\PublisherNameAddress}[3]{%
  #1\IfEmptyBibField{place}{\SwapBreak{,} #3}{}%
}
\IfOption{jpa}{\RequirePackage{amsjpa}[2000/02/02]}{}
\endinput
%%
%% End of file `amsrefs.sty'.
\endinput}%
}{%
    \let\@tempa\@empty
}
\@tempa
%    \end{macrocode}
%
%    \begin{macrocode}
\IfFileExists{url.sty}{%
    \RequirePackage{url}\relax
    \@gobble
}{%
    \@firstofone
}
{
    \DeclareRobustCommand{\url}[1]{%
        \def\@tempa{#1}%
        \texttt{\@urlsetup $\expandafter\strip@prefix\meaning\@tempa$}%
    }%
    \def\@urlsetup{%
        \check@mathfonts \textfont\@ne\the\font \textfont\z@\the\font
        \@apply\@urlfix{\do\+\do\=\do\:\do\-\do\.\do\,\do\;}%
        \@apply\@urlbreak{\do\&\do\/\do\?}%
    }%
    \def\@urlbreak#1{%
        \mathcode`#1="8000
        \begingroup \lccode`\~=`#1 \lowercase{\endgroup \edef~}%
        {\mathchar\number`#1\penalty\hyphenpenalty}%
    }%
    \def\@urlfix#1{%
        \mathcode`#1=`#1\relax
    }%
}
%    \end{macrocode}
%
%    \begin{macrocode}
\@ifundefined{NormalCatcodes}{\RequirePackage{pcatcode}\relax}{}
\PushCatcodes\NormalCatcodes
\ProvidesPackage{amsrefs}[2004/06/07 v1.71]
%    \end{macrocode}
%
%    \begin{macrocode}
%% WARNING WARNING WARNING: Catcode of apostrophe ' is letter
%% throughout this file.
\catcode`\'=11 % letter
%    \end{macrocode}
%
% \subsection{Utilities}
%
%    Some of these useful functions are also found in AMS
%    document classes.
%
%    \begin{macro}{\after@deleting@token}
%    Similar in concept to \cs{afterassignment}, except it deletes the
%    next token in the stream before putting its argument
%    back into the input.  Useful for skipping past tokens during
%    parsing.
%    \begin{macrocode}
\def\after@deleting@token#1{%
    \afterassignment#1%
    \let\@let@token= % Don't delete this space!
}
%    \end{macrocode}
%    \end{macro}
%
%    \begin{macro}{\@ifempty}
%    \begin{macro}{\@ifnotempty}
%    Some frequently used tests for empty arguments.  Note that an
%    argument consisting entirely of spaces (e.g.,
%    \verb*|\@ifempty{ }|) counts as empty.
%    \begin{macrocode}
\long\def\@ifempty#1{\@xifempty#1@@..\@nil}

\long\def\@xifempty#1#2@#3#4#5\@nil{%
    \ifx#3#4\@xp\@firstoftwo\else\@xp\@secondoftwo\fi
}

\long\def\@ifnotempty#1{\@ifempty{#1}{}}
%    \end{macrocode}
%    \end{macro}
%    \end{macro}
%
%    \begin{macro}{\macrotext}
%    \begin{macrocode}
\def\macrotext{\expandafter\strip@prefix\meaning}
%    \end{macrocode}
%    \end{macro}
%
%    \begin{macro}{\vdef}
%    ``Verbatim'' def.
%    \begin{macrocode}
\def\vdef#1#2{%
    \def#1{#2}%
    \edef#1{\macrotext#1}%
}
%    \end{macrocode}
%    \end{macro}
%
%    \begin{macro}{\auto@protect}
%    Sometimes it's convenient to render a given control sequence
%    unexpandable for a time.  \cs{auto@protect} provides a way to do
%    that.\footnote{There really should be a special name for macros
%    that, like \cs{auto@protect}, take a control sequence as an
%    argument and redefine that control sequence in order to achieve
%    some special effect.  Pending happier inspiration, I'm
%    going to call them ``wrapper'' macros.\index{wrapper macros,
%    definition}}
%
%    An earlier version of this code read simply |\let#1\relax| but
%    that had the disadvantage of making all \cs{auto@protect}ed
%    macros compare equal via \cs{ifx}.  This version allows macros to
%    keep their identities under comparisons.
%    \begin{macrocode}
\def\auto@protect#1{\def#1{\@nx#1}}
%    \end{macrocode}
%    \end{macro}
%
%    \begin{macro}{\g@undef}
%    Globally undefine a control sequence.
%    \begin{macrocode}
\def\g@undef#1{\global\let#1\relax}
%    \end{macrocode}
%    \end{macro}
%
%    \begin{macro}{\@concat}
%    Concatenate onto the end of a token list.  Expands everything.
%    \begin{macrocode}
\def\@concat#1#2{\edef#1{#1#2}}
%    \end{macrocode}
%    \end{macro}
%
%    \begin{macro}{\add@toks@}
%    This saves a few tokens of main memory and a lot of typing.
%    \begin{macrocode}
\def\add@toks@{\addto@hook\toks@}
%    \end{macrocode}
%    \end{macro}
%
%    \begin{macro}{\@lappend}
%    Append an element to a \cs{do}-delimited list.  As long as the
%    element to be appended (|#2|) is a single token, nothing is
%    expanded.  If it contains multiple tokens, all tokens after the
%    first will be expanded.
%    \begin{macrocode}
\def\@lappend#1#2{%
    \begingroup
        \def\do{\@nx\do\@nx}%
        \edef\@tempa{\def\@nx#1{#1\do#2}}%
    \@xp\endgroup
    \@tempa
}
%    \end{macrocode}
%    \end{macro}
%
%    \begin{macro}{\@apply}
%    Apply a macro to each element of a \cs{do}-delimited list.
%    \begin{macrocode}
\def\@apply#1#2{%
    \let\do#1%
    #2%
}
%    \end{macrocode}
%    \end{macro}
%
%    \begin{macro}{\get@numberof}
%    This is a generic macro for counting the number of elements in a
%    \lat/-style list.
%    The first argument is a \cs{count} register that will receive the
%    final count; the second argument is the control sequence that
%    separates elements of the list, and the third argument is the
%    list itself.  So, for example,
%    \begin{verbatim}
%    \get@numberof\@tempcnta\do\dospecials
%\end{verbatim}
%    would count the number of special characters in \cs{dospecials}
%    and store the number in \cs{@tempcnta}.
%    \begin{macrocode}
\def\get@numberof#1#2#3{%
    \begingroup
        \def#2{\advance\@tempcnta\@ne \@gobble}%
        \@tempcnta\z@
        #3\relax
        \edef\@tempb{#1=\the\@tempcnta\relax}%
    \@xp\endgroup
    \@tempb
}
%    \end{macrocode}
%    \end{macro}
%
%    \begin{macro}{\safe@set}
%    This is a quick and dirty way of extracting an integer prefix
%    from a string and assigning it to a counter.  If the string does
%    not begin with an integer, the counter receives the value~0.  The
%    suffix after the integer prefix is discarded.  (But bad things
%    will happen if the string contains the token \cs{@nil}.)
%    \begin{macrocode}
\def\safe@set#1#2{%
    \afterassignment\@nilgobble
    #1=0#2\relax\@nil
}
%    \end{macrocode}
%    \end{macro}
%
%    \begin{macro}{\@chomp}
%    Vaguely reminiscent of Perl's \texttt{chomp} function, which removes
%    a substring from the end of a variable, but ours works with
%    tokens (more-or-less) and takes the substring to be removed as
%    its second argument.  Note the use of \cs{@empty} to anchor the
%    chomped substring to the end of the string.  Note also that the
%    second argument will be fully expanded during the chomping.
%    \begin{macrocode}
\def\@chomp#1#2{%
    \begingroup
        \toks@\@emptytoks
        \def\@chomper##1##2#2\@empty##3\@nil{%
            \ifx\@let@token\bgroup
                \toks@{{##1}##2}%
            \else
                \toks@{##1##2}%
            \fi
        }%
        \@xp\chomp@ #1\@empty#2\@empty\@nil
        \edef\@tempa{\def\@nx#1\@xp{\the\toks@}}%
    \@xp\endgroup
    \@tempa
}
%    \end{macrocode}
%    \end{macro}
%
%    \begin{macro}{\chomp@}
%    Before passing control to \cs{@chomper}, we peek ahead at the
%    next token in the stream.  That way, if the next token is an open
%    brace, we know we need to surround \cs{@chomper}'s first argument
%    with braces.  Unfortunately, this might still remove braces from
%    the second argument, but I think that's ok for our purposes.
%    \begin{macrocode}
\def\chomp@{%
    \futurelet\@let@token
    \@chomper
}
%    \end{macrocode}
%    \end{macro}
%
%    \begin{macro}{\amsrefs@warning}
%    \begin{macrocode}
\def\amsrefs@warning{\PackageWarning{amsrefs}}
%    \end{macrocode}
%    \end{macro}
%
%    \begin{macro}{\amsrefs@error}
%    \begin{macrocode}
\def\amsrefs@error{\PackageError{amsrefs}}
%    \end{macrocode}
%    \end{macro}
%
%    \begin{macro}{\MessageBreakNS}
%    This suppresses the leading space in \cs{on@line} in error and
%    warning messages.
%    \begin{macrocode}
\def\MessageBreakNS{\MessageBreak\romannumeral`\^^@}
%    \end{macrocode}
%    \end{macro}
%
%    \begin{macro}{\@addpunct}
%    The \cs{@addpunct} function is defined by AMS document classes and
%    the \pkg{amsgen} package. But if we find it undefined we had better
%    define it.
%    \begin{macrocode}
\@ifundefined{@addpunct}{%
    \def\@addpunct#1{%
        \relax\ifhmode
            \ifnum\spacefactor>\@m \else#1\fi
        \fi
    }
    \def\frenchspacing{%
        \sfcode`\.1006
        \sfcode`\?1005
        \sfcode`\!1004
        \sfcode`\:1003
        \sfcode`\;1002
        \sfcode`\,1001\relax
    }
}{}
%    \end{macrocode}
%    \end{macro}
%
%    \begin{macro}{\nopunct}
%    Omit any following punctuation that would normally be inserted by
%    \cs{@addpunct}.
%    \begin{macrocode}
\providecommand{\nopunct}{\spacefactor \@nopunctsfcode}
%    \end{macrocode}
%    \end{macro}
%
%    \begin{macro}{\@nopunctsfcode}
%    \begin{macrocode}
\def\@nopunctsfcode{1007 }
%    \end{macrocode}
%    \end{macro}
%
% \subsection{Declaring package options}
%
%    We call the \pkg{ifoption} package to facilitate some option tests.
%    \begin{macrocode}
\RequirePackage{ifoption}[2000/02/15]
%    \end{macrocode}
%
%    The \opt{sorted} option is a no-op and is no longer documented.
%    I'm only leaving it here for backwards compatibility.
%    \begin{macrocode}
\DeclareExclusiveOptions{sorted,citation-order}
%    \end{macrocode}
%    The \opt{alphabetic} option corresponds to the standard \bst{alpha}
%    biblio style with labels like Knu66 (three letters from name plus
%    two digits of year). Maybe should provide an alias LllYY for this
%    option. Numeric is the default since it is commoner in AMS
%    publications.
%    \begin{macrocode}
\DeclareExclusiveOptions{alphabetic,shortalphabetic,author-year,numeric}
%    \end{macrocode}
%
%    \begin{option}{y2k}
%    \begin{macrocode}
\DeclareBooleanOption{y2k}
%    \end{macrocode}
%    \end{option}
%
%    \begin{option}{nobysame}
%    \begin{macrocode}
\DeclareBooleanOption{nobysame}
%    \end{macrocode}
%    \end{option}
%
%    The standard \bst{abbrv} bibliography style uses abbreviations for
%    month names and journal names, and first names of people are
%    abbreviated to their initials. Since the second test bibliography
%    that I tested with had unabbreviated month names but abbreviated
%    journal names, perhaps it is a good idea to let these choices be
%    specified separately.
%    \begin{macrocode}
\DeclareBooleanOption{short-journals}
%    \end{macrocode}
%    \begin{macrocode}
\DeclareBooleanOption{short-publishers}
%    \end{macrocode}
%    The \opt{short-journals} and \opt{short-publishers} options only
%    affect journal and publisher names that are defined with
%    \cn{DefineJournal} and \cn{DefinePublisher} commands.
%    \begin{macrocode}
\DeclareBooleanOption{short-months}
%    \end{macrocode}
%
%    \begin{macrocode}
\DeclareBooleanOption{initials}
%    \end{macrocode}
%    Nevertheless, it's to be expected that the preceding four options
%    would typically be used together, so we provide a short-hand for
%    requesting them all.
%    \begin{macrocode}
\DeclareOption{abbrev}{%
    \@pass@ptions
        \@currext
        {initials,short-months,short-journals,short-publishers}%
        \@currname
}
%    \end{macrocode}
%
%    In the bibliography, if a title or something is enclosed in quotes,
%    should the closing quotes go inside the punctuation (logical
%    position) rather than outside (traditional)? These options give you
%    a choice.
%    \begin{macrocode}
\DeclareExclusiveOptions{traditional-quotes,logical-quotes}
%    \end{macrocode}
%
%    A sequence of cites will be sorted and ranges of length three or
%    greater will be compressed if these options so indicate.  Note
%    that the \opt{non-sorted-cites} option automatically disables
%    compression.  This is probably a feature.
%    \begin{macrocode}
\DeclareExclusiveOptions{sorted-cites,non-sorted-cites}
\DeclareExclusiveOptions{non-compressed-cites,compressed-cites}
%    \end{macrocode}
%
%    In the bibliography, print page numbers showing where each
%    entry was cited.
%    \begin{macrocode}
\DeclareBooleanOption{backrefs}
%    \end{macrocode}
%
%    Option for giving information about the available options:
%    \begin{macrocode}
\DeclareBooleanOption{?}
%    \end{macrocode}
%
%    This option means to forgo loading of the \pkg{textcmds} and
%    \pkg{mathscinet} packages.
%    \begin{macrocode}
\DeclareBooleanOption{lite}
%    \end{macrocode}
%
%    This option can be used by later releases as a sign that fall-back
%    adaptations need to be done.
%    \begin{macrocode}
\DeclareBooleanOption{beta}
%    \end{macrocode}
%
%    This one is obsolete now.
%    \begin{macrocode}
\DeclareBooleanOption{jpa}
%    \end{macrocode}
%
%    \begin{macrocode}
\DeclareBooleanOption{bibtex-style}
%    \end{macrocode}
%
%    \begin{macrocode}
\ExecuteOptions{numeric,traditional-quotes,sorted-cites,compressed-cites}

\ProcessOptions\relax

\ProcessExclusiveOptions
%    \end{macrocode}
%
%    \begin{macrocode}
\IfOption{backrefs}{%
    \IfFileExists{hyperref.sty}{%
        \RequirePackage{hyperref}[1999/07/08]
    }{}%
    \IfFileExists{backref.sty}{%
        \RequirePackage{backref}[1999/05/30]
    }{}%
}{}
%    \end{macrocode}
%
% \subsubsection{The \opt{?} option}
%
%    Note that in the following auxiliary package list, \pkg{getwidth}
%    is not (yet) included.
%    \begin{macrocode}
\IfOption{?}{%
    \typeout{^^J%
        Documentation for the amsrefs package is found in amsrdoc.dvi^^J%
        (or .pdf or .tex).
        ^^J%
    }%
}{}%
%    \end{macrocode}
%
% \subsection{Loading auxiliary packages}
%    Now, if these other packages make use of the \pkg{pcatcode} package
%    like they should, then we don't need to make any fuss here about
%    the special catcode of \texttt{'}. Just load the packages.
%    \begin{macrocode}
\RequirePackage{rkeyval}[2001/12/22]
%    \end{macrocode}
%
% \subsubsection{The \opt{lite} option}
%
%    In my opinion, this is misguided, since \pkg{amsrefs} shouldn't
%    be loading these packages to begin with.  But it's too late to
%    change it now.
%
%    \begin{macrocode}
\IfOption{lite}{% True? Then don't load the next two packages.
}{%              False? OK, let's load them:
    \RequirePackage{textcmds}[2001/12/14]
    \RequirePackage{mathscinet}[2002/01/01]
}
%    \end{macrocode}
%
% \subsection{Key-value setup}
%
%    \begin{macro}{\BibField}
%    This provides easy access to individual fields for user-defined
%    formatting functions.
%    \begin{macrocode}
\newcommand{\BibField}[1]{\csname bib'#1\endcsname}
%    \end{macrocode}
%    \end{macro}
%
%    \begin{macro}{\IfEmptyBibField}
%    A convenient partial application of \cn{rkvIfEmpty}.
%    \begin{macrocode}
\newcommand{\IfEmptyBibField}{\rkvIfEmpty{bib}}
%    \end{macrocode}
%    \end{macro}
%
% \subsubsection{Standard field names (the \texttt{bib} group)}
%
%    And here are the predefined key names. You could always add some
%    more if you needed them. Only worry is about compatibility if you
%    want to share your data with other people.
%
%    \begin{macro}{\fld@elt}
%    \begin{macro}{\name}
%    We want the list macros used above to be unexpandable except when
%    special processing is done.  (It's not clear to me there's any
%    real benefit to using these instead of just using \cn{do}.---dmj)
%    \begin{macrocode}
\let\fld@elt=?
\let\name=?
%    \end{macrocode}
%    \end{macro}
%    \end{macro}
%
%    First the fields that could be repeated more than once in a single
%    entry. Maybe publisher should be allowed to repeat also, for
%    co-published works. But then need to worry about the address
%    handling.
%    \begin{macrocode}
\DefineAdditiveKey{bib}{author}{\name}
\DefineAdditiveKey{bib}{editor}{\name}
\DefineAdditiveKey{bib}{translator}{\name}
\DefineAdditiveKey{bib}{contribution}{\fld@elt}
\DefineAdditiveKey{bib}{isbn}{\fld@elt}
\DefineAdditiveKey{bib}{issn}{\fld@elt}
\DefineAdditiveKey{bib}{review}{\fld@elt}
\DefineAdditiveKey{bib}{partial}{\fld@elt}
%    \end{macrocode}
%
%    \begin{macrocode}
\DefineSimpleKey{bib}{address}
\DefineSimpleKey{bib}{book}
\DefineSimpleKey{bib}{booktitle}
\DefineSimpleKey{bib}{conference}
%\DefineSimpleKey{bib}{contributor}
\DefineSimpleKey{bib}{copula}
\DefineSimpleKey{bib}{date}
\DefineSimpleKey{bib}{doi}
\DefineSimpleKey{bib}{edition}
\DefineSimpleKey{bib}{eprint}
\DefineSimpleKey{bib}{fulljournal}
\DefineSimpleKey{bib}{hyphenation}
\DefineSimpleKey{bib}{institution}
\DefineSimpleKey{bib}{journal}
\DefineSimpleKey{bib}{label}
\DefineSimpleKey{bib}{language}
\DefineSimpleKey{bib}{name}
\DefineSimpleKey{bib}{note}
\DefineSimpleKey{bib}{number}
\DefineSimpleKey{bib}{organization}
\DefineSimpleKey{bib}{pages}
\DefineSimpleKey{bib}{part}
\DefineSimpleKey{bib}{place}
\DefineSimpleKey{bib}{publisher}
\DefineSimpleKey{bib}{reprint}
\DefineSimpleKey{bib}{school}
\DefineSimpleKey{bib}{series}
\DefineSimpleKey{bib}{setup}
\DefineSimpleKey{bib}{status}
\DefineSimpleKey{bib}{subtitle}
\DefineSimpleKey{bib}{title}
\DefineSimpleKey{bib}{translation}
\DefineSimpleKey{bib}{type}
\DefineSimpleKey{bib}{url}
\DefineSimpleKey{bib}{volume}
\DefineSimpleKey{bib}{xref}
\DefineSimpleKey{bib}{year}
%    \end{macrocode}
%    The \fld{transition} key is used when we want to insert punctuation
%    or other material at a given point in the sequence
%    unconditionally. The key appears to have a non-empty value to
%    \cn{IfEmptyBibField}, but its value (expansion) is empty.
%    \begin{macrocode}
\DefineDummyKey{bib}{transition}
%    \end{macrocode}
%
% \subsubsection{Auxiliary properties (the \texttt{prop} group)}
%
%    \begin{macrocode}
\DefineSimpleKey{prop}{inverted}
\DefineSimpleKey{prop}{language}
%    \end{macrocode}
%
% \subsection{Bibliography type specifications}
%
%    \begin{macro}{\BibSpec}
%    Accumulate specification material in \cs{toks@}, then define
%    \cs{setbib@TYPE} from it.
%    \begin{macrocode}
\newcommand{\BibSpec}[2]{%
    \toks@\@emptytoks
    \@ifnotempty{#2}{%
%    \end{macrocode}
%    The \cs{@ifnextchar} removes an optional |+| at the beginning of
%    a specification.  From then on, each time \cs{bibspec@scan} is
%    invoked, it expects to find four arguments.  The four
%    \cs{@empty}s appended to the specification (|#2|) below ensure
%    that this is so.
%    \begin{macrocode}
        \@ifnextchar{+}{\@xp\bibspec@scan\@gobble}{\bibspec@scan}%
        #2\@empty\@empty\@empty\@empty
    }%
    \@xp\edef\csname setbib@#1\endcsname{\the\toks@}%
}
%    \end{macrocode}
%    \end{macro}
%
%    \begin{macro}{\bibspec@scan}
%    The \cs{bibspec@scan} function scans one field specification
%    from the second arg of \cn{BibSpec}. Each field specification has
%    the form
%    \begin{verbatim}
%    +{punctuation}{prelim material}{field name}
%\end{verbatim}
%    Note however that because the initial |+| is stripped off by
%    \cs{BibSpec} (see above), the actual order that \cs{bibspec@scan}
%    reads the field specification is
%    \begin{verbatim}
%    #1={punctuation} #2={prelim material} #3={field name} #4=+
%\end{verbatim}
%    where the fourth argument is actually expected to be either the
%    |+| from the following specification, or one of the special
%    \cs{@empty} tokens inserted by \cs{BibSpec}.  If it is neither of
%    these special values, it means we have a malformed specification;
%    so, we issue an error and then try to pick up where we left off.
%    \begin{macrocode}
\def\bibspec@scan#1#2#3#4{%
    \add@toks@{\bib@append{#1}{#2}}%
    \edef\@tempa{%
        \toks@{\the\toks@ \@xp\@nx\csname bib'#3\endcsname}%
    }%
    \@tempa
    \ifx\@empty#4%
        \@xp\@gobble % end the recursion
    \else
        \ifx +#4\else\bibspec@scan@error\fi
    \fi
    \bibspec@scan
}
%    \end{macrocode}
%    \end{macro}
%
%    \begin{macro}{\bibspec@scan@error}
%    \begin{macrocode}
\def\bibspec@scan@error{\amsrefs@error{Bad BibSpec: Expected '+'}}
%    \end{macrocode}
%    \end{macro}
%
%    \begin{macro}{\bib@append}
%    The function \cs{bib@append} prints the value of a field, together
%    with associated punctuation and font changes, unless the value is
%    empty. Arg 1 is punctuation (that may need to be swapped with a
%    preceding line break), arg 2 gives the space to be added after the
%    punctuation, and possibly a function to be applied to the contents
%    of arg 3, which is
%    a macro containing the field
%    value. So if we have \cn{moo} and \cn{bib'pages}, from
%    \verb'pages={21\ndash 44}', then we want to arrange to call
%    \begin{verbatim}
%    \moo{21\ndash 44}
%\end{verbatim}
%    We don't want to simply call \cs{moo}\cs{bib'bar} because that makes
%    it rather difficult for \cs{moo} to look at the contents of
%    \cs{bib@bar}.
%    \begin{macrocode}
\def\bib@append#1#2#3{%
    \ifx\@empty#3%
    \else
%    \end{macrocode}
%    \begin{bug}
%    Need better error message here.
%    \end{bug}
%    \begin{macrocode}
        \ifx\relax#3%
            \errmessage{#3=\relax}%
        \else
            \begingroup
                \series@index\m@ne
                \def\current@bibfield{#3}%
                \@ifempty{#1}{%
                    \@temptokena{\ifnum\lastkern=\@ne\ignorespaces\fi #2}%
                }{%
                    \@temptokena{\SwapBreak{#1}#2}%
                }%
                \toks@\@xp{#3}%
                \edef\@tempa{\the\@temptokena{\the\toks@}}%
                \rkvIfAdditive#3{}{%
                    \get@current@properties
                    \select@auxlanguage
                }%
                \@tempa
            \endgroup
        \fi
    \fi
}
%    \end{macrocode}
%    \end{macro}
%
%    \begin{macro}{\select@auxlanguage}
%    \begin{macrocode}
\def\select@auxlanguage{%
    \ifx\prop'language\@empty
    \else
        \@xp\selectlanguage\@xp{\prop'language}%
    \fi
}
%    \end{macrocode}
%    \end{macro}
%
%    \begin{macro}{\erase@field}
%    There are some fields that can appear in more than one place in a
%    reference, depending on context.  For example, if a book has an
%    editor but no author, the editor appears at the beginning of the
%    entry, but if the book has both an editor and an author, the
%    editor appears at the end of the entry. A simple way to handle
%    this is to ``erase'' the \fld{editor} field after printing it,
%    which is what \cs{erase@field} is for.
%
%    The obvious definition of \cs{erase@field} is
%    \begin{verbatim}
%    \def\erase@field#1{\global\let#1\@empty}
%\end{verbatim}
%    but that doesn't work because the top-level value of
%    \pkg{rkeyval} fields isn't \cs{@empty}; instead, it contains a
%    setter function used by \cs{RestrictedSetKeys} when processing a
%    key-value list (see \cs{rkv@DSAK}, \cs{rsk@set@a} and
%    \cs{rsk@set@b}).
%
%    On the other hand, rewriting the field locally won't work
%    either, since \cs{erase@field} will typically be executed inside
%    the group established by \cs{bib@append}.  Instead, we want to
%    rewrite the value right after \cs{bib@append}'s group ends.  One
%    way to do this would be to keep a list of fields to be erased and
%    have \cs{bib@append} iterate over the list after its
%    \ncn{endgroup}.
%
%    However, as long as the call to \cs{erase@field}
%    is never nested within any deeper groups, it's simpler just to
%    use \cs{aftergroup}, which is what we'll do (``Sufficient unto
%    the day is the evil thereof'' and all that).
%    \begin{macrocode}
\def\erase@field#1{%
    \aftergroup\let\aftergroup#1\aftergroup\@empty
}
%    \end{macrocode}
%    \end{macro}
%
%    \begin{macro}{\get@current@properties}
%    This retrieves the auxiliary properties for the current field
%    value, as defined by \cs{current@bibfield} and
%    \cs{series@index}.
%    \begin{macrocode}
\def\get@current@properties{%
    \begingroup
        \@xp\get@nth@property\@xp\@tempa\current@bibfield\series@index
        \edef\@tempa{%
            \@nx\RestrictedSetKeys{}{prop}{%
                \def\@nx\@tempa{\@nx\prop@reset \@nx\the\@nx\rsk@toks}%
            }{\@tempa}%
        }%
        \@tempa
    \@xp\endgroup
    \@tempa
}
%    \end{macrocode}
%    \end{macro}
%
%    \begin{macro}{\BibSpecAlias}
%    This is a \cs{def} rather than a \cs{let} because using \cs{let}
%    would make \cn{BibSpecAlias} statements order-sensitive in a way
%    that seems frequently to be a stumbling block to unwary package
%    writers.  But then we should probably do at least the simplest
%    kind of infinite loop check.
%    \begin{macrocode}
\newcommand{\BibSpecAlias}[2]{%
    \@xp\def\@xp\@tempa\@xp{\csname setbib@#1\@xp\endcsname}%
    \@xp\ifx\csname setbib@#2\endcsname\@tempa
        \amsrefs@error{%
            Mirror alias #1->#2 not allowed (infinite loop)}\@ehc
    \else
        \@xp\def\csname setbib@#1\@xp\endcsname
            \@xp{\csname setbib@#2\endcsname}%
    \fi
}
%    \end{macrocode}
%    \end{macro}
%
% \subsection{The standard bibliography types}
%
%    \begin{macrocode}
\BibSpec{article}{%
    +{}  {\PrintAuthors}                {author}
    +{,} { \textit}                     {title}
    +{.} { }                            {part}
    +{:} { \textit}                     {subtitle}
    +{,} { \PrintContributions}         {contribution}
    +{.} { \PrintPartials}              {partial}
    +{,} { }                            {journal}
    +{}  { \textbf}                     {volume}
%    \end{macrocode}
%    The date form is tricky depending on presence or absence of DOI.
%    \begin{macrocode}
    +{}  { \PrintDatePV}                {date}
    +{,} { \issuetext}                  {number}
    +{,} { \eprintpages}                {pages}
    +{,} { }                            {status}
    +{,} { \PrintDOI}                   {doi}
    +{,} { available at \eprint}        {eprint}
    +{}  { \parenthesize}               {language}
    +{}  { \PrintTranslation}           {translation}
    +{;} { \PrintReprint}               {reprint}
    +{.} { }                            {note}
    +{.} {}                             {transition}
    +{}  {\SentenceSpace \PrintReviews} {review}
}

\BibSpec{partial}{%
    +{}  {}                             {part}
    +{:} { \textit}                     {subtitle}
    +{,} { \PrintContributions}         {contribution}
    +{,} { }                            {journal}
    +{}  { \textbf}                     {volume}
    +{}  { \PrintDatePV}                {date}
    +{,} { \issuetext}                  {number}
    +{,} { \eprintpages}                {pages}
}

\BibSpec{contribution}{%
    +{} {}                             {type}
    +{} { by \PrintNameList}           {author}
}

\BibSpec{book}{%
    +{}  {\PrintPrimary}                {transition}
    +{,} { \textit}                     {title}
    +{.} { }                            {part}
    +{:} { \textit}                     {subtitle}
    +{,} { \PrintEdition}               {edition}
    +{}  { \PrintEditorsB}              {editor}
    +{,} { \PrintTranslatorsC}          {translator}
    +{,} { \PrintContributions}         {contribution}
    +{,} { }                            {series}
    +{,} { \voltext}                    {volume}
    +{,} { }                            {publisher}
    +{,} { }                            {organization}
    +{,} { }                            {address}
    +{,} { \PrintDateB}                 {date}
    +{,} { }                            {status}
    +{}  { \parenthesize}               {language}
    +{}  { \PrintTranslation}           {translation}
    +{;} { \PrintReprint}               {reprint}
    +{.} { }                            {note}
    +{.} {}                             {transition}
    +{}  {\SentenceSpace \PrintReviews} {review}
}

\BibSpec{collection.article}{%
    +{}  {\PrintAuthors}                {author}
    +{,} { \textit}                     {title}
    +{.} { }                            {part}
    +{:} { \textit}                     {subtitle}
    +{,} { \PrintContributions}         {contribution}
    +{,} { \PrintConference}            {conference}
    +{}  {\PrintBook}                   {book}
    +{,} { }                            {booktitle}
    +{,} { \PrintDateB}                 {date}
    +{,} { pp.~}                        {pages}
    +{,} { }                            {status}
    +{,} { \PrintDOI}                   {doi}
    +{,} { available at \eprint}        {eprint}
    +{}  { \parenthesize}               {language}
    +{}  { \PrintTranslation}           {translation}
    +{;} { \PrintReprint}               {reprint}
    +{.} { }                            {note}
    +{.} {}                             {transition}
    +{}  {\SentenceSpace \PrintReviews} {review}
}

\BibSpec{conference}{%
    +{}  {}                        {title}
    +{}  {\PrintConferenceDetails} {transition}
}

\BibSpec{innerbook}{%
    +{,} { }                            {title}
    +{.} { }                            {part}
    +{:} { }                            {subtitle}
    +{,} { \PrintEdition}               {edition}
    +{}  { \PrintEditorsB}              {editor}
    +{,} { \PrintTranslatorsC}          {translator}
    +{,} { \PrintContributions}         {contribution}
    +{,} { }                            {series}
    +{,} { \voltext}                    {volume}
    +{,} { }                            {publisher}
    +{,} { }                            {organization}
    +{,} { }                            {address}
    +{,} { \PrintDateB}                 {date}
    +{.} { }                            {note}
}

\BibSpec{report}{%
    +{}  {\PrintPrimary}                {transition}
    +{,} { \textit}                     {title}
    +{.} { }                            {part}
    +{:} { \textit}                     {subtitle}
    +{,} { \PrintEdition}               {edition}
    +{,} { \PrintContributions}         {contribution}
    +{,} { Technical Report }           {number}
    +{,} { }                            {series}
    +{,} { }                            {organization}
    +{,} { }                            {address}
    +{,} { \PrintDateB}                 {date}
    +{,} { \eprint}                     {eprint}
    +{,} { }                            {status}
    +{}  { \parenthesize}               {language}
    +{}  { \PrintTranslation}           {translation}
    +{;} { \PrintReprint}               {reprint}
    +{.} { }                            {note}
    +{.} {}                             {transition}
    +{}  {\SentenceSpace \PrintReviews} {review}
}

\BibSpec{thesis}{%
    +{}  {\PrintAuthors}                {author}
    +{,} { \textit}                     {title}
    +{:} { \textit}                     {subtitle}
    +{,} { \PrintThesisType}            {type}
    +{,} { }                            {organization}
    +{,} { }                            {address}
    +{,} { \PrintDateB}                 {date}
    +{,} { \eprint}                     {eprint}
    +{,} { }                            {status}
    +{}  { \parenthesize}               {language}
    +{}  { \PrintTranslation}           {translation}
    +{;} { \PrintReprint}               {reprint}
    +{.} { }                            {note}
    +{.} {}                             {transition}
    +{}  {\SentenceSpace \PrintReviews} {review}
}
%    \end{macrocode}
%
%    \begin{macrocode}
\BibSpecAlias{periodical}{book}
\BibSpecAlias{collection}{book}
\BibSpecAlias{proceedings}{book}
\BibSpecAlias{manual}{book}
\BibSpecAlias{miscellaneous}{book}
\BibSpecAlias{misc}{miscellaneous}
\BibSpecAlias{unpublished}{book}
\BibSpecAlias{proceedings.article}{collection.article}
\BibSpecAlias{techreport}{report}
%    \end{macrocode}
%
%    \begin{macro}{\setbib@incollection}
%    \begin{macrocode}
\edef\setbib@incollection{%
    \@xp\@nx\csname setbib@collection.article\endcsname
}
%    \end{macrocode}
%    \end{macro}
%
%    \begin{macro}{\setbib@inproceedings}
%    \begin{macrocode}
\edef\setbib@inproceedings{%
    \@xp\@nx\csname setbib@collection.article\endcsname
}
%    \end{macrocode}
%    \end{macro}
%
%    Some more entry types for implementing abbreviations.
%    \begin{macrocode}
\BibSpec{name}{%
    +{}  {\PrintAuthors}    {name}
}

\BibSpec{publisher}{%
    +{,} { } {publisher}
    +{,} { } {address}
}
%    \end{macrocode}
%
% \subsection{The \env{biblist} environment}
%
%    The \env{biblist} environment can be used with a section or chapter
%    heading.
%
%    Use a standard \lat/ counter for numbering bibliography items.
%    \begin{macrocode}
\newcounter{bib}
%    \end{macrocode}
%
%    \begin{environment}{biblist}
%    \begin{macrocode}
\newenvironment{biblist}{%
    \setcounter{bib}\z@
    \@biblist
}{%
    \@endbiblist
}
%    \end{macrocode}
%    \end{environment}
%
%    \begin{environment}{biblist*}
%    \begin{macrocode}
\newenvironment{biblist*}{%
    \@biblist
}{%
    \@endbiblist
}
%    \end{macrocode}
%    \end{environment}
%
%    \begin{macro}{\@biblist}
%    \begin{macrocode}
\newcommand\@biblist[1][]{%
    \stepcounter{bib@env}
    \normalfont
    \footnotesize
    \labelsep .5em\relax
    \list{\BibLabel}{%
        \restore@labelwidth
        \@maxlabelwidth\z@
        \@nmbrlisttrue
        \def\@listctr{bib}%
        \let\makelabel\bib@mklab
        #1\relax
    }%
    \sloppy
%    \end{macrocode}
%    Discourage page breaks within bibliography entries and disable them
%    completely for entries that are less than four lines long.
%    \begin{macrocode}
    \interlinepenalty\@m
    \clubpenalty\@M
    \widowpenalty\clubpenalty
    \frenchspacing
    \ResetCapSFCodes
}
%    \end{macrocode}
%    \end{macro}
%
%    \begin{macro}{\@endbiblist}
%    Change error for empty list (no items) to warning, to allow authors
%    to leave their bibliography temporarily empty during writing:
%    \begin{macrocode}
\def\@endbiblist{%
    \save@labelwidth
    \def\@noitemerr{\@latex@warning{Empty bibliography list}}%
    \endlist
}
%    \end{macrocode}
%    \end{macro}
%
%    \begin{macro}{\@maxlabelwidth}
%    \begin{macrocode}
\newdimen\@maxlabelwidth
%    \end{macrocode}
%    \end{macro}
%
%    \begin{macro}{\bib@mklab}
%    \begin{macrocode}
\def\bib@mklab#1{%
    \settowidth\@tempdima{#1}%
    \ifdim \@tempdima > \@maxlabelwidth
        \global\@maxlabelwidth\@tempdima
    \fi
    #1\hfil
}
%    \end{macrocode}
%    \end{macro}
%
%    \begin{macrocode}
\newcounter{bib@env}
%    \end{macrocode}
%
%    \begin{macro}{\save@labelwidth}
%    \begin{macrocode}
\def\save@labelwidth{%
    \if@filesw
        \immediate\write\@auxout{%
            \string\newlabel{[bibenv:\the\c@bib@env]}{\the\@maxlabelwidth}%
        }%
    \fi
}
%    \end{macrocode}
%    \end{macro}
%
%    \begin{macro}{\restore@labelwidth}
%    \begin{macrocode}
\def\restore@labelwidth{%
    \@xp\ifx \csname r@[bibenv:\the\c@bib@env]\endcsname \relax
        \resetbiblist{00}%
    \else
        \@xp\labelwidth\csname r@[bibenv:\the\c@bib@env]\endcsname
        \leftmargin\labelwidth
        \advance\leftmargin\labelsep
    \fi
}
%    \end{macrocode}
%    \end{macro}
%
%    \begin{macro}{\ResetCapSFCodes}
%    Presumably this is here because there has been a problem in the
%    past with packages that change the \cs{catcode}s of capital
%    letters.
%    \begin{macrocode}
\providecommand{\ResetCapSFCodes}{%
    \count@=`\A
    \def\@tempa{%
        \sfcode\count@=\@m
        \advance\count@\@ne
            \ifnum\count@>`\Z\relax \expandafter\@gobble \fi
        \@tempa
    }%
    \@tempa
}
%    \end{macrocode}
%    \end{macro}
%
%    \begin{macro}{\CurrentBib}
%    In case this is undefined sometimes.
%    \begin{macrocode}
\def\CurrentBib{??}
%    \end{macrocode}
%    \end{macro}
%
%    \begin{macro}{\BibLabel}
%    \begin{macrocode}
\newcommand{\BibLabel}{%
    [\hyper@anchorstart{cite.\CurrentBib}\relax\thebib\hyper@anchorend]%
}
%    \end{macrocode}
%    \end{macro}
%
%    \begin{macro}{\resetbiblist}
%    \begin{macrocode}
\newcommand{\resetbiblist}[1]{%
    \settowidth\labelwidth{\def\thebib{#1}\BibLabel}%
    \leftmargin\labelwidth
    \ifdim\labelwidth=\z@
        \leftmargin=1em
        \itemindent=-\leftmargin
    \else
        \advance\leftmargin\labelsep
    \fi
}
%    \end{macrocode}
%    \end{macro}
%
% \subsection{Processing bibliography entries}
%
%    There are several things one might want to do when a \cn{bib} entry
%    is encountered:
%    \begin{enumerate}
%
%    \item Format and print it.  This corresponds to the direct entry
%    of bibliography items as described in section~2.1 of the users's
%    guide.
%
%    \item Copy it into a \fn{.bbl} file.  This corresponds to the use
%    of \cn{bibselect} and an external \fn{.ltb} database as described
%    in section~2.2 of the user's guide.
%
%    \item Store the full information in memory.  This is done by
%    \cn{bib*}.
%
%    \end{enumerate}
%
%    \begin{macro}{\bib}
%    Here is where the rubber hits the road.
%    \begin{macrocode}
\newcommand{\bib}{%
    \begingroup
        \@ifstar{%
            \@tempswatrue
            \let\@bibdef\star@bibdef
            \BibItem
        }{%
            \@tempswafalse
            \BibItem
        }%
}
%    \end{macrocode}
%    \end{macro}
%
%    \begin{macro}{\BibItem}
%    \MacroArgs
%    \macroarg{1}{citekey}
%    \macroarg{2}{bibtype}
%    \begin{macrocode}
\newcommand{\BibItem}[2]{%
    \def\@tempa{#1}%
    \edef\@tempb{%
        \@nx\@bibdef\@xp\@nx\csname setbib@#2\endcsname{#2}%
            {\macrotext\@tempa}%
    }%
    \@tempb
}
%    \end{macrocode}
%    \end{macro}
%
%    \begin{macro}{\@bibdef}
%    \cs{@bibdef} is a pointer to the procedure that should be handed
%    the entry's key-value pairs.  It has one of four values:
%    \begin{enumerate}
%
%    \item \cs{star@bibdef}
%
%    \item \cs{normal@bibdef}
%
%    \item \cs{copy@bibdef}
%
%    \item \cs{selective@bibdef}
%
%    \end{enumerate}
%    \MacroArgs
%    \macroarg{1}{\cs{setbib@}bibtype}
%    \macroarg{2}{bibtype}
%    \macroarg{3}{citekey}
%    \begin{macrocode}
\AtBeginDocument{\let\@bibdef\normal@bibdef}
%    \end{macrocode}
%    \end{macro}
%
%    \begin{macro}{\bib@exec}
%    And \cs{bib@exec} is a pointer to the procedure that
%    \cs{normal@bibdef} will invoke to process the key-value pairs
%    after they've been  parsed.  It has one
%    of these values:
%    \begin{enumerate}
%
%    \item \cs{bib@store}
%
%    \item \cs{bib@print}
%
%    \end{enumerate}
%    \MacroArgs
%    \macroarg{1}{citekey}
%    \macroarg{2}{\cs{the}\cs{rsk@toks}}
%    \macroarg{3}{\cs{setbib@}bibtype}
%    \begin{macrocode}
\AtBeginDocument{\let\bib@exec\bib@print}
%    \end{macrocode}
%    \end{macro}
%
% \subsubsection{\cs{@bibdef} Implementations}
%
%    \begin{macro}{\normal@bibdef}
%    \MacroArgs
%    \macroarg{1}{\cs{setbib@}bibtype}
%    \macroarg{2}{bibtype}
%    \macroarg{3}{citekey}
%    \begin{macrocode}
\def\normal@bibdef#1#2#3{%
%    \end{macrocode}
%    \cs{CurrentBibType} is used by \pkg{export-bibtex}, but there
%    might be a better way to handle it. (dmj)
%    \begin{macrocode}
    \def\CurrentBibType{#2}%
    \ifx\relax#1%
        \amsrefs@error{Undefined entry type: #2}\@ehc
        \let#1\setbib@misc
    \fi
    \RestrictedSetKeys{}{bib}%
        {\bib@exec{#3}{\the\rsk@toks}{#1}\endgroup}%
}

\let\@bibdef\normal@bibdef
%    \end{macrocode}
%    \end{macro}
%
%    \begin{macro}{\star@bibdef}
%    \MacroArgs
%    \macroarg{1}{\cs{setbib@}bibtype}
%    \macroarg{2}{bibtype}
%    \macroarg{3}{citekey}
%    \begin{macrocode}
\def\star@bibdef{%
    \let\bib@exec\bib@store
    \normal@bibdef
}
%    \end{macrocode}
%    \end{macro}
%
%    \begin{macro}{\copy@bibdef}
%    This is a variation that copies everything into the \fn{.bbl}
%    file.  Used by \cn{bibselect*} and \cn{bib*} inside \fn{.ltb}
%    files.
%    \begin{macrocode}
\def\copy@bibdef{%
    \if@tempswa
        \@xp\defer@bibdef
    \else
        \@xp\copy@bibdef@a
    \fi
}
%    \end{macrocode}
%    \end{macro}
%
%    \begin{macro}{\copy@bibdef@a}
%    \begin{macrocode}
\def\copy@bibdef@a#1#2#3#4{%
    \@open@bbl@file
    \process@xrefs{#4}%
    \bbl@write{%
        \string\bib\if@tempswa*\fi{#3}{#2}\string{\iffalse}\fi
    }%
%    \end{macrocode}
%    Since we're supplying our own definition of \cs{rsk@set}, we
%    don't actually need the group argument, so we leave it out to
%    save a few tokens.
%    \begin{macrocode}
    \RestrictedSetKeys{\global\let\rsk@set\bbl@copy}\@empty
        {\bbl@write{\iffalse{\fi\string}^^J}%
         \endgroup}{#4}%
}
%    \end{macrocode}
%    \end{macro}
%
%    \begin{macrocode}
\catcode`\:=11

\def\modify@xref@fields{%
    \let\set:bib'author\output@xref@a
    \let\set:bib'editor\output@xref@a
    \let\set:bib'translator\output@xref@a
    \let\set:bib'journal\output@xref@a
    \let\set:bib'publisher\output@xref@a
    \def\set:bib'xref##1##2{\output@xref@{##1}\@empty}%
}

\catcode`\:=12

\def\process@xrefs#1{%
    \begingroup
        \RestrictedSetKeys{\modify@xref@fields}{bib}{\the\rsk@toks}{#1}%
    \endgroup
}

\def\output@xref@a#1#2{%
    \def\@tempa{#1}%
    \lowercase{\def\@tempb{#1}}%
    \ifx\@tempa\@tempb
        \output@xref@{#1}%
    \fi
}

\def\output@xref@#1{%
    \@ifnotempty{#1}{%
        \@ifundefined{bi@#1}{}{%
            \begingroup
                \let\star@bibdef\copy@bibdef@a
                \csname bi@#1\endcsname
            \endgroup
        }%
        \@xp\g@undef\csname bi@#1\endcsname
    }%
}
%    \end{macrocode}
%
%    \begin{macro}{\bbl@copy}
%    \begin{macrocode}
\def\bbl@copy#1\endcsname#2{%
    \begingroup
        \def\@tempa{#1}%
        \toks@{{#2}}%
        \star@{\bbl@copy@a}{}%
}
%    \end{macrocode}
%    \end{macro}
%
%    \begin{macro}{\bbl@copy@a}
%    \begin{macrocode}
\def\bbl@copy@a#1{%
        \@ifnotempty{#1}{%
            \add@toks@{*{#1}}%
        }%
        \bbl@write{ \space\@tempa=\the\toks@,}%
    \endgroup
    \rsk@resume
}
%    \end{macrocode}
%    \end{macro}
%
%    \begin{macro}{\selective@bibdef}
%    This is a variation that ignores anything not having a known
%    citation key.  Used by \cn{bibselect}.
%
%    \MacroArgs
%    \macroarg{1}{\cs{setbib@}bibtype}
%    \macroarg{2}{bibtype}
%    \macroarg{3}{citekey}
%    \begin{macrocode}
\def\selective@bibdef#1#2#3{%
    \@xp\selbibdef@a\csname b@#3\endcsname{#1}{#2}{#3}%
}
%    \end{macrocode}
%    \end{macro}
%
%    \begin{macro}{\selbibdef@a}
%    \begin{macrocode}
\def\selbibdef@a#1{%
    \def\@tempa{\endgroup\@gobblefour}%
    \ifx\relax#1\else \@xp\selbibdef@b#1\@nil \fi
    \@tempa
}
%    \end{macrocode}
%    \end{macro}
%
%    \begin{macro}{\selbibdef@b}
%    \begin{macrocode}
\def\selbibdef@b#1#2#3\@nil{%
    \ifx 1#2\let\@tempa\copy@bibdef\fi
}
%    \end{macrocode}
%    \end{macro}
%
%    \begin{macro}{\defer@bibdef}
%    This is a variation that ignores anything not having a known
%    citation key.  Used by \cn{bibselect}.
%
%    \MacroArgs
%    \macroarg{1}{\cs{setbib@}bibtype}
%    \macroarg{2}{bibtype}
%    \macroarg{3}{citekey}
%    \macroarg{4}{key-val pairs}
%    \begin{macrocode}
\def\defer@bibdef#1#2#3#4{%
        \@xp\gdef\csname bi@#3\endcsname{%
            \bib*{#3}{#2}{#4}%
        }%
        \@xp\addto@defer@list \csname bi@#3\endcsname
    \endgroup
}
%    \end{macrocode}
%    \end{macro}
%
%    \begin{macro}{\bibdefer@list}
%    \begin{macrocode}
\let\bibdefer@list\@empty
%    \end{macrocode}
%    \end{macro}
%
%    \begin{macro}{\addto@defer@list}
%    \begin{macrocode}
\def\addto@defer@list#1{%
    \begingroup
        \def\do{\@nx\do\@nx}%
        \xdef\bibdefer@list{\bibdefer@list\do#1}%
    \endgroup
}
%    \end{macrocode}
%    \end{macro}
%
% \subsubsection{\cs{bib@exec} Implementations}
%
%    \begin{macro}{\bib@store}
%    This is the easy one.  It just stores the entire set of key-value
%    pairs in \cs{bi@}\emph{citekey}.
%    \begin{macrocode}
\def\bib@store#1{%
    \afterassignment\@gobble
    \@xp\xdef\csname bi@#1\endcsname
}
%    \end{macrocode}
%    \end{macro}
%
%    \begin{macro}{\numeric@refs}
%    \begin{macrocode}
\def\numeric@refs{00}
%    \end{macrocode}
%    \end{macro}
%
%    \begin{macro}{\bib@print}
%    \MacroArgs
%    \macroarg{1}{citekey}
%    \macroarg{2}{\cs{the}\cs{rsk@toks}}
%    \macroarg{3}{\cs{setbib@}bibtype}
%    \begin{macrocode}
\def\bib@print#1#2#3{%
    \bib@start{#1}%
        \let\setbib@@#3%
        #2\relax       % execute definitions locally
        \bib@resolve@xrefs
        \bib@field@patches
        \bib@selectlanguage
        \generate@label
        \bib'setup
        \bib@cite{#1}%
        \kern\@ne sp
        \ifx\setbib@@\setbib@article
            \ifx\bib'booktitle\@empty
                \ifx\bib'book\@empty
                    \ifx\bib'conference\@empty
                    \else
                        \let\setbib@@\setbib@incollection
                    \fi
                \else
                    \let\setbib@@\setbib@incollection
                \fi
            \else
                \let\setbib@@\setbib@incollection
            \fi
        \fi
        \setbib@@
    \bib@end
}
%    \end{macrocode}
%    \end{macro}
%
%    \begin{macro}{\bib@print@inner}
%    Note that the order of the arguments is reversed with respect to
%    \cs{bib@print}.  Maybe that isn't such a great idea.
%    \MacroArgs
%    \macroarg{1}{\cs{setbib@}bibtype}
%    \macroarg{2}{\cs{the}\cs{rsk@toks}}
%    \begin{macrocode}
\def\bib@print@inner#1#2{%
    \begingroup
        #2\relax       % execute definitions locally
        \bib@field@patches
        \bib'setup
        #1%
    \endgroup
}
%    \end{macrocode}
%    \end{macro}
%
%    \begin{macro}{\current@citekey}
%    \begin{macrocode}
\let\current@citekey\@empty
%    \end{macrocode}
%    \end{macro}
%
%    \begin{macro}{\prev@citekey}
%    \begin{macrocode}
\let\prev@citekey\@empty
%    \end{macrocode}
%    \end{macro}
%
%    \begin{macro}{\bib@start}
%    There used to be more to it.
%    \begin{macrocode}
\def\bib@start#1{%
    \begingroup
        \def\current@citekey{#1}%
}
%    \end{macrocode}
%    \end{macro}
%
%    \begin{macro}{\bib@end}
%    Instead of being handled by \cs{bib@end}, ending punctuation is
%    normally handled via the \fld{transition} field (q.v.)
%    \begin{macrocode}
\def\bib@end{%
        \relax
        \@xp\PrintBackRefs\@xp{\CurrentBib}%
        \par
        \save@primary
        \global\let\prev@citekey\current@citekey
    \endgroup
}
%    \end{macrocode}
%    \end{macro}
%
% \subsubsection{Resolving cross-references}
%
%    \begin{macro}{\bib@resolve@xrefs}
%    \begin{macrocode}
\def\bib@resolve@xrefs{%
    \xref@check@c\bib'xref
    \xref@check@a\bib'author
    \xref@check@a\bib'editor
    \xref@check@a\bib'translator
    \xref@check@b\bib'journal
    \xref@check@b\bib'publisher
}
%    \end{macrocode}
%    \end{macro}
%
%    \begin{macro}{\xref@check@a}
%    Resolve a contributor (typically a \cn{DefineName}) alias.
%    Requires rebuilding the list.
%    \begin{macrocode}
\def\xref@check@a#1{%
    \ifx\@empty#1\relax
    \else
        \begingroup
            \toks@\@emptytoks
            \@temptokenb\@emptytoks
            \series@index\z@
            \def\name{\xref@check@aa#1}%
            #1\relax
            \edef\@tempa{%
                \def\@nx#1{\the\toks@}%
                \the\@temptokenb
            }%
        \@xp\endgroup
        \@tempa
    \fi
}
%    \end{macrocode}
%    \end{macro}
%
%    \begin{macro}{\xref@check@aa}
%    \begin{macrocode}
\def\xref@check@aa#1#2{%
    \advance\series@index\@ne
    \def\@tempa{#2}%
    \lowercase{\def\@tempb{#2}}%
    \ifx\@tempa\@tempb
        \ifx\@tempa\@empty
            \add@toks@{\name{}}%
        \else
            \@ifundefined{bi@#2}{%
                \BibAbbrevWarning{#2}%
                \add@toks@{\name{#2}}%
            }{%
                \xref@check@ab#1{#2}%
            }%
        \fi
    \else
        \add@toks@{\name{#2}}%
    \fi
}
%    \end{macrocode}
%    \end{macro}
%
%    \begin{macro}{\xref@check@ab}
%    \begin{macrocode}
\def\xref@check@ab#1#2{%
    \csname bi@#2\endcsname
    \ifx\@empty\bib'name
        \@temptokena{#2}%
    \else
        \@temptokena\@xp{\bib'name}%
        \get@property\@tempa\bib'name
        \edef\@tempa{%
            \@nx\addto@hook\@temptokenb{%
                \@nx\reset@nth@property\@nx#1\the\series@index{\@tempa}%
            }%
        }%
        \@tempa
    \fi
    \edef\@tempa{\@nx\add@toks@{\@nx\name{\the\@temptokena}}}%
    \@tempa
}
%    \end{macrocode}
%    \end{macro}
%
%    \begin{macro}{\xref@check@b}
%    Resolve a \fld{journal} or \fld{publisher} alias (typically a
%    \cn{DefinePublisher} or \cn{DefineJournal} alias).
%    \begin{macrocode}
\def\xref@check@b#1{%
    \ifx\@empty#1%
    \else
        \toks@\@xp{#1}%
        \edef\@tempb{\lowercase{\def\@nx\@tempa{\the\toks@}}}%
        \@tempb
        \ifx\@tempa#1\relax % all lowercase
            \@ifundefined{bi@#1}{%
                \BibAbbrevWarning{#1}%
            }{%
%    \end{macrocode}
%    We pass control to \cs{xref@check@c} here to handle inheritance
%    of multiple fields properly.  This means some of the checking
%    we've just done gets done again, but I can live with that.
%    \begin{macrocode}
                \let#1\@empty
                \xref@check@c\@tempa
            }%
        \fi
    \fi
}
%    \end{macrocode}
%    \end{macro}
%
%    \begin{macro}{\xref@check@c}
%    Resolve an \fld{xref} field.
%    \begin{macrocode}
\def\xref@check@c#1{%
    \ifx#1\@empty
    \else
        \begingroup
            \let\DSK@def\xref@add@toks
            \let\DSK@append\xref@append
            \toks@\@emptytoks
            \let\bib@reset\@empty
%    \end{macrocode}
%    The \cs{@for} here is just a fancy way of expanding \arg{1}. (Or
%    is it?)
%    \begin{macrocode}
            \@for\xref@ID:=#1\do{%
                \@ifundefined{bi@\xref@ID}{%
                    \XRefWarning{\xref@ID}%
                }{%
                    \csname bi@\xref@ID\endcsname
                }%
            }%
            \edef\@tempa{\endgroup\the\toks@}%
        \@tempa
    \fi
}
%    \end{macrocode}
%    \end{macro}
%
%    \begin{macro}{\xref@add@toks}
%    If any title occurs in an \fld{xref}ed item, assume that it is a book
%    title. This might not always be the best assumption? Let's see how
%    it goes though. [mjd,2001-12-11]
%
%    \MacroArgs
%    \macroarg{1}{\cs{bib'}field}
%    \macroarg{2}{value}
%    \begin{macrocode}
\def\xref@add@toks#1#2#3{%
    \ifx#1\@empty
        \edef\@tempa{%
            \@nx\add@toks@{\@xp\@nx\csname\rkv@setter#1\endcsname{#2}{#3}}%
        }%
        \@tempa
    \else
        \in@\bib'title{#1}%
        \ifin@
            \ifx\bib'booktitle\@empty
                \edef\@tempa{%
                    \@nx\add@toks@{%
                        \@xp\@nx\csname set:bib'booktitle\endcsname
                    }%
                }%
                \@tempa
                \add@toks@{{#2}{#3}}%
            \fi
        \fi
    \fi
}
%    \end{macrocode}
%    \end{macro}
%
%    \begin{macrocode}
\def\xref@append#1#2#3#4{%
    \edef\@tempa{%
        \@nx\add@toks@{\@xp\@nx\csname\rkv@setter#2\endcsname{#3}{#4}}%
    }%
    \@tempa
}
%    \end{macrocode}
%
%    \begin{macro}{\BibAbbrevWarning}
%    \begin{macrocode}
\def\BibAbbrevWarning#1{\amsrefs@warning{Abbreviation '#1' undefined}}
%    \end{macrocode}
%    \end{macro}
%
%    \begin{macro}{\XrefWarning}
%    \begin{macrocode}
\def\XRefWarning#1{\amsrefs@warning{Xref '#1' undefined}}
%    \end{macrocode}
%    \end{macro}
%
% \subsubsection{Bib field preprocessing}
%
%    \begin{macro}{\current@primary}
%    \begin{macrocode}
\let\current@primary\@empty
%    \end{macrocode}
%    \end{macro}
%
%    \begin{macro}{\previous@primary}
%    \begin{macrocode}
\let\previous@primary\@empty
%    \end{macrocode}
%    \end{macro}
%
%    \begin{macro}{\save@primary}
%    \begin{macrocode}
\IfOption{nobysame}{%
    \let\save@primary\@empty
}{%
    \def\save@primary{%
        \global\let\previous@primary\current@primary
    }%
}
%    \end{macrocode}
%    \end{macro}
%
%    \begin{macro}{\bib@field@patches}
%    \begin{macrocode}
\def\bib@field@patches{%
    \ifx\bib'author\@empty
        \ifx\bib'editor\@empty
            \let\current@primary\bib'translator
            \let\print@primary\PrintTranslatorsA
        \else
            \let\current@primary\bib'editor
            \let\print@primary\PrintEditorsA
        \fi
    \else
        \let\current@primary\bib'author
        \let\print@primary\PrintAuthors
    \fi
    \ifx\bib'address\@empty
        \let\bib'address\bib'place
    \fi
    \ifx\bib'organization\@empty
        \ifx\bib'institution\@empty
            \let\bib'organization\bib'school
        \else
            \let\bib'organization\bib'institution
        \fi
    \fi
    \ifx\bib'date\@empty
        \ifx\bib'year\@empty
            \let\bib@year\bib'status
        \else
            \bib@parsedate\bib'year
        \fi
    \else
        \bib@parsedate\bib'date
    \fi
%    \end{macrocode}
%    Example~21 on page~74 of \emph{Mathematics into
%    Type}~\cite{SOS99} seems to indicate that when the year serves as
%    the volume number, the date should be suppressed.  If so, this is
%    where that is done.
%    \begin{macrocode}
    \def\@tempa{year}%
    \ifx\bib'volume\@tempa
        \let\bib'volume\bib@year
        \let\bib'date\@empty
    \fi
%    \end{macrocode}
%    \cs{bib'language} is used for producing the printed rendition of
%    the language. \cs{bib@language} needs to be in the form required by
%    \cn{selectlanguage}.
%    \begin{macrocode}
    \bib@language@fixup
}
%    \end{macrocode}
%    \end{macro}
%
% \subsubsection{Date setup}
%
%    \begin{macro}{\bib@year}
%    \begin{macrocode}
\let\bib@year\@empty
%    \end{macrocode}
%    \end{macro}
%
%    \begin{macro}{\bib@month}
%    \begin{macrocode}
\let\bib@month\@empty
%    \end{macrocode}
%    \end{macro}
%
%    \begin{macro}{\bib@day}
%    \begin{macrocode}
\let\bib@day\@empty
%    \end{macrocode}
%    \end{macro}
%
%    \begin{macro}{\bib@parsedate}
%    Parse an ISO 8601 date into its year, month and day components,
%    but without actually verifying that any of the components are
%    numeric.  Hmmm.
%    \begin{macrocode}
\def\bib@parsedate#1{%
    \@xp\bib@parsedate@a#1---\@nil
}
%    \end{macrocode}
%    \end{macro}
%
%    \begin{macro}{\bib@parsedate@a}
%    \begin{macrocode}
\def\bib@parsedate@a#1-#2-#3-#4\@nil{%
    \def\bib@year{#1}%
    \def\bib@month{#2}%
    \def\bib@day{#3}%
%    \end{macrocode}
%    The rest of this macro tries to rewrite \cs{bib'date} into a
%    normalized form.  I'm not sure if this is a good idea.
%    \begin{macrocode}
    \ifx\@empty\bib@day
        \ifx\@empty\bib@month
            \let\bib'date\bib@year
        \else
            \def\bib'date{#1-#2}%
        \fi
    \else
        \def\bib'date{#1-#2-#3}%
    \fi
}
%    \end{macrocode}
%    \end{macro}
%
% \subsubsection{Language setup}
%
%    \begin{macro}{\bib@language@fixup}
%    \begin{macrocode}
\def\bib@language@fixup{%
    \ifx\bib'hyphenation\@empty
        \ifx\bib'language\@empty
            \let\bib@language\biblanguagedefault
        \else
            \let\bib@language\bib'language
        \fi
    \else
        \let\bib@language\bib'hyphenation
    \fi
    \def\@tempa##1 ##2\@nil{\lowercase{\def\bib@language{##1}}}%
%    \end{macrocode}
%    The mysterious \cs{@firstofone} here is to preserve the space
%    before the \cs{@nil}.
%    \begin{macrocode}
    \@firstofone{\@xp\@tempa\bib@language} \@nil
}
%    \end{macrocode}
%    \end{macro}
%
%    \begin{macro}{\bib@selectlanguage}
%    For \cn{bib} purposes we are interested mainly in testing whether
%    the hyphenation patterns are the same. So we use an
%    if-same-patterns test (by which \pkg{babel}'s `english' and
%    `american' compare as equal) rather than an if-same-language test.
%    Also, the way that the \cn{selectlanguage} command checks to see
%    whether a language has been properly defined for babel use is to
%    see if \cs{dateLANGUAGE} is defined. And if we tried to select an
%    undefined language, the result would be a \lat/ error.
%    \begin{macrocode}
\def\bib@selectlanguage{%
    \@ifsame@patterns{\languagename}{\bib@language}{}{%
        \@ifundefined{date\bib@language}{}{%
            \@xp\selectlanguage\@xp{\bib@language}%
        }%
    }%
}
%    \end{macrocode}
%    \end{macro}
%
%    \begin{macro}{\@ifsame@patterns}
%    \begin{macrocode}
\def\@ifsame@patterns#1#2{%
    \@xp\@ifsamepat\csname l@#1\@xp\endcsname\csname l@#2\endcsname
}
%    \end{macrocode}
%    \end{macro}
%
%    \begin{macro}{\@ifsamepat}
%    \begin{macrocode}
\def\@ifsamepat#1#2{%
    \ifnum \ifx\relax#1\m@ne\else#1\fi = \ifx\relax#2\m@ne\else#2\fi
        \@xp\@firstoftwo
    \else
        \@xp\@secondoftwo
    \fi
}
%    \end{macrocode}
%    \end{macro}
%
%    \begin{macro}{\languagename}
%    \begin{macro}{\biblanguageEnglish}
%    \begin{macro}{\biblanguagedefault}
%    \begin{macro}{\bib@language}
%    \begin{macrocode}
\providecommand{\languagename}{english}
\def\biblanguageEnglish{english}
\let\biblanguagedefault\biblanguageEnglish
\let\bib@language\@empty
%    \end{macrocode}
%    \end{macro}
%    \end{macro}
%    \end{macro}
%    \end{macro}
%
% \subsubsection{Citation label setup}
%
%    \begin{macro}{\generate@label}
%    \begin{macrocode}
\let\generate@label\relax
%    \end{macrocode}
%    \end{macro}
%
%    \begin{macro}{\cite@label}
%    \begin{macrocode}
\def\cite@label{\@currentlabel}
%    \end{macrocode}
%    \end{macro}
%
%    \begin{macro}{\alpha@label}
%    \begin{macrocode}
\let\alpha@label\relax
%    \end{macrocode}
%    \end{macro}
%
%    \begin{macro}{\bib@cite}
%    When \cs{bib@cite} is called, author name and year are available
%    in \cs{bib@author} and \cs{bib@year}.
%    \MacroArgs
%    \macroarg{1}{citekey}
%    \begin{macrocode}
\def\bib@cite#1{%
    \def\CurrentBib{#1}%
    \alpha@label            % modify \thebib if necessary
    \item\leavevmode
    \SK@\SK@@label{#1}%
    \@xp\bib@cite@a\csname b@#1\endcsname
    \bibcite@write{#1}%
}
%    \end{macrocode}
%    \end{macro}
%
%    \begin{macrocode}
\def\bib@cite@a#1{%
    \ifx\relax#1%
        \begingroup
            \auto@protect\etaltext
            \protected@edef\@tempa{%
                \gdef\@nx#1{%
                    \@nx\citesel 01{\cite@label}{\bib@label@year}{}%
                }%
            }%
        \@xp\endgroup
        \@tempa
    \else
        \@xp\bib@cite@check\@xp#1#1\@empty\@empty\@empty\@empty\@empty
    \fi
}
%    \end{macrocode}
%
%    \begin{macro}{\bib@cite@check}
%    For the citation key we want to check if it is already defined. But
%    there is a slight problem. There is already one control
%    sequence in use for each bibliography entry, to store the label or
%    the author/year information needed by \cn{cite}. If we introduce
%    another control sequence to check whether a particular cite is
%    multiply defined, then we double the number of control sequences
%    used. For a large bibliography in a book this is fairly serious.
%    This is addressed by using a \cs{citesel} function.
%
%    \MacroArgs
%    \macroarg{1}{\cs{b@}citekey}
%    \macroarg{2}{\cs{citesel}}
%    \macroarg{3}{cited?}
%    \macroarg{4}{used?}
%    \macroarg{5}{label}
%    \macroarg{6}{year}
%    \macroarg{7}{backrefs}
%    \begin{macrocode}
\def\bib@cite@check#1#2#3#4#5#6#7{%
    \ifx 1#4\relax
        \DuplicateBibKeyWarning
    \else
%    \end{macrocode}
%    This has gotten \emph{way} out of hand.
%    \begin{macrocode}
        \begingroup
            \auto@protect\etaltext
            \@apply\auto@protect\amsrefs@textsymbols
            \@apply\auto@protect\amsrefs@textaccents
            \@tempswafalse
            \in@\CitePrintUndefined{#5}%
            \ifin@
                \let\@tempa\@empty
            \else
                \def\@tempa{#5}%
            \fi
            \ifx\@tempa\@empty
            \else
                \@xp\ifx\@xp\@currentlabel\cite@label
                    \edef\@tempb{\cite@label}%
                \else
                    \let\@tempb\cite@label
                \fi
                \ifx\@tempa\@tempb
                    \def\@tempa{#6}%
                    \ifx\@tempa\bib@label@year
                    \else
                        \@tempswatrue
                    \fi
                \else
                    \@tempswatrue
                \fi
            \fi
            \if@tempswa
                \@ifempty{#6}{%
                    \def\@tempa{#5}%
                    \let\@tempb\cite@label
                }{%
                    \def\@tempa{#5, #6}%
                    \def\@tempb{\cite@label, \bib@label@year}%
                }%
                \amsrefs@warning{Citation label for \extr@cite#1 is
                    changing from `\@tempa ' to `\@tempb '}%
            \fi
            \protected@edef\@tempa{%
                \gdef\@nx#1{%
                    \@nx\citesel #31{\cite@label}{\bib@label@year}{#7}%
                }%
            }%
        \@xp\endgroup
        \@tempa
    \fi
}
%    \end{macrocode}
%    \end{macro}
%
%    \begin{macro}{\bib@label@year}
%    \begin{macrocode}
\let\bib@label@year\@empty
%    \end{macrocode}
%    \end{macro}
%
%    \begin{macro}{\DuplicateBibKeyWarning}
%    \begin{macrocode}
\def\DuplicateBibKeyWarning{%
    \amsrefs@warning{%
        Duplicate \protect\bib\space key
        `\CurrentBib ' detected\MessageBreakNS}%
}
%    \end{macrocode}
%    \end{macro}
%
%    \begin{macro}{\bibcite@write}
%    \begin{macrocode}
\def\bibcite@write#1{%
    \if@filesw
        \let\citesel\citesel@write
        \csname b@#1\endcsname
    \fi
}
%    \end{macrocode}
%    \end{macro}
%
%    \begin{macro}{\citesel@write}
%    \begin{macrocode}
\def\citesel@write#1#2#3#4#5{%
    \begingroup
        \toks@{{#3}{#4}}%
        \immediate\write\@auxout{\string\bibcite{\CurrentBib}{\the\toks@}}%
    \endgroup
}
%    \end{macrocode}
%    \end{macro}
%
%    Because duplicate bibs are caught immediately, we don't need
%    \cs{bibcite} to run \cs{@testdef}.
%    \begin{macrocode}
\AtEndDocument{\let\bibcite\@gobbletwo}
%    \end{macrocode}
%
%    \subsubsection{Printing the bibliography}
%
%    \begin{macro}{\bibname}
%    \begin{macrocode}
\providecommand{\bibname}{Bibliography}
%    \end{macrocode}
%    \end{macro}
%
%    \begin{macro}{\refname}
%    \begin{macrocode}
\providecommand{\refname}{References}
%    \end{macrocode}
%    \end{macro}
%
%    \begin{environment}{bibchapter}
%    We need to take a little extra trouble here to pre-expand the
%    \cn{bibname}.
%    \begin{macrocode}
\newenvironment{bibchapter}[1][\bibname]{%
    \begingroup
        \protected@edef\@{\endgroup\protect\chapter*{#1}}%
        \@
}{\par}
%    \end{macrocode}
%    \end{environment}
%
%    \begin{environment}{bibsection}
%    And here to pre-expand the \cn{refname}.
%    \begin{macrocode}
\newenvironment{bibsection}[1][\refname]{%
    \begingroup
        \protected@edef\@{\endgroup\protect\section*{#1}}%
        \@
}{\par}
%    \end{macrocode}
%    \end{environment}
%
%    \begin{environment}{bibdiv}
%    Here we try to guess whether this is a book-like document or an
%    article-like document.
%    \begin{macrocode}
\@ifundefined{chapter}{%
    \newenvironment{bibdiv}{\bibsection}{\endbibsection}
}{%
    \newenvironment{bibdiv}{\bibchapter}{\endbibchapter}
}
%    \end{macrocode}
%    \end{environment}
%
%    This is what the standard \cls{book} class has for the bibliography
%    title:
%    \begin{verbatim}
%    \newenvironment{thebibliography}[1]
%         {\chapter*{\bibname
%            \@mkboth{\MakeUppercase\bibname}{\MakeUppercase\bibname}}%
%          \list{\@biblabel{\@arabic\c@enumiv}}%
%\end{verbatim}
%
%    \begin{environment}{thebibliography}
%    \begin{macrocode}
\renewenvironment{thebibliography}[1]{%
    \bibdiv
    \biblist[\resetbiblist{#1}]%
}{%
    \endbiblist
    \endbibdiv
}
%    \end{macrocode}
%    \end{environment}
%
% \subsection{Name, journal and publisher abbreviations}
%
%    The commands \cn{DefineName}, \cn{DefinePublisher}, and
%    \cn{DefineJournal} are provided to make abbreviations a little
%    easier.
%
%    \begin{macro}{\DefineName}
%    \begin{macrocode}
\newcommand{\DefineName}[2]{%
    \bib*{#1}{name}{name={#2}}%
}
%    \end{macrocode}
%    \end{macro}
%
%    \begin{macro}{\DefineJournal}
%    \begin{macrocode}
\newcommand{\DefineJournal}[4]{%
    \bib*{#1}{periodical}{
        issn={#2},
        journal={#4}
    }%
}
%    \end{macrocode}
%    \end{macro}
%
%    \begin{macro}{\DefinePublisher}
%    Note that an explicit \fld{address} field in a \cn{bib} entry will
%    override the \fld{address} supplied as part of a
%    \cn{DefinePublisher}.
%    \begin{macrocode}
\newcommand{\DefinePublisher}[4]{%
    \bib*{#1}{publisher}{%
        publisher={#3},
        address={#4}
    }%
}
%    \end{macrocode}
%    \end{macro}
%
% \subsection{Processing \fn{.ltb} files}
%
%    If you have a file that contains \pkg{amsrefs}-style
%    \cn{bib} entries, you can use it as a database and extract items
%    from it for use in another document. In typical relatively simple
%    scenarios, the extraction can be done by \lat/ itself on the first
%    pass, so that citations in the text will be successfully resolved
%    on the second pass (possibly even the first, depending on what kind
%    of bibliography sorting is used).
%
%    \begin{macro}{\bibselect}
%    \begin{macrocode}
\newcommand{\bibselect}{%
    \@ifstar{%
        \let\@bibdef\copy@bibdef
        \BibSelect
    }{%
        \let\@bibdef\selective@bibdef
        \BibSelect
    }%
}
%    \end{macrocode}
%    \end{macro}
%
%    \begin{macro}{\BibSelect}
%    \begin{macrocode}
\newcommand{\BibSelect}[2][\bblname]{%
    \if@filesw
        \typeout{Trying to create bbl file `#1.bbl' ...}%
        \def\bibselect@msg{%
            \typeout{ ... rats. Unable to create bbl file.}%
        }%
        \let\@open@bbl@file\OpenBBLFile
        \@for\@tempa:=#2\do{\ReadBibData{\@tempa}}%
    \fi
    \@close@bbl@file
    \@apply\g@undef\bibdefer@list
    \global\let\bibdefer@list\@empty
%    \end{macrocode}
%    Now read the \fn{.bbl} file we just created.
%    \begin{macrocode}
    \let\@bibdef\normal@bibdef
    \@input@{#1.bbl}%
    \let\BibSelect\MultipleBibSelectWarning
}
%    \end{macrocode}
%    \end{macro}
%
%    \begin{macro}{\MultipleBibSelectWarning}
%    \begin{macrocode}
\newcommand\MultipleBibSelectWarning[2][]{%
    \amsrefs@warning{%
        Multiple \string\bibselect 's found (only one
        \string\bibselect\space per biblist environment is allowed)%
    }%
}
%    \end{macrocode}
%    \end{macro}
%
%    \begin{macro}{\bblname}
%    \begin{macrocode}
\def\bblname{\jobname}
%    \end{macrocode}
%    \end{macro}
%
%    \begin{macro}{\bib@dbfile}
%    \begin{macrocode}
\newread\bib@dbfile
%    \end{macrocode}
%    \end{macro}
%
%    \begin{macro}{\ReadBibData}
%    \begin{macrocode}
\newcommand{\ReadBibData}[1]{%
    \IfFileExists{#1.ltb}{%
        \openin\bib@dbfile=\@filef@und \relax
    }{%
        \IfFileExists{#1.ltx}{%
            \openin\bib@dbfile=\@filef@und \relax
        }{%
            \IfFileExists{#1.tex}{%
                \openin\bib@dbfile=\@filef@und \relax
            }{%
                \begingroup
                    \NoBibDBFile{#1}%
                    \let\ReadBibData@a\endgroup
            }%
        }%
    }%
    \ReadBibData@a
}
%    \end{macrocode}
%    \end{macro}
%
%    \begin{macro}{\NoBibDBFile}
%    \begin{macrocode}
\def\NoBibDBFile#1{%
    \amsrefs@warning{No data file #1.ltb (.ltx, .tex) found}%
}
%    \end{macrocode}
%    \end{macro}
%
%    \begin{macro}{\ReadBibData@a}
%    \begin{macrocode}
\def\ReadBibData@a{%
    \ProvidesFile{\@filef@und}\relax
    \begingroup
        \let\star@bibdef\defer@bibdef
        \ReadBibLoop
    \endgroup
    \closein\bib@dbfile
}
%    \end{macrocode}
%    \end{macro}
%
%    \begin{macro}{\ReadBibLoop}
%    \begin{macrocode}
\def\ReadBibLoop{%
    \ifeof\bib@dbfile
        \@xp\@gobble
    \else
        \read\bib@dbfile to\CurLine
%    \end{macrocode}
%    The \cs{@empty} is in case \cs{CurLine} is empty.
%    \begin{macrocode}
        \@xp\ReadBibLoop@a\CurLine\@empty\@nil
    \fi
    \ReadBibLoop
}
%    \end{macrocode}
%    \end{macro}
%
%    \begin{macro}{\ReadBibLoop@e}
%    This traps top-level \cn{bib} commands.  Note that:
%    \begin{itemize}
%
%    \item
%    If \cs{CurLine} doesn't contain a complete \cn{bib} entry, the
%    code chokes.
%
%    \item
%    I \cn{bib} is not the very first non-space token in a line, it
%    will not be recognized.
%
%    \end{itemize}
%
%    \begin{macrocode}
\long\def\ReadBibLoop@a#1#2\@nil{%
    \ifx\bib#1%
        \CurLine % just exec it
    \else
%    \end{macrocode}
%    We're not done yet.  The line may contain something like
%    \cn{DefineName}, so we need to expand the first macro in the line
%    and see if it starts with \cn{bib}.  But first we check to make
%    sure that the token we're about to expand isn't \cn{endinput}.
%    \begin{macrocode}
        \ifx\endinput#1%
            \let\ReadBibLoop\@empty
        \else
%    \end{macrocode}
%    And this \cs{@empty} is for the admittedly unlikely case that
%    \cs{CurLine} isn't empty, but its expansion is.
%    \begin{macrocode}
            \@xp\ReadBibLoop@b#1#2\@empty\@nil
        \fi
    \fi
}
%    \end{macrocode}
%    \end{macro}
%
%    \begin{macro}{\ReadBibLoop@b}
%    \begin{macrocode}
\long\def\ReadBibLoop@b#1#2\@nil{%
    \ifx\bib#1%
        \CurLine % just exec it
    \fi
}
%    \end{macrocode}
%    \end{macro}
%
%    \begin{macrocode}
\let\bbl@out=\relax
\let\bbl@write\@gobble
\let\@open@bbl@file\relax
\let\@close@bbl@file\relax
%    \end{macrocode}
%
%    \begin{macro}{\OpenBBLFile}
%    \begin{macrocode}
\def\OpenBBLFile{%
    \if@filesw
        % Just use the next unused output stream
        \count@\count17
        \advance\count@\@ne
        \ifnum\count@<\sixt@@n
            \global\chardef\bbl@out=\count@
            \immediate\openout\bbl@out=\bblname.bbl\relax
            \global\let\@close@bbl@file\CloseBBLFile
            \gdef\bbl@write{\immediate\write\bbl@out}%
        \else
            \ch@ck\count@\sixt@@n\write
        \fi
    \fi
    \global\let\@open@bbl@file\relax
}
%    \end{macrocode}
%    \end{macro}
%
%    \begin{macro}{\CloseBBLFile}
%    \begin{macrocode}
\def\CloseBBLFile{%
    \immediate\closeout\bbl@out\relax
    \global\let\@close@bbl@file\relax
    \global\let\bbl@write\@gobble
    \global\let\bbl@out\relax
}
%    \end{macrocode}
%    \end{macro}
%
% \subsection{Citation processing}
%
% \subsubsection{The \cs{citesel} structure}
%
%    The information used by \cn{cite} for key \texttt{moo} is stored in
%    \cs{b@moo} in the form
%    \begin{verbatim}
%    \citesel{status1}{status2}{label}{year}{backref-info}
%\end{verbatim}%
%
%    The first status flag is~1 if this key has already been cited
%    earlier in the same document; 0~otherwise. This is used in some
%    bibliography schemes to print a full list of author names for the first
%    citation and an abbreviated author list for subsequent citations.
%
%    The second status flag is~1 if this key has already been used by a
%    define-cite command (such as \cn{bib}); 0~otherwise. This makes it
%    possible to issue a warning message as soon as the conflict is seen,
%    on the first \lat/ run, instead of on a subsequent run during the
%    processing of the \fn{.aux} file.
%
%    When an author/year citation scheme is in use, args 3 and~4 hold
%    respectively author names and year. Otherwise arg~3 simply holds
%    a cite label and arg~4 is empty.
%
%    And finally, arg~5 holds a list of backref pointers indicating the
%    locations in the document where this entry has been cited.
%
%    \begin{macro}{\citesel@update}
%    \begin{macrocode}
\def\citesel@update#1#2#3#4#5#6{%
    \gdef#6{\citesel 1#2{#3}{#4}{#5}}%
}
%    \end{macrocode}
%    \end{macro}
%
%    \begin{macro}{\citesel@number}
%    \begin{macrocode}
\def\citesel@number#1#2#3#4#5{#3}
%    \end{macrocode}
%    \end{macro}
%
%    \begin{macro}{\citesel@year}
%    \begin{macrocode}
\def\citesel@year#1#2#3#4#5{#4}
%    \end{macrocode}
%    \end{macro}
%
%    \begin{macro}{\citesel}
%    \begin{macrocode}
\let\citesel\citesel@number
%    \end{macrocode}
%    \end{macro}
%
% \subsubsection{The basic \cn{cite} command}
%
%    Here is the difference between the various optional forms of
%    \cn{cite}:
%    \begin{verbatim}
%    \cite{xyz}        -> \cite@a\citesel{xyz}{}
%                      -> \cite@bc\b@xyz\citesel{}
%
%    \cite{xyz}*{blub} -> \cite@a\citesel{xyz}{blub}
%                      -> \cite@bc\b@xyz\citesel{blub}
%
%    \cite[blub]{xyz}  -> \cite@a\citesel{xyz}{blub}
%                      -> \cite@bc\b@xyz\citesel{blub}
%\end{verbatim}
%
%    Canceling the old \lat/ definition of \verb*'\cite ' prevents
%    certain problems that could arise with the \pkg{showkeys} package.
%    \begin{macrocode}
\expandafter\let\csname cite \endcsname\relax
%    \end{macrocode}
%
%    \begin{macro}{\cite}
%    Need to handle the standard [...] option for compatibility's sake.
%    \begin{macrocode}
\renewcommand{\cite}[2][]{%
    \if\cite@single#2,\@gobble \else\MultipleCiteKeyWarning{#2}{#1}\fi
    \@ifempty{#1}{%
        \cites@o{#2}%
    }{%
        \ObsoleteCiteOptionWarning
        \cites@a{*{#1}}{#2}%
    }%
}
%    \end{macrocode}
%    \end{macro}
%
%    \begin{macro}{\MultipleCiteKeyWarning}
%    \begin{macrocode}
\def\MultipleCiteKeyWarning#1#2{%
    \amsrefs@warning{%
        Use of \string\cites\space is recommended instead of %
        \string\cite\space\MessageBreak
        for multiple cites '#1'}%
    \@ifnotempty{#2}{%
        \amsrefs@warning{Star option requires \string\citelist\space here}%
    }%
    \global\let\MultipleCiteKeyWarning\@gobbletwo
}
%    \end{macrocode}
%    \end{macro}
%
%    \begin{macro}{\ObsoleteCiteOptionWarning}
%    \begin{macrocode}
\def\ObsoleteCiteOptionWarning{%
    \amsrefs@warning{%
        The form \string\cite{...}*{...} is recommended\MessageBreak
        instead of \string\cite[...]{...}}%
    \global\let\ObsoleteCiteOptionWarning\@empty
}
%    \end{macrocode}
%    \end{macro}
%
%    \begin{macro}{\cite@single}
%    \begin{macrocode}
\edef\cite@single#1,#2{\iffalse{\fi\iffalse{\fi\string}#2.\string}}
%    \end{macrocode}
%    \end{macro}
%
%    \begin{macro}{\cites@o}
%    \begin{macrocode}
\def\cites@o#1{\star@{\cites@oo{#1}}{}}
%    \end{macrocode}
%    \end{macro}
%
%    \begin{macro}{\cites@oo}
%    \begin{macrocode}
\def\cites@oo#1#2{\@ifempty{#2}{\cites@a{}{#1}}{\cites@a{*{#2}}{#1}}}
%    \end{macrocode}
%    \end{macro}
%
%    \begin{macro}{\cites@a}
%    \begin{macrocode}
\def\cites@a#1#2{%
    \begingroup
        \toks@{\endgroup \cites@b{#1}}%
        \vdef\@tempa{#2}%
        \edef\@tempa{%
            \the\toks@ \@firstofone{\@xp\zap@space\@tempa} \@empty
        }%
        \@tempa,\@empty
        \edef\@tempa{\endgroup\@nx\citelist{\the\toks@}}%
        \@tempa
}
%    \end{macrocode}
%    \end{macro}
%
%    \begin{macro}{\cites@b}
%    \begin{macrocode}
\def\cites@b#1#2,#3{%
    \begingroup
        \toks@{\InnerCite{#2}#1}%
        \ifx\@empty#3\@xp\@gobble\fi
        \cites@c#3%
}
%    \end{macrocode}
%    \end{macro}
%
%    \begin{macro}{\cites@c}
%    \begin{macrocode}
\def\cites@c#1,#2{%
    \add@toks@{\InnerCite{#1}}%
    \ifx\@empty#2\@xp\@gobble\fi
    \cites@c#2%
}
%    \end{macrocode}
%    \end{macro}
%
%    \begin{macro}{\citeleft}
%    \begin{macro}{\citeright}
%    \begin{macro}{\citemid}
%    \begin{macro}{\citepunct}
%    These variables are named to follow the precedent set by
%    Arseneau's \pkg{cite} package. \cs{citemid} is used to separate a
%    citation label from additional information such as \qq{Theorem
%    4.9}. \cs{citepunct} is used to separate multiple cites, unless
%    one of the cites has additional associated information, in which
%    case \cs{CiteAltPunct} is used.
%    \begin{macrocode}
\def\citeleft{[}
\def\citeright{]}
%    \end{macrocode}
%    \begin{macrocode}
\def\citemid{,\penalty9999 \space}
\def\citepunct{,\penalty9999 \hskip.13em plus.1em minus.05em\relax}
%    \end{macrocode}
%    \end{macro}
%    \end{macro}
%    \end{macro}
%    \end{macro}
%
%    \begin{macro}{\citeAltPunct}
%    When a citation list contains one or more citations with optional
%    arguments, we replace \cs{citemid} by \cs{CiteAltPunct}.
%    \begin{macrocode}
\def\citeAltPunct{;\ }
%    \end{macrocode}
%    \end{macro}
%
%    \begin{macro}{\citeform}
%    This is used for formatting the citation label.  It can be used,
%    for example, to bolden the labels (as in \cls{amsbook} and
%    \cls{amsproc}) or to do more elaborate things such as convert the
%    numbers to roman numerals.  By default, it's just a no-op.
%
%    Note that currently there is no corresponding macro for changing
%    the formatting of \cn{cite}'s optional argument.  This is
%    probably a bug.
%    \begin{macrocode}
\providecommand{\citeform}{\@firstofone}
%    \end{macrocode}
%    \end{macro}
%
%    \begin{macro}{\citelist}
%    The \cs{@citelist} indirection turns out to be helpful in
%    implementing the \cs{ocites} command for the \opt{author-year}
%    option.
%    \begin{macrocode}
\DeclareRobustCommand{\citelist}{\@citelist}
%    \end{macrocode}
%    \end{macro}
%
%    \begin{macro}{\@citelist}
%    \begin{macrocode}
\def\@citelist#1{%
    \leavevmode
    \begingroup
        \@citestyle
        \citeleft\nopunct   % suppress first \citepunct
        \cite@begingroup
            \in@*{#1}%
            \ifin@
                \let\citepunct\citeAltPunct
            \fi
            \let\cite@endgroup\@empty
            \cites@init
            \def\citeleft{\@addpunct{\citepunct}}%
            \let\citeright\ignorespaces
            \def\cite{\InnerCite}%
            \process@citelist{#1}%
        \endgroup
        \citeright
    \endgroup
}
%    \end{macrocode}
%    \end{macro}
%
%    \begin{macro}{\@citestyle}
%    Reset the font to an upright, medium font (e.g. \fnt{cmr}), per
%    AMS style.  Also set $\cs{mathsurround} = 0\,\mathrm{pt}$ just in
%    case there are subscripts in the cite numbers (from
%    \cn{etalchar}, for example).
%    \begin{macrocode}
\providecommand{\@citestyle}{\m@th\upshape\mdseries}
%    \end{macrocode}
%    \end{macro}
%
%    \begin{macro}{\cite@begingroup}
%    Grouping that encloses an entire cite block (a single cite or a
%    list of cites).
%    \begin{macrocode}
\def\cite@begingroup{\begingroup\let\cite@begingroup\relax}
%    \end{macrocode}
%    \end{macro}
%
%    \begin{macro}{\cite@endgroup}
%    \begin{macrocode}
\let\cite@endgroup\endgroup
%    \end{macrocode}
%    \end{macro}
%
%    \begin{macro}{\cites@init}
%    This needs to be called at the beginning of a list of cites to
%    reset a few things.
%    \begin{macrocode}
\def\cites@init{%
    \gdef\prev@names{???}%
    \let\cites@init\@empty
}
%    \end{macrocode}
%    \end{macro}
%
%    \begin{macro}{\InnerCite}
%    \begin{macrocode}
\newcommand{\InnerCite}[1]{\star@{\cite@a\citesel{#1}}{}}
%    \end{macrocode}
%    \end{macro}
%
%    \begin{macro}{\cite@a}
%    The job of \cs{cite@a} is to convert the cite key to all catcode-12
%    characters and remove any spaces it might contain before passing
%    it on to \cs{cite@b}.
%
%    \MacroArgs
%    \macroarg{1}{\cs{CITESEL}}
%    \macroarg{2}{citekey}
%    \begin{macrocode}
\def\cite@a#1#2{%
    \BackCite{#2}%
    \cite@begingroup
        \cites@init
        \let\citesel#1\relax
        \ifx\citesel\citesel@author
            \let\citeleft\@empty
            \let\citeright\@empty
        \fi
        \begingroup
            \toks@{\endgroup \cite@b}%
            \vdef\@tempa{#2}%
            \edef\@tempa{%
                \the\toks@{\@firstofone{\@xp\zap@space\@tempa} \@empty}%
            }%
        \@tempa
}
%    \end{macrocode}
%    \end{macro}
%
%    \begin{macro}{\cite@b}
%    \MacroArgs
%    \macroarg{1}{citekey}
%    \macroarg{2}{star-optional-arg}
%    \begin{macrocode}
\def\cite@b#1#2{%
    \@xp\cite@bc\csname b@#1\@xp\endcsname {#1}{#2}%
}
%    \end{macrocode}
%    \end{macro}
%
%    \begin{macro}{\cite@bc}
%    If it's uninitialized, plug in an empty cite structure.
%    \cs{cite@bc} should be executed only once for a given instance of a
%    cite key. All further processing should go through \cs{cite@cj}.
%    \begin{macrocode}
\def\cite@bc#1#2{%
    \ifx#1\@@undefined \global\let#1\relax \fi
    \ifx#1\relax       \global\let#1\empty@cite \fi
    \@xp\cite@nobib@test#1{}{}{}{}\@nil#1%
    \cite@cj#1%
}
%    \end{macrocode}
%    \end{macro}
%
%    \begin{macro}{\empty@cite}
%    \begin{macrocode}
\def\empty@cite{\citesel 00{}{}{}}
%    \end{macrocode}
%    \end{macro}
%
%    \begin{macro}{\cite@nobib@test}
%    If arg 4 is empty, it means there wasn't any \cn{bib} command that
%    defined a valid label.
%
%    \MacroArgs
%    \macroarg{1}{\cs{citesel}}
%    \macroarg{2}{cited?}
%    \macroarg{3}{used?}
%    \macroarg{4}{label}
%    \macroarg{5}{backrefs}
%    \macroarg{6}{\cs{b@}citekey}
%    \begin{macrocode}
\def\cite@nobib@test#1#2#3#4#5\@nil#6{%
    \@ifempty{#4}{%
        \G@refundefinedtrue
        \UndefinedCiteWarning#6%
        \xdef#6{\@nx\citesel #2#3{%
            \@nx\CitePrintUndefined{\extr@cite#6}}{}{}}%
    }{}%
}
%    \end{macrocode}
%    \end{macro}
%
%    \begin{macro}{\UndefinedCiteWarning}
%    This is a copy of the standard warning from \cs{@citex}.
%    \begin{macrocode}
\def\UndefinedCiteWarning#1{%
    \@latex@warning{%
        Citation `\extr@cite#1' on page \thepage\space undefined}%
}
%    \end{macrocode}
%    \end{macro}
%
%    \begin{macro}{\CitePrintUndefined}
%    \begin{macrocode}
\DeclareRobustCommand{\CitePrintUndefined}[1]{%
    \begingroup\fontshape{n}\fontseries\mddefault \ttfamily ?#1\endgroup
}
%    \end{macrocode}
%    \end{macro}
%
%    \begin{macro}{\CPU@normal}
%    This has to be a \cs{let}, not a \cs{def}.
%    \begin{macrocode}
\let\CPU@normal\CitePrintUndefined
%    \end{macrocode}
%    \end{macro}
%
%    \begin{macro}{\cite@cj}
%    \MacroArgs
%    \macroarg{1}{\cs{b@}citekey}
%    \macroarg{2}{star-optional-arg}
%    \begin{macrocode}
\def\cite@cj#1#2{%
        \leavevmode
            \begingroup
                \cite@cb#1% write info to aux file
                \ar@SK@cite#1%
                \@citeleft
                \ar@hyperlink{#1}%
                \@ifnotempty{#2}{\citemid{#2}}%
                \citeright
            \endgroup
            \ignorespaces % ignore spaces inside \citelist
        \cite@endgroup
}
%    \end{macrocode}
%    \end{macro}
%
%    \begin{macro}{\@citeleft}
%    The following definition provides some indirection that helps to
%    deal with author-year object cites.
%    \begin{macrocode}
\def\@citeleft{\citeleft}
%    \end{macrocode}
%    \end{macro}
%
%    \begin{macro}{\cite@cb}
%    \begin{macrocode}
\def\cite@cb#1{%
    \if@filesw
        \immediate\write\@auxout{\string\citation{\extr@cite#1}}%
    \fi
%    \end{macrocode}
%    Define \cs{citesel} to make \cs{b@whatever} update itself.
%    \begin{macrocode}
    \begingroup
        \let\citesel\citesel@update
        #1#1%
    \endgroup
}
%    \end{macrocode}
%    \end{macro}
%
%    \begin{macro}{\extr@cite}
%    Extract \emph{citekey} from \cs{b@}\emph{citekey}.
%    \begin{macrocode}
\def\extr@cite{\@xp\@gobblethree\string}
%    \end{macrocode}
%    \end{macro}
%
% \subsubsection{Fancier \cn{cite} commands}
%
%    \begin{macro}{\cites}
%    A list of simple cites. Make it robust in case used inside a figure
%    caption. (But then also, by the way, listoffigures should provide
%    special handling.)
%    \begin{macrocode}
\DeclareRobustCommand{\cites}{\cites@a{}}
%    \end{macrocode}
%    \end{macro}
%
%    \begin{macro}{\citen}
%    This is just to keep the showkeys package from clobbering the wrong
%    part of our definition of \cn{cite}:
%    \begin{macrocode}
\providecommand{\citen}{\ocite}
%    \end{macrocode}
%    \end{macro}
%
%    \begin{macro}{\ycite}
%    \cn{cite} gets redefined inside of \cn{citelist}, so we need to
%    \cs{def} \cn{ycite} here instead of just \cs{let}ting everything
%    to \cn{cite}.
%    \begin{macrocode}
\def\ycite{\cite}
%    \end{macrocode}
%    \end{macro}
%
%    \begin{macro}{\ycites}
%    \begin{macrocode}
\let\ycites\cites
%    \end{macrocode}
%    \end{macro}
%
%    \begin{macro}{\ocite}
%    \begin{macrocode}
\let\ocite\ycite
%    \end{macrocode}
%    \end{macro}
%
%    \begin{macro}{\ocites}
%    \begin{macrocode}
\let\ocites\cites
%    \end{macrocode}
%    \end{macro}
%
%    \begin{macro}{\fullcite}
%    \begin{macrocode}
\let\fullcite\cite
%    \end{macrocode}
%    \end{macro}
%
%    \begin{macro}{\fullocite}
%    \begin{macrocode}
\let\fullocite\ocite
%    \end{macrocode}
%    \end{macro}
%
%    \begin{macro}{\citeauthor}
%    \begin{macrocode}
\let\citeauthor\ycite
%    \end{macrocode}
%    \end{macro}
%
%    \begin{macro}{\citeauthory}
%    \begin{macrocode}
\let\citeauthory\ycite
%    \end{macrocode}
%    \end{macro}
%
% \subsubsection{The \cn{nocite} command}
%
%    \begin{macro}{\nocite}
%    \begin{macrocode}
\renewcommand{\nocite}[1]{\othercites{#1}}
%    \end{macrocode}
%    \end{macro}
%
%    \begin{macro}{\othercites}
%    \begin{macrocode}
\newcommand{\othercites}[1]{%
    \cite@begingroup
        \let\cite@endgroup\@empty
        \def\citelist{\othercitelist}%
        \cites{#1}%
}
%    \end{macrocode}
%    \end{macro}
%
%    \begin{macro}{\othercitelist}
%    \begin{macrocode}
\newcommand{\othercitelist}[1]{%
    \cite@begingroup
        \let\cite@endgroup\@empty
        \cites@init
        \let\citeleft\relax
        \let\citeright\ignorespaces
        \def\InnerCite{\OtherCite}%
        \def\cite@cj ##1##2{%
            \begingroup
                \@xp\citesel##1%
                \cite@cb ##1%
            \endgroup
%    \end{macrocode}
%    If we detect |\nocite{*}|, we globally alias
%    \cs{selective@bibdef} to \cs{copy@bibdef} so that all succeeding
%    \cn{bibselect} commands act like \cn{bibselect*}.
%    \begin{macrocode}
            \@xp\ifx\csname b@*\endcsname ##1%
                \global\let\selective@bibdef\copy@bibdef
            \fi
            \ignorespaces
            \cite@endgroup
        }%
    #1\relax
    \endgroup
}
%    \end{macrocode}
%    \end{macro}
%
%    \begin{macro}{\OtherCite}
%    \begin{macrocode}
\def\OtherCite#1{\cite@a\citesel@other{#1}{}{}}
%    \end{macrocode}
%    \end{macro}
%
%    \begin{macro}{\citesel@other}
%    \begin{macrocode}
\def\citesel@other#1#2#3#4#5#6{}
%    \end{macrocode}
%    \end{macro}
%
%    \begin{macro}{\b@*}
%    This provides a dummy definition to keep things like |\nocite{*}|
%    from generating an error message.
%    \begin{macrocode}
\@namedef{b@*}{\citesel 11{*}{*}{*}}
%    \end{macrocode}
%    \end{macro}
%
% \subsubsection{Citation sorting}
%
%    \begin{macro}{\process@citelist@sorted}
%    \begin{macrocode}
\def\process@citelist@sorted#1{%
    \ifx\citesel\citesel@number
        \cite@sorted@s #1\cite@sorted@e
    \else
        \NonNumericCiteWarning
        \process@citelist@unsorted{#1}%
    \fi
}
%    \end{macrocode}
%    \end{macro}
%
%    \begin{macro}{\NonNumericCiteWarning}
%    \begin{macrocode}
\def\NonNumericCiteWarning{%
    \amsrefs@warning{%
        Unable to confirm that cite keys are numeric: not sorting%
    }%
}
%    \end{macrocode}
%    \end{macro}
%
%    \begin{macro}{\process@citelist@unsorted}
%    \begin{macrocode}
\def\process@citelist@unsorted#1{%
    \ignorespaces#1\relax
}
%    \end{macrocode}
%    \end{macro}
%
%    \begin{macro}{\process@citelist}
%    By default, citation lists will be sorted.
%    \begin{macrocode}
\let\process@citelist\process@citelist@sorted
%    \end{macrocode}
%    \end{macro}
%
%    \begin{macro}{\CPU@sort}
%    By defining this as \TeX's maxint, undefined cites migrate to the
%    end of a sorted list.
%    \begin{macrocode}
\def\CPU@sort#1{2147483647}
%    \end{macrocode}
%    \end{macro}
%
%    \begin{macro}{\cite@sorted@s}
%    Here's where we prepare to sort the citations and (optionally)
%    compress ranges.
%    \begin{macrocode}
\def\cite@sorted@s{%
    \begingroup
        \let\CitePrintUndefined\CPU@sort
        \let\cite@cjs\cite@cj
        \let\cite@cj\cite@compress
        \begingroup
            \toks@\@emptytoks
            \let\cite@cj\cite@sort
            \ignorespaces
}
%    \end{macrocode}
%    \end{macro}
%
%    \begin{macro}{\cite@sorted@e}
%    \begin{macrocode}
\def\cite@sorted@e{%
        \@xp\endgroup
        \the\toks@
        \cite@dash
        \prev@cite
    \endgroup
}
%    \end{macrocode}
%    \end{macro}
%
%    \begin{macro}{\cite@sort}
%    This is essentially an insertion sort.  I think.
%
%    \MacroArgs
%    \macroarg{1}{\cs{b@}citekey}
%    \macroarg{2}{optional arg}
%    \begin{macrocode}
\def\cite@sort#1#2{%
    \safe@set\@tempcnta#1% highest number so far
    \toks@{\cite@cj#1{#2}}%
    \@temptokena\toks@
    \let\cite@cj\cite@sort@a
    \ignorespaces
}
%    \end{macrocode}
%    \end{macro}
%
%    \begin{macro}{\cite@sort@a}
%    \begin{macrocode}
\def\cite@sort@a#1#2{%
    \safe@set\@tempcntb#1%
    \ifnum\@tempcntb > \@tempcnta
        \add@toks@{\cite@cj#1{#2}}%
        \@tempcnta\@tempcntb
    \else
        \let\cite@cj\cite@sort@b
        \toks@\@emptytoks
        \def\@tempb{\add@toks@{\cite@cj#1{#2}}}%
        \the\@temptokena
        \@tempb
        \let\cite@cj\cite@sort@a
    \fi
    \@temptokena\toks@
    \ignorespaces
}
%    \end{macrocode}
%    \end{macro}
%
%    \begin{macro}{\cite@sort@b}
%    \begin{macrocode}
\def\cite@sort@b#1#2{%
    \safe@set\count@#1%
    \ifnum\@tempcntb < \count@
        \@tempb
        \let\@tempb\@empty
    \fi
    \add@toks@{\cite@cj#1{#2}}%
    \ignorespaces
}
%    \end{macrocode}
%    \end{macro}
%
% \subsubsection{Range compression}
%
%    When the time comes to apply compression, we have at our disposal a
%    list of internal cite calls that looks like this:
%    \begin{verbatim}
%    \cite@cj\b@aaa{opta}\cite@cj\b@bbb{optb}...\cite@cj\b@zzz{optz}
%\end{verbatim}
%    where
%    \[ \ncn{b@aaa} < \ncn{b@bbb} < \cdots < \ncn{b@zzz} \]
%    and the \texttt{opt} arguments are possibly null.
%    To print the citations while collapsing sequences of 3 or more
%    contiguous numbers into ranges of the form $n$--$m$, we bind
%    \cs{cite@cj} to a suitably clever function and then execute the
%    list.  In the absence of optional arguments, here's the algorithm:
%
% \begin{enumerate}
% \item[Begin.]
%    Enter state~0.  This is done by \cs{cite@sorted@s}.
% \item[State 0.]
%^^A    This is the initial state.
%    The current citation is the beginning
%    of a range (possibly a singleton range).  Print it.  Then, set
%    $\mathit{prevnum} := \mathit{number}$ and enter state~1.
%
% \item[State 1.]
%    The current citation might be the second element of a range.
%    \begin{enumerate}
%    \item[Case a)]
%    $\mathit{number} = \mathit{prevnum} + 1$. Then the current
%    item is definitely the second element of a range.  It might be
%    the last element of the range, but we won't know until we examine
%    the following citation.  So, save the current citation in
%    \cs{prev@cite}, set $\mathit{prevnum} := \mathit{number}$, and go
%    to state~2.
%
%    \item[Case b)]
%    $\mathit{number} \ne \mathit{prevnum} + 1$.  The current citation
%    is the beginning of a new range.  Print it, set
%    $\mathit{prevnum} := \mathit{number}$ and remain in state~1.
%    (This is essentially identical to stage~0.)
%
%    \end{enumerate}
%
% \item[State 2.]\leavevmode
%    The current citation might be the third (or later) element of a
%    range.
%    \begin{enumerate}
%   \item[Case a)]
%    $\mathit{number} = \mathit{prevnum} + 1$.  The current element is
%    definitely part of a range.  It might be the last
%    element of the range, but again we won't know until we examine
%    the following citation.  Save the current citation
%    in \cs{prev@cite} and set $\mathit{prevnum} := \mathit{number}$.
%    Remain in state~2.
%
% \item[Case b)]
%    $\mathit{number} \ne \mathit{prevnum} + 1$.  The previous
%    citation was the end of a range and the current citation is the
%    beginning of a new range.  Print a dash followed by
%    \cs{prev@cite}, then set $\mathit{prevnum} := \mathit{number}$
%    and enter state~1.
%
%    \end{enumerate}
%
% \item[End.] If \cs{prev@cite} is not empty, print it, preceded by a
%    dash if we were in the middle of a range.  (This is done by
%    \cs{cite@sorted@e}.)
%
% \end{enumerate}
%
%    The presence of optional arguments complicates things
%    somewhat, since a citation with an optional argument should never
%    participate in range compression.  In other words, when we come
%    across an optional argument, we should finish off the preceding
%    range, print the current citation, and then return to the initial
%    state.  More precisely, here are the actions taken in each state
%    when there is an optional argument:
%
%    \begin{enumerate}
%
%    \item[State 0.]
%    Print the current citation and remain in state~0.
%
%    \item[State 1.]
%    Print the current citation and return to state~0.
%
%    \item[State 2.]
%    Print a dash followed by \cs{prev@cite}.  Then print the current
%    citation and return to state~0.
%
%    \end{enumerate}
%
%    \begin{macro}{\prev@cite}
%    \begin{macrocode}
\let\prev@cite\@empty
%    \end{macrocode}
%    \end{macro}
%
%    \begin{macro}{\prev@cite@cb}
%    There's one further complication: Even though we're suppressing
%    some of the citation numbers, we need to make sure that each
%    citation is recorded in the \fn{.aux} file.  So, in case~2a,
%    before we overwrite \cs{prev@cite}, we first invoke
%    \cs{prev@cite@cb} to record the previous citation (if any).
%    \begin{macrocode}
\def\prev@cite@cb{%
    \ifx\@prev@cite\@empty
    \else
        \begingroup
            \def\cite@print##1##2{%
                \cite@cb##1%
            }%
            \prev@cite
        \endgroup
    \fi
}
%    \end{macrocode}
%    \end{macro}
%
%    \begin{macro}{\cite@print}
%    \begin{macrocode}
\def\cite@print#1#2{%
    \begingroup
        \let\CitePrintUndefined\CPU@normal
        \cite@cjs#1{#2}%
    \endgroup
}
%    \end{macrocode}
%    \end{macro}
%
%    \begin{macro}{\cite@dash}
%    Ok, I lied.  There was more than one further complication.
%    Suppose that when we hit the end of the list, we're in state~2.
%    We need to know whether to output a dash or a comma.  (For
%    example, both the sequences $[2, 3]$ and $[1, 2, 3]$ will end in
%    state~2 with $\mathit{prevcite} = 3$, but in the former case we
%    want a comma before the~3 and in the latter case we want a dash.)
%    So, rather than printing the dash explicitly, we use
%    \cs{cite@dash} to keep track of whether a dash is needed.
%    \begin{macrocode}
\let\cite@dash\@empty
%    \end{macrocode}
%    \end{macro}
%
%    \begin{macro}{\print@one@dash}
%    \begin{macrocode}
\def\print@one@dash{%
    \textendash \nopunct
    \let\cite@dash\@empty
}
%    \end{macrocode}
%    \end{macro}
%
%    State 0, 1 and~2 each correspond to a different binding for
%    \cs{cite@cj}. Here they are.  The role of $\mathit{prevnum}$ is
%    played by \cs{@tempcnta}, with \cs{@tempcntb} assisting as
%    $\mathit{number}$ at times.
%
%    \begin{macro}{\cite@compress}
%    State~0:
%    \begin{macrocode}
\def\cite@compress#1#2{%
    \cite@print#1{#2}%
    \@ifempty{#2}{%
        \safe@set\@tempcnta#1%
        \let\cite@cj\cite@compress@a
    }{}%
}
%    \end{macrocode}
%    \end{macro}
%
%    \begin{macro}{\cite@compress@a}
%    State~1:
%    \begin{macrocode}
\def\cite@compress@a#1#2{%
    \@ifempty{#2}{%
        \advance\@tempcnta\@ne
        \safe@set\@tempcntb#1%
        \ifnum\@tempcnta=\@tempcntb
            \def\prev@cite{\cite@print#1{}}%
            \let\cite@cj\cite@compress@b
        \else
            \cite@print#1{}%
            \@tempcnta\@tempcntb
        \fi
    }{%
        \cite@print#1{#2}%
        \let\cite@cj\cite@compress
    }%
}
%    \end{macrocode}
%    \end{macro}
%
%    \begin{macro}{\cite@compress@b}
%    State~2:
%    \begin{macrocode}
\def\cite@compress@b#1#2{%
    \@ifempty{#2}{%
        \advance\@tempcnta\@ne
        \safe@set\@tempcntb#1%
        \ifnum\@tempcnta=\@tempcntb
            \let\cite@dash\print@one@dash
            \prev@cite@cb
            \def\prev@cite{\cite@print#1{}}%
        \else
            \cite@dash
            \prev@cite
            \let\prev@cite\@empty
            \cite@print#1{}%
            \@tempcnta\@tempcntb
            \let\cite@cj\cite@compress@a
        \fi
    }{%
        \cite@dash
        \prev@cite
        \let\prev@cite\@empty
        \cite@print#1{#2}%
        \let\cite@cj\cite@compress
    }%
}
%    \end{macrocode}
%    \end{macro}
%
% \subsubsection{Munging the \fn{.aux} file}
%
%    \begin{macro}{\bibcite}
%    When processing the \fn{.aux} file at begin-document, this is what
%    \cs{bibcite} will do:
%    \begin{macrocode}
\def\bibcite#1{\@xp\bibcite@a\csname b@#1\endcsname}
%    \end{macrocode}
%    \end{macro}
%
%    \begin{macro}{\bibcite@a}
%    \MacroArgs
%    \macroarg{1}{\cs{b@}citekey}
%    \macroarg{2}{\embrace{\emph{label}}\embrace{} or \embrace{\emph{author}}\embrace{\emph{year}}}
%    \begin{macrocode}
\def\bibcite@a#1#2{%
%    \end{macrocode}
%    Most of the time arg 1 will already be defined, by an earlier
%    \cs{citedest} command in the \fn{.aux} file. Then we just need to change
%    the number.
%    \begin{macrocode}
    \ifx\relax#1%
        \gdef#1{\citesel 00#2{}}%
    \else
        \begingroup
            \@xp\bibcite@b\@xp#1#1{#2}%
        \endgroup
    \fi
}
%    \end{macrocode}
%    \end{macro}
%
%    \begin{macro}{\bibcite@b}
%
%    \MacroArgs
%    \macroarg{1}{\cs{b@}citekey}
%    \macroarg{2}{\cs{citesel}}
%    \macroarg{3}{cited?}
%    \macroarg{4}{used?}
%    \macroarg{5}{label}
%    \macroarg{6}{year}
%    \macroarg{7}{backrefs}
%    \macroarg{8}{\embrace{\emph{newlabel}}\embrace{\emph{newyear}}}
%    \begin{macrocode}
\def\bibcite@b#1#2#3#4#5#6#7#8{\gdef#1{\citesel#3#4#8{#7}}}
%    \end{macrocode}
%    \end{macro}
%
%    \begin{macro}{\citedest}
%    The \cs{citedest} command goes into the \fn{.aux} file to provide
%    back-reference support.
%    \begin{macrocode}
\newcommand{\citedest}[1]{\@xp\cite@dest\csname b@#1\endcsname}
%    \end{macrocode}
%    \end{macro}
%
%    \begin{macro}{\cite@dest}
%    \begin{macrocode}
\def\cite@dest#1{%
    \ifx\relax#1%
        \gdef#1{\citesel 00{}{}{}}%
    \fi
    \@xp\cite@dest@b\@xp#1#1%
}
%    \end{macrocode}
%    \end{macro}
%
%    \begin{macro}{\cite@dest@b}
%    \MacroArgs
%    \macroarg{1}{\cs{b@}citekey}
%    \macroarg{2}{\cs{citesel}}
%    \macroarg{3}{cited?}
%    \macroarg{4}{used?}
%    \macroarg{5}{label}
%    \macroarg{6}{year}
%    \macroarg{7}{backrefs}
%    \macroarg{8}{\embrace{\emph{more backrefs}}}
%    \begin{macrocode}
\def\cite@dest@b#1#2#3#4#5#6#7#8{%
    \@ifempty{#7}{%
        \def#1{\citesel #3#4{#5}{#6}{{#8}}}%
    }{%
        \gdef#1{\citesel #3#4{#5}{#6}{#7,{#8}}}%
    }%
}
%    \end{macrocode}
%    \end{macro}
%
% \subsubsection{Back references}
%
%    \begin{macro}{\ifBR@verbose}
%    \begin{macrocode}
\@ifundefined{ifBR@verbose}{\let\ifBR@verbose\iffalse \let\fi\fi}{}%
%    \end{macrocode}
%    \end{macro}
%
%    \begin{macro}{\BackCite}
%    \begin{macrocode}
\let\BackCite\@gobble
%    \end{macrocode}
%    \end{macro}
%
%    \begin{macro}{\back@cite}
%    \begin{macrocode}
\def\back@cite#1{%
    \ifBR@verbose
        \PackageInfo{backref}{back cite \string `\extr@cite#1'}%
    \fi
    \Hy@backout{#1}%
}
%    \end{macrocode}
%    \end{macro}
%
%    \begin{macro}{\print@backrefs}
%    In an AMS-style bibliography, the backref info might follow the
%    final period of the reference, or it might follow some
%    \emph{Mathematical Reviews} info, without a period.
%    \begin{macrocode}
\def\print@backrefs#1{%
    \space\SentenceSpace$\uparrow$\csname br@#1\endcsname
}
%    \end{macrocode}
%    \end{macro}
%
%    \begin{macro}{\PrintBackRefs}
%    \begin{macrocode}
\let\PrintBackRefs\@gobble
%    \end{macrocode}
%    \end{macro}
%
% \subsubsection{\pkg{hyperref} and \pkg{showkeys} support}
%
%    \begin{macro}{\ar@hyperlink}
%    \begin{macrocode}
\def\ar@hyperlink#1{\hyper@@link [cite]{}{cite.\extr@cite#1}{#1}}
%    \end{macrocode}
%    \end{macro}
%
%    \begin{macro}{\ar@SK@cite}
%    \begin{macrocode}
\def\ar@SK@cite#1{\@bsphack\@xp\SK@\@xp\SK@@ref\@xp{\extr@cite#1}\@esphack}
%    \end{macrocode}
%    \end{macro}
%
%    Turn off \pkg{hyperref} and \pkg{showkeys} support if those
%    packages don't appear to be loaded.
%    \begin{macrocode}
\AtBeginDocument{%
    \@ifundefined{hyper@@link}{%
        \let\ar@hyperlink\@firstofone
        \let\hyper@anchorstart\@gobble
        \let\hyper@anchorend\relax
    }{}%
    \@ifundefined{SK@@label}{%
        \let\ar@SK@cite\@gobble
        \let\SK@@label\@gobble
        \let\SK@\@gobbletwo
    }{}%
}
%    \end{macrocode}
%
% \subsection{Lexical structure of names}
%
%    Before we can begin parsing names, we need to give some thought
%    to the lexical structure of names.  For the remainder of this
%    document, when we refer to a ``name'' and especially when we
%    speak of a name as a macro argument, we assume that the only
%    tokens contained in the name are
%    \begin{itemize}
%
%    \item
%    letters and punctuation (i.e., characters with catcode 11 or~12),
%
%    \item
%    ties (the token |~|$_{13}$),
%
%    \item
%    accent commands, such as |\"| or |\k|,
%
%    \item
%    text symbol macros, such as |\i|, |\ae| or |\cprime|,
%
%    \item
%    grouping characters (braces).
%
%    \end{itemize}
%
%    In addition to their normal function of delimiting macro
%    arguments, braces inside names have the following special
%    functions:
%    \begin{enumerate}
%
%    \item They are used to indicate that multiple characters should
%    be considered a single ``compound'' character when extracting
%    initials.  For example, |Yuri| becomes |Y.|, but |{Yu}ri| becomes
%    |Yu.|
%
%    \label{initial-braces}
%    An important aspect of this use of braces is that it only applies
%    to the first characters of a given name.  As we'll see below,
%    this has important implications for our parsing code, which must
%    preserve braces at the beginning of given names, but can be more
%    cavalier with braces in other positions.
%
%    \item
%    Spaces and commas are ordinarily interpreted as name separators,
%    rather than name components.  Similarly, periods and hyphens
%    usually have a special interpretation.  All these characters can
%    be stripped of their special meanings by putting them within
%    braces.
%
%    \end{enumerate}
%
%    In practice, it might be possible to insert other tokens (such as
%    macros) into names as long as they either (a)~are non-expandable
%    or (b)~expand into a series of tokens of the above enumerated
%    types.  However, in such cases it will probably be safer to
%    declare the macro in question as either a text accent or a text
%    symbol.
%
% \subsubsection{Text accents}
%
%    Syntactically, a text accent is a macro that takes a single,
%    undelimited argument, i.e, it has a ``prototype'' of
%    |macro:#1->|.  Semantically, the implication is that it takes a
%    letter (the \emph{base}) as an argument and produces a glyph that
%    for certain purposes can be considered equivalent to the
%    base (see the discussion of stem comparison on
%    page~\pageref{stem-comparison}).\footnote{Note that this is meant
%    to be a pragmatic definition for the purposes of this package.
%    No claim is made to greater generality.}
%
%    \begin{macro}{\amsrefs@textaccents}
%    This will contain a list of accent commands in standard \LaTeX\
%    format (i.e., separated by the token \cs{do}).  For example,
%    after registering the \cn{"} and \cn{'} accents, it will contain
%    \begin{verbatim}
%        \do \"\do \'
%\end{verbatim}
%    \begin{macrocode}
\let\amsrefs@textaccents\@empty
%    \end{macrocode}
%    \end{macro}
%
%    \begin{macro}{\DeclareNameAccent}
%    \MacroArgs
%    \macroarg{1}{accent}
%    \begin{macrocode}
\def\DeclareNameAccent{%
    \@lappend\amsrefs@textaccents
}
%    \end{macrocode}
%    \end{macro}
%
%    Here are all the standard \LaTeX\ accents, as well as a few
%    nonstandard accents from the \pkg{mathscinet} package.
%    \begin{macrocode}
\DeclareNameAccent\"
\DeclareNameAccent\'
\DeclareNameAccent\.
\DeclareNameAccent\=
\DeclareNameAccent\^
\DeclareNameAccent\`
\DeclareNameAccent\~%
\DeclareNameAccent\b
\DeclareNameAccent\c
\DeclareNameAccent\d
\DeclareNameAccent\H
\DeclareNameAccent\k
\DeclareNameAccent\r
\DeclareNameAccent\t
\DeclareNameAccent\u
\DeclareNameAccent\v
%    \end{macrocode}
%    From \pkg{mathscinet}:
%    \begin{macrocode}
\DeclareNameAccent\utilde
\DeclareNameAccent\uarc
\DeclareNameAccent\dudot
\DeclareNameAccent\lfhook
\DeclareNameAccent\udot
\DeclareNameAccent\polhk
\DeclareNameAccent\soft
%    \end{macrocode}
%    \cn{etalchar} and \cn{etaltext} are sort of accent-like if you
%    look at them in the right light.
%    \begin{macrocode}
\DeclareNameAccent\etalchar
%\DeclareNameAccent\etaltext
%    \end{macrocode}
%
% \subsubsection{Text symbols}
%
%    Syntactically, a text symbol is a macro with a empty parameter
%    text, i.e., a prototype of |macro:->|.  Semantically, it's a
%    letter-like glyph that should not be considered equivalent
%    to any other glyph or group of glyphs.  In addition, it may exist
%    in both upper- and lowercase variants, unlike text accents, where
%    we consider the case to be an attribute of the base letter, not
%    of the accent.\footnote{As with text accents, this is not
%    intended as a fully general definition.}
%
%    \begin{macro}{\amsrefs@textsymbols}
%    This is analogous to \cs{amsrefs@textaccents} but a little more
%    complicated due to the need to store lowercase equivalents. It
%    consists of a list of double entries of the form
%    \begin{verbatim}
%        \do \symbol \do \lcsymbol
%\end{verbatim}
%    which means that \cs{symbol} is a text symbol whose corresponding
%    lowercase version is \cs{lcsymbol}.  (Note that nothing is
%    implied about whether \cs{symbol} is to be considered as
%    uppercase or lowercase.)  For example, in
%    \begin{verbatim}
%        \do \ae \do \ae \do \OE \do \oe
%\end{verbatim}
%    the first four tokens indicate that \cn{ae} is a text symbol with
%    lowercase equivalent \cn{ae}, while the last four tokens indicate
%    that \cn{OE} is a text symbol with lowercase equivalent \cn{oe}.
%    This scheme is somewhat redundant, but pleasingly simple.
%
%    This also duplicates some of the information in \cs{@uclclist},
%    but it seems safer to do this than to modify \cs{@uclclist}.
%    \begin{macrocode}
\let\amsrefs@textsymbols\@empty
%    \end{macrocode}
%    \end{macro}
%
%    \begin{macro}{\DeclareNameSymbol}
%    \MacroArgs
%    \macroarg{1}{symbol}
%    \macroarg{2}{lowercase}
%    \begin{macrocode}
\def\DeclareNameSymbol#1#2{%
    \@lappend\amsrefs@textsymbols#1%
    \@lappend\amsrefs@textsymbols#2%
    \ifx#1#2\else
        \@lappend\amsrefs@textsymbols#2%
        \@lappend\amsrefs@textsymbols#2%
    \fi
}
%    \end{macrocode}
%    \end{macro}
%
%    Here are the standard \LaTeX\ and \pkg{mathscinet} text
%    symbols.
%
%    \label{anomalies}
%    Note that \cn{i} and \cn{j} are anomalous in being syntactically
%    like text symbols, but semantically more like text accents.
%    \begin{macrocode}
\DeclareNameSymbol\i\i
\DeclareNameSymbol\j\j
\DeclareNameSymbol\AE\ae
\DeclareNameSymbol\OE\oe
\DeclareNameSymbol\O\o
\DeclareNameSymbol\DH\dh
\DeclareNameSymbol\DJ\dj
\DeclareNameSymbol\L\l
\DeclareNameSymbol\NG\ng
\DeclareNameSymbol\SS\ss
\DeclareNameSymbol\TH\th
%    \end{macrocode}
%    From \pkg{mathscinet}:
%    \begin{macrocode}
\DeclareNameSymbol\Dbar\dbar
\DeclareNameSymbol\lasp\lasp
\DeclareNameSymbol\rasp\rasp
\DeclareNameSymbol\cprime\cprime
\DeclareNameSymbol\cdprime\cdprime
\DeclareNameSymbol\bud\bud
\DeclareNameSymbol\cydot\cydot
%    \end{macrocode}
%    |~| can be considered a text symbol in much the same way that
%    \cn{etalchar} can be considered an accent.
%    \begin{macrocode}
\DeclareNameSymbol~~%
%    \end{macrocode}
%
% \subsubsection{\cs{edef}-like macros for names}
%    The following macros all behave sort of like \cs{edef}, in the
%    sense that
%    \begin{verbatim}
%    \X@edef\foo{name}
%\end{verbatim}
%    defines \cs{foo} to be the result of expanding |name| and
%    applying a certain transformation to it.
%
%    \begin{macro}{\normalize@edef}
%    This converts accents in the name to a normalized form where the
%    accent and its argument are surrounded by braces.  E.g., after
%    \begin{verbatim}
%    \normalize@edef\cs{P\'olya}
%\end{verbatim}
%    \cs{cs} will contain |P{\'o}lya|. (This might result in a
%    redundant layer of braces if the original text contained, say,
%    ``\verb+P{\'o}lya+'', but that's ok.)  This lets us extract the
%    first $n$ characters from a name by using \TeX's macro
%    argument-gobbling mechanism without worrying that an accent
%    will be separated from its base letter.
%    As a bonus, it also replaces ties (|~|) by spaces.
%    \begin{macrocode}
\def\normalize@edef#1#2{%
    \begingroup
        \@apply\auto@protect\amsrefs@textsymbols
        \@apply\wrap@accent\amsrefs@textaccents
%    \end{macrocode}
%    Redefine \cs{@tabacckludge} in case someone wants to use this
%    with the \pkg{inputenc} package.
%    \begin{macrocode}
        \let\@tabacckludge\use@accent
        \let~\space
        \edef\@tempa{\def\@nx#1{#2}}%
    \@xp\endgroup
    \@tempa
}
%    \end{macrocode}
%    \end{macro}
%
%    \begin{macro}{\use@accent}
%    This is identical to \cs{@nameuse} except for the addition of the
%    \cs{string}, which, as per \fn{ltoutenc.dtx}, guards against the
%    eventuality that something like |'| might be active at the point
%    of use.  We don't expect to find a \cs{bib} in the middle of a
%    \env{tabbing} environment (do we?)\ so we
%    \begin{macrocode}
\def\use@accent#1{\csname\string#1\endcsname}
%    \end{macrocode}
%    \end{macro}
%
%    \begin{macro}{\wrap@accent}
%    Here's a wrapper macro that causes an accent to become
%    auto-wrapping.  E.g., after \verb+\wrap@accent\'+, \verb+\'o+
%    will expand to \verb+{\'o}+.
%    \begin{macrocode}
\def\wrap@accent#1{%
    \def#1##1{{\@nx#1##1}}%
}
%    \end{macrocode}
%    \end{macro}
%
%    \begin{macro}{\lc@edef}
%    This converts all the characters in a name to all lowercase,
%    using the mapping defined by \cs{amsrefs@textsymbols}.  So, after
%    \begin{verbatim}
%    \lc@edef\cs{P\'olya}
%\end{verbatim}
%    \cs{cs} will contain |p\'olya|.  Note that accents are not
%    wrapped and ties are passed through unmolested.
%    \begin{macrocode}
\def\lc@edef#1#2{%
    \begingroup
        \let\@tabacckludge\use@accent %%??
        \@apply\auto@protect\amsrefs@textaccents
        \@apply\lc@do\amsrefs@textsymbols
        \edef\@tempa{\lowercase{\def\@nx#1{#2}}}%
    \@xp\endgroup
    \@tempa
}
%    \end{macrocode}
%    \end{macro}
%
%    \begin{macro}{\lc@do}
%    This is a slighly more complicated wrapper macro than previous
%    ones.
%    The first argument is a text symbol; the second argument is the
%    lowercase variant of the symbol.  If they're the same (i.e., the
%    first argument is a lowercase text symbol), we \cs{auto@protect}
%    it.  Otherwise we define the first symbol to expand to the
%    second.
%    \begin{macrocode}
\def\lc@do#1\do#2{%
    \ifx#1#2%
        \auto@protect#1%
    \else
        \def#1{#2}%
    \fi
}
%    \end{macrocode}
%    \end{macro}
%
%    \begin{macro}{\purge@edef}
%    Removes accents and braces from a name and converts ties to
%    spaces, leaving only letters, punctuation and text symbols.  For
%    example,
%    \begin{verbatim}
%    \lc@edef\cs{P{\'o}lya}
%\end{verbatim}
%    will put |Polya| in~\cs{cs}.
%    \begin{macrocode}
\def\purge@edef#1#2{%
    \begingroup
        \@apply\auto@protect\amsrefs@textsymbols
        \let~\space
        \@apply\purge@accent\amsrefs@textaccents
        \let\@tabacckludge\@gobble
%    \end{macrocode}
%    As mentioned above (page~\pageref{anomalies}), \cn{i} and \cn{j}
%    are semantically like text accents; hence, they require special
%    treatment here.
%    \begin{macrocode}
        \def\i{i}%
        \def\j{j}%
        \edef\@tempa{#2}%
        \toks@\@emptytoks
        \@xp\purge@edef@ \@tempa \@nil
        \edef\@tempa{\def\@nx#1{\the\toks@}}%
    \@xp\endgroup
    \@tempa
}
%    \end{macrocode}
%    \end{macro}
%
%    \begin{macro}{\purge@edef@}
%    Peek ahead so \cs{purge@edef@a} will know whether its argument
%    was originally surrounded by braces.
%    \begin{macrocode}
\def\purge@edef@{%
    \futurelet\@let@token
    \purge@edef@a
}
%    \end{macrocode}
%    \end{macro}
%
%    \begin{macro}{\purge@edef@a}
%    Process a single ``chunk'' (i.e., one macro-argument's worth) of
%    the name.
%    \begin{macrocode}
\def\purge@edef@a#1{%
%    \end{macrocode}
%    If we've run into the \cs{@nil} terminator, we're done.
%    \begin{macrocode}
    \ifx\@let@token\@nil
        \let\@tempa\@empty
    \else
%    \end{macrocode}
%    Otherwise, if the argument was originally surrounded by braces,
%    process it recursively before processing the remainder of the
%    token stream.
%    \begin{macrocode}
        \ifx\@let@token\bgroup
            \def\@tempa{%
                \purge@edef@ #1\@nil
                \purge@edef@
            }%
        \else
%    \end{macrocode}
%    If the argument is a single unbracketed token, just copy it into
%    the output.
%    \begin{macrocode}
            \add@toks@{#1}%
            \let\@tempa\purge@edef@
        \fi
    \fi
    \@tempa
}
%    \end{macrocode}
%    \end{macro}
%
%    \begin{macro}{\purge@accent}
%    This is similar to \cs{wrap@accent} but it removes the accent
%    command (and possibly a layer of braces surrounding the accent's
%    argument).
%    \begin{macrocode}
\def\purge@accent#1{%
    \def#1##1{##1}%
}
%    \end{macrocode}
%    \end{macro}
%
% \subsection{Name parsing}
%
%    Parsing names is somewhat complicated because parts of the name
%    can (in principle) be empty (G=given, S=surname, J=jr)):
%    \begin{quote}
%    \nametest{Doe, John, Jr.}
%    \nametest{Doe, John}
%    \nametest{Doe, , Jr.}
%    \nametest{Doe}
%    \nametest{, John, Jr.}
%    \nametest{, John}
%    \nametest{, , Jr.}
%    \nametest{}
%    \end{quote}
%    Not all of these forms are legal, of course, but that's no excuse
%    for not parsing them correctly.
%
%    We also want to be somewhat lenient about the placement of
%    spaces:
%    \begin{quote}
%    \nametest{ Doe,John,Jr.}
%    \end{quote}
%
%    However, because one must have some standards, we assume there
%    are no spaces in the following positions in the input:
%    \begin{enumerate}
%    \item before periods,
%
%    \item before commas,
%
%    \item at the end of the name,
%
%    \item before or after hyphens.
%    \end{enumerate}
%
%    Thus, we make no attempt to compensate for the misplaced spaces
%    in examples like these:
%    \begin{quote}
%    \nametest{Doe , J ., Jr. }
%    \nametest{Doe, J. - M.}
%    \end{quote}
%
%    Also, unless we are generating initials, we don't try to
%    normalize spaces \emph{after} periods:
%    \begin{quote}
%    \nametest{Doe, J.M.}\\ (not G=\embrace{J. M.})
%    \end{quote}
%
%    Finally, since we allow authors to group together characters that
%    should be treated as a single unit, we need to be careful to
%    preserve the author's markup in cases like these:
%
%    \begin{quote}
%    \nametest{Doe, {Yu}ri}
%    \nametest{Doe, {Yu}}
%    \end{quote}
%
%    This is harder than it seems.  For example, consider a naive
%    implementation that uses delimited arguments to pull the name
%    apart:
%    \begin{verbatim}
%    \def\parsename#1,#2\@nil{%
%        \def\bib'surname{#1}%
%        \def\bib'given{#2}%
%    }
%
%    \parsename Doe, {Yu}ri\@nil
%\end{verbatim}
%    Unfortunately, this results in the space after the comma
%    becoming part of \cs{bib'given}: ``\verb+ {Yu}ri+''.
%
%    Our next thought would be to modify the definition slightly to
%    trick \TeX\ into gobbling the space:
%    \begin{verbatim}
%    \def\parsename#1,#2#3\@nil{%
%        \def\bib'surname{#1}%
%        \def\bib'given{#2#3}%
%    }
%\end{verbatim}
%    Now the space is gone, but---surprise!---so are the braces:
%    ``\verb+Yuri+''.  In addition, this approach makes it difficult
%    to handle empty name parts correctly.
%
%    To sidestep these problems, instead of blindly gobbling macro
%    arguments, we use \cs{futurelet} to look ahead at certain
%    strategic moments so we can take the appropriate action (see
%    \cs{get@namepart@d}--|f|).  We only really care about preserving
%    braces at the start of names (page~\pageref{initial-braces}),
%    which simplifies things somewhat.\label{brace-preservation}
%
%    \begin{macro}{\name@split}
%    \cs{name@split} parses a name into its three parts and stores
%    them in \cs{bib'surname}, \cs{bib'given} and \cs{bib'jr}.  If the
%    \opt{initials} option is in force, it also extracts the initials
%    from the given name and stores them in \cs{bib'initials}.
%
%    It expects the name to be parsed to be terminated by \cs{@nil}
%    and to contain at least three commas.  Thus the usual way to
%    invoke it is
%    \begin{quote}
%    \cs{name@split} $\langle\mathit{name}\rangle$\verb+,,,\@nil+
%    \end{quote}
%
%    \cs{name@split} just uses \cs{get@namepart} to peal off the
%    surname and then passes control to \cs{name@split@given}.
%    (Note the spiffy continuation-passing programming style.)
%    \begin{macrocode}
\def\name@split{%
    \get@namepart\bib'surname\name@split@given
}
%    \end{macrocode}
%    \end{macro}
%
%    \begin{macro}{\name@split@given}
%    Pretty much the same, \em{mutatis mutandis}\ldots
%    \begin{macrocode}
\def\name@split@given{%
    \get@namepart\bib'given\name@split@jr
}
%    \end{macrocode}
%    \end{macro}
%
%    \begin{macro}{\name@split@jr}
%    And again\ldots
%    \begin{macrocode}
\def\name@split@jr{%
    \get@namepart\bib'jr\name@split@finish
}
%    \end{macrocode}
%    \end{macro}
%
%    \begin{macro}{\name@split@finish}
%    We have all three parts now.  Do some consistency checking,
%    extract the initials from the given name, and then call
%    \cs{@nilgobble} to remove anything (such as extra commas) left on
%    the stack.
%    \begin{macrocode}
\def\name@split@finish{%
    \ifx\bib'surname\@empty \EmptyNameWarning \fi
%    \end{macrocode}
%    Theoretically, we could try to check for uninverted names here,
%    but only at the risk of producing spurious warnings when the name
%    really does only have one part (\verb+author={Arvind}+).
%
%    A possible solution: Now that we have the
%    \attr{inverted} attribute, we could issue a warning if the given
%    name is empty and the family name contains a space.  I'm sure
%    someone could find valid input that would still generate a
%    spurious warning, but this would take care of the most common
%    cases.  This bears more thinking about.
%    \begin{macrocode}
%%      \ifx\@empty\bib'given
%%          \NameCheck \bib'surname ??\@nil
%%      \else
        \extract@initials\bib'given
%%      \fi
    \@nilgobble
}
%    \end{macrocode}
%    \end{macro}
%
%    \begin{macro}{\get@namepart}
%    Now for the fun part.  \cs{get@namepart} takes two arguments.
%    The first (the destination) should be a control sequence; the
%    second (the continuation) will normally also be a control
%    sequence, though technically we only require that it be a single
%    token.  \cs{get@namepart} scans everything up to the next level-0
%    comma, places it in the destination, and then calls the
%    continuation.
%    \begin{macrocode}
\def\get@namepart#1#2{%
%    \end{macrocode}
%    Save the destination in \cs{toks@} and the continuation in
%    \cs{@temptokena}.  It's unfortunate that this trashes the
%    previous contents of those token lists (as well as the contents
%    of \cs{@tempa} later on), but preliminary attempts to rewrite the
%    code to leave the calling environment unchanged were not
%    encouraging.
%    \begin{macrocode}
    \toks@{#1}%
    \@temptokena{#2}%
    \get@namepart@a
}
%    \end{macrocode}
%    \end{macro}
%
%    \begin{macro}{\get@namepart@a}
%    Now peek ahead at the next token in the stream and call
%    \cs{get@namepart@b} to examine it.
%    \begin{macrocode}
\def\get@namepart@a{%
    \futurelet\@let@token
    \get@namepart@b
}
%    \end{macrocode}
%    \end{macro}
%
%    \begin{macro}{\get@namepart@b}
%    If the next token is a space token, we want to delete it.
%    Otherwise we're ready to read the name.
%    \begin{macrocode}
\def\get@namepart@b{%
    \ifx\@let@token\@sptoken
        \@xp\get@namepart@c
    \else
        \@xp\get@namepart@d
    \fi
}
%    \end{macrocode}
%    \end{macro}
%
%    \begin{macro}{\get@namepart@c}
%    The next token is a space; we delete it and restart
%    \cs{get@namepart@a}, in case there are multiple spaces.
%    \begin{macrocode}
\def\get@namepart@c{%
    \after@deleting@token\get@namepart@a
}
%    \end{macrocode}
%    \end{macro}
%
%    \begin{macro}{\get@namepart@d}
%    We're at the beginning of the name part.  However, there are
%    still two special cases we have to watch out for.  First, the
%    next token might be a comma, meaning that this name part is
%    empty.  Second, the next token might be an open brace (\verb+{+),
%    which we have to be sure to copy into the destination.  So, we
%    peek ahead again before proceeding.
%    \begin{macrocode}
\def\get@namepart@d{%
    \futurelet\@let@token
    \get@namepart@e
}
%    \end{macrocode}
%    \end{macro}
%
%    \begin{macro}{\get@namepart@e}
%    If the next token is a comma, it means the name part is empty;
%    so, we set the destination to an empty list and then arrange to
%    execute the continuation after deleting the comma.  Otherwise we
%    call \cs{get@namepart@f} to read a non-empty name, leaving
%    \cs{@let@token} undisturbed so that \cs{get@namepart@f} knows
%    what's coming up.
%    \begin{macrocode}
\def\get@namepart@e{%
    \ifx\@let@token,%
        \@xp\let\the\toks@\@empty
        \edef\@tempa{%
            \@nx\after@deleting@token\the\@temptokena
        }%
        \@xp\@tempa
    \else
        \@xp\get@namepart@f
    \fi
}
%    \end{macrocode}
%    \end{macro}
%
%    \begin{macro}{\get@namepart@f}
%    We know whether or not the name begins with a brace, but we don't
%    know if the corresponding group contains the entire name or only
%    part of it.  By reading the name as two arguments, we can handle
%    all cases correctly.\footnote{More or less.  If the second
%    argument is brace-delimited, the braces will be lost.  But as
%    mentioned above (page~\pageref{brace-preservation}), we don't
%    really care.}
%
%    Note that the arguments are not expanded.
%    \begin{macrocode}
\def\get@namepart@f#1#2,{%
    \ifx\@let@token\bgroup
        \@xp\def\the\toks@{{#1}#2}%
    \else
        \@xp\def\the\toks@{#1#2}%
    \fi
    \the\@temptokena
}
%    \end{macrocode}
%    \end{macro}
%
%    \begin{macro}{\EmptyNameWarning}
%    Or translator or contributor or\ldots
%    \begin{macrocode}
\def\EmptyNameWarning{\amsrefs@warning{Empty contributor name}}
%    \end{macrocode}
%    \end{macro}
%
% \subsection{Extracting initials}
%
%    Extracting initials from the author's given name is tricky
%    because of the numerous special cases that need to be
%    handled. Consider the following examples, some of which are
%    admittedly contrived:
%    \begin{quote}
%    \initialtest{Arvind}
%    \initialtest{Bing, R H}
%    \initialtest{Harish, \'Etienne}
%    \initialtest{Harish, \'E.}
%    \initialtest{Harish, \'{E}.}
%    \initialtest{Harish, {\'E}.}
%    \initialtest{Harish, \'E}
%    \initialtest{Harish, \'Etienne-P\^{\i}erre}
%    \initialtest{Jones, David}
%    \initialtest{Jones, David-Michael}
%    \initialtest{Katzenbach, Nicholas {deB}elleville}
%    \initialtest{Katzenbach, Nicholas deB.}
%    \initialtest{Matiyasevich, {Yu}ri}
%    \initialtest{Matiyasevich, {Yu}}
%    \initialtest{Matiyasevich, Yu.}
%    \end{quote}
%    When processing initials, we loosen our strictures on spaces
%    inside the given name by not requiring spaces after periods and
%    tolerating them around hyphens and after the name:
%    \begin{quote}
%    \initialtest{Jones, D.M.}
%    \initialtest{Jones, David - Michael}
%    \initialtest{Jones, David , Jr.}
%    \end{quote}
%    (Strictly speaking, only the support for the first of these
%    examples was a deliberate design decision; the other two are
%    side-effects of the implementation.  In any case, toleration of
%    these quirks is in no way an endorsement of them, especially
%    since they may make it more difficult for third-party software to
%    correctly process bibliography entries.)
%
%^^A    We also make one additional assumption: a hyphen does not occur
%^^A    as the last character in the name.
%
% \subsubsection{The algorithm}
%
%    As a running example, consider the following contrived input:
%    \begin{quote}
%    {\tt \verb+\'E.-P\^{\i}erre J.K. M+}
%    \end{quote}
%    which we want to turn into {\makeatletter``{\'E}.~\ini@hyphen
%    P.~J.~K.~M''}.
%
%    We precede by stages.
%    \begin{enumerate}
%
%    \item
%    Normalize the name by surrounding accents and their arguments by
%    braces:
%    \begin{quote}
%    \verb+{\'E}.-P{\^\i }erre J.K. M+
%    \end{quote}
%    We also replace \verb+~+s by spaces at this stage.
%
%    \item
%    Replace each hyphen (\verb+-+) by ``\verb*+ \ini@hyphen +'':
%    \begin{quote}
%    \verb+{\'E}. \ini@hyphen  P{\^\i }erre J.K. M+
%    \end{quote}
%
%    \item
%    Add a space after each period:
%    \begin{quote}
%    \verb+{\'E}.  \ini@hyphen  P{\^\i }erre J. K.  M+
%    \end{quote}
%
%    \item
%    Now we have the name as a list of space-separated components.
%    (In our example, the components are ``\verb+{\'E}.+'',
%    ``\verb+\ini@hyphen+'', ``\verb+P{\^\i }erre+'', ``\verb+J.+'',
%    ``\verb+K.+'', and ``\verb+M+''.)  We loop through the compoments
%    and replace each one by its ``initialized'' form.  There are
%    four cases:\label{initial_cases}
%    \begin{enumerate}
%
%    \item
%    The component ends in a period.  Copy it and add the token
%    \verb+~+.  (In our example, these are the
%    components ``\verb+{\'E}.+'', ``\verb+J.+'' and ``\verb+K.+''.)
%
%    \item
%    The component consists of a single (possibly compound) character
%    without a period.  Again, copy it and add
%    \verb+~+.  (In our example, this is the component
%    ``\verb+M+''.)
%
%    \item
%    The component is the token \verb+\ini@hyphen+.  Copy it.
%
%    \item
%    The component consts of two or more (possibly compound)
%    characters without a period (e.g., ``\verb+P{\^\i }erre+'').
%    Copy the first character and add the tokens \verb+.~+.
%    \end{enumerate}
%
%    \item
%    The token list generated above will end with an unwanted |~|.
%    Delete it.
%
%    \end{enumerate}
%
%    The end result is
%    \begin{verbatim}
%    {\'E}.~\ini@hyphen P.~J.~K.~M
%\end{verbatim}
%    which, when typeset, does indeed produce
%    {\makeatletter''{\'E}.~\ini@hyphen P.~J.~K.~M''}.\footnote{Tying
%    all the characters together is potentially undesirable when, as
%    in the example, there are a large number of pieces in the given
%    name.}
%
% \subsubsection{The implementation}
%
%    \begin{macro}{\extract@initials}
%    This is pretty straightforward.
%    \begin{macrocode}
\def\extract@initials#1{%
    \begingroup
        \auto@protect\ini@hyphen
        \auto@protect\nobreakspace
        \let~\relax
        \@apply\auto@protect\amsrefs@textsymbols
        \@apply\auto@protect\amsrefs@textaccents
        \normalize@edef\@tempa{#1}%
        \ifx\@tempa\@empty
        \else
%    \end{macrocode}
%    It would be nice if \cs{process@hyphens} and \cs{process@dots}
%    commuted, and they almost do.  However, suppose you have the
%    (admittedly contrived) name |Yu.-{Yu}|, which should be turned
%    into ``Yu.-Yu''.  If \cs{process@dots} is applied first, the
%    braces around the second ``Yu'' get removed, so the output is
%    ``Yu.-Y.''.  (Even worse would be |P.-\'E|, which would produce
%    ``P.-\'.'')
%    \begin{macrocode}
            \process@hyphens\@tempa
            \process@dots\@tempa
            \process@names\@tempa
            \@chomp\@tempa{~}%
        \fi
        \edef\@tempa{\def\@nx\bib'initials{\@tempa}}%
    \@xp\endgroup
    \@tempa
}
%    \end{macrocode}
%    \end{macro}
%
%    \begin{macro}{\ini@hyphen}
%    The \cs{unskip} removes the space at the end of a potential (and
%    probable) preceding |~|, but leaves the \cs{nobreak} penalty.
%    \begin{macrocode}
\def\ini@hyphen{\unskip-\nobreak}
%    \end{macrocode}
%    \end{macro}
%
%    \begin{macro}{\process@hyphens}
%    This follows the same general pattern as \cs{get@namepart}, but
%    with an extra layer of grouping to avoid unwanted side-effects.
%    Otherwise, it uses the same parsing techniques.
%
%    One difference is that there is no explicit continuation:
%    instead, we iterate by repeatedly calling
%    \cs{process@one@hyphen@d} until we run into the \cs{@nil}
%    marker.
%    \begin{macrocode}
\def\process@hyphens#1{%
    \begingroup
        \toks@\@emptytoks
        \@xp\process@one@hyphen #1-\@nil
        \edef\@tempa{\the\toks@}%
%    \end{macrocode}
%    Because of the |-| we have to stick in as a delimiter above,
%    \cs{process@one@hyphen} will always generate unwanted code at the
%    end of the name.  We now delete it.  (This also has the necessary
%    side-effect of expanding the \cs{space} macros into space
%    characters.)
%    \begin{macrocode}
        \@chomp\@tempa{ \ini@hyphen\space}%
        \edef\@tempa{\def\@nx#1{\@tempa}}%
    \@xp\endgroup
    \@tempa
}
%    \end{macrocode}
%    \end{macro}
%
%    \begin{macro}{\process@one@hyphen}
%    Cf.\ \cs{get@namepart@a}.
%    \begin{macrocode}
\def\process@one@hyphen{%
    \futurelet\@let@token
    \process@one@hyphen@a
}
%    \end{macrocode}
%    \end{macro}
%
%    \begin{macro}{\process@one@hyphen@a}
%    Cf.\ \cs{get@namepart@b} and \cs{extract@initial@a}.
%
%    The tests for \cs{@nil} and~|-| here are purely to supply better
%    error recovery.  Without them, a hyphen at the end of the given
%    name (.e.g, |author={Doe, John-}|) would produce a very
%    mysterious error message.  Since it's unlikely the hyphen really
%    belongs there, we delete it, but we also issue a warning to the
%    author. (It will still show up as part of the full given name,
%    though.)
%
%    We borrow \cs{fsa@n} from \pkg{rkeyval} to keep track of the
%    appropriate next action.
%    \begin{macrocode}
\def\process@one@hyphen@a{%
    \ifx\@let@token\@nil
        \let\fsa@n\@gobble
    \else
        \ifx\@let@token -%
            \TrailingHyphenWarning
            \let\fsa@n\process@one@hyphen@b
        \else
            \ifx\@let@token\@sptoken
                \let\fsa@n\process@one@hyphen@b
            \else
                \let\fsa@n\process@one@hyphen@c
            \fi
        \fi
    \fi
    \fsa@n
}
%    \end{macrocode}
%    \end{macro}
%
%    \begin{macro}{\process@one@hyphen@b}
%    Cf.\ \cs{get@namepart@c}.
%    \begin{macrocode}
\def\process@one@hyphen@b{%
    \after@deleting@token\process@one@hyphen
}
%    \end{macrocode}
%    \end{macro}
%
%    \begin{macro}{\process@one@hyphen@c}
%    Cf.\ \cs{get@namepart@f}.
%    \begin{macrocode}
\def\process@one@hyphen@c#1#2-{%
    \ifx\bgroup\@let@token
        \add@toks@{{#1}#2 \ini@hyphen\space}%
    \else
        \add@toks@{#1#2 \ini@hyphen\space}%
    \fi
    \futurelet\@let@token
    \process@one@hyphen@d
}
%    \end{macrocode}
%    \end{macro}
%
%    \begin{macro}{\process@one@hyphen@d}
%    Here we just check for \cs{@nil} and terminate if we detect it.
%    Otherwise, we start over.
%    \begin{macrocode}
\def\process@one@hyphen@d{%
    \ifx\@let@token\@nil
        \@xp\@gobble
    \else
        \@xp\process@one@hyphen
    \fi
}
%    \end{macrocode}
%    \end{macro}
%
%    \begin{macro}{\TrailingHyphenWarning}
%    Or translator or contributor or\ldots
%    \begin{macrocode}
\def\TrailingHyphenWarning{%
    \amsrefs@warning{Trailing hyphen deleted from name}%
}
%    \end{macrocode}
%    \end{macro}
%
%    \begin{macro}{\process@dots}
%    This is almost completely parallel to \cs{process@hyphens}.
%    \begin{macrocode}
\def\process@dots#1{%
    \begingroup
        \toks@\@emptytoks
        \@xp\process@one@dot #1.\@nil
        \edef\@tempa{\the\toks@}%
        \@chomp\@tempa{. }%
%    \end{macrocode}
%    Since it's legitimate for names to end in periods, we might
%    still have an unwanted space at the end of the name, so we delete
%    it too.
%    \begin{macrocode}
        \@chomp\@tempa{ }%
        \edef\@tempa{\def\@nx#1{\@tempa}}%
    \@xp\endgroup
    \@tempa
}
%    \end{macrocode}
%    \end{macro}
%
%    \begin{macro}{\process@one@dot}
%    \begin{macrocode}
\def\process@one@dot{%
    \futurelet\@let@token
    \process@one@dot@a
}
%    \end{macrocode}
%    \end{macro}
%
%    \begin{macro}{\process@one@dot@a}
%    This is a bit different from \cs{process@one@hyphen@a} since we
%    expect names sometimes to end in a period---or even two
%    periods---not least because of the |.| we add as a delimiter when
%    invoking \cs{process@one@dot}.
%    \begin{macrocode}
\def\process@one@dot@a{%
    \ifx\@let@token .%
        \def\fsa@n{\after@deleting@token\process@bare@dot}%
    \else
        \ifx\@let@token\@sptoken
            \let\fsa@n\process@one@dot@b
        \else
            \let\fsa@n\process@one@dot@c
        \fi
    \fi
    \fsa@n
}
%    \end{macrocode}
%    \end{macro}
%
%    \begin{macro}{\process@bare@dot}
%    \begin{macrocode}
\def\process@bare@dot{%
    \add@toks@{. }%
    \futurelet\@let@token
    \process@one@dot@d
}
%    \end{macrocode}
%    \end{macro}
%
%    \begin{macro}{\process@one@dot@b}
%    \begin{macrocode}
\def\process@one@dot@b{%
    \after@deleting@token\process@one@dot
}
%    \end{macrocode}
%    \end{macro}
%
%    \begin{macro}{\process@one@dot@c}
%    \begin{macrocode}
\def\process@one@dot@c#1#2.{%
    \ifx\bgroup\@let@token
        \add@toks@{{#1}#2. }%
    \else
        \add@toks@{#1#2. }%
    \fi
    \futurelet\@let@token
    \process@one@dot@d
}
%    \end{macrocode}
%    \end{macro}
%
%    \begin{macro}{\process@one@dot@d}
%    \begin{macrocode}
\def\process@one@dot@d{%
    \ifx\@let@token\@nil
        \@xp\@gobble
    \else
        \@xp\process@one@dot
    \fi
}
%    \end{macrocode}
%    \end{macro}
%
%    \begin{macro}{\process@names}
%    This is very similar to \cs{process@hyphens} and
%    \cs{process@dots}, but with a couple of twists, as noted below.
%    \begin{macrocode}
\def\process@names#1{%
    \begingroup
        \toks@\@emptytoks
        \@xp\extract@initial #1 \@nil
        \edef\@tempa{\def\@nx#1{\the\toks@}}%
    \@xp\endgroup
    \@tempa
}
%    \end{macrocode}
%    \end{macro}
%
%    \begin{macro}{\extract@initial}
%    Scan through the token stream replacing words by their initials
%    until we hit the terminating~|'|$_{11}$
%    \begin{macrocode}
\def\extract@initial{%
    \futurelet\@let@token
    \extract@initial@a
}
%    \end{macrocode}
%    \end{macro}
%
%    \begin{macro}{\extract@initial@a}
%    As with \cs{process@one@hyphen@a}, the test for |'|$_{11}$ here
%    is purely to provide better recovery, this time in case the given
%    name has a trailing space (.e.g, |author={Doe, John }|).
%    But since we're just deleting whitespace, we don't bother issuing
%    a warning.
%    \begin{macrocode}
\def\extract@initial@a{%
    \ifx\@let@token\@nil
        \let\fsa@n\@gobble
    \else
        \ifx\@let@token\@sptoken
            \let\fsa@n\extract@initial@b
        \else
            \let\fsa@n\extract@initial@c
        \fi
    \fi
    \fsa@n
}
%    \end{macrocode}
%    \end{macro}
%
%    \begin{macro}{\extract@initial@b}
%    \begin{macrocode}
\def\extract@initial@b{%
    \after@deleting@token\extract@initial
}
%    \end{macrocode}
%    \end{macro}
%
%    \begin{macro}{\extract@initial@c}
%    Here, instead of just copying the name, we extract its initials
%    and copy those.
%    \begin{macrocode}
\def\extract@initial@c#1#2 {%
    \ifx\@let@token\bgroup
%    \end{macrocode}
%    Note that we double-brace the first argument to avoid having to
%    test \cs{@let@token} again inside \cs{@extract@initial}.
%    \begin{macrocode}
        \@extract@initial {{#1}}#2\@nil
    \else
        \@extract@initial #1#2\@nil
    \fi
    \futurelet\@let@token
    \extract@initial@d
}
%    \end{macrocode}
%    \end{macro}
%
%    \begin{macro}{\extract@initial@d}
%    \begin{macrocode}
\def\extract@initial@d{%
    \ifx\@let@token\@nil
        \@xp\@gobble
    \else
        \@xp\extract@initial
    \fi
}
%    \end{macrocode}
%    \end{macro}
%
%    \begin{macro}{\@extract@initial}
%    This handles the four cases mentioned on
%    page~\pageref{initial_cases}.
%    \begin{macrocode}
\def\@extract@initial#1#2\@nil{%
    \ifx\ini@hyphen#1%
        \add@toks@{\ini@hyphen}%
    \else
        \in@{.\@nil}{#1#2\@nil}% Look for a period at the end of the name
        \ifin@
            \add@toks@{#1#2~}%
        \else
            \count@chars\@tempcnta{#1#2}%
            \ifnum\@tempcnta > \@ne
                \add@toks@{#1.~}%
            \else
                \add@toks@{#1~}%
            \fi
        \fi
    \fi
}
%    \end{macrocode}
%    \end{macro}
%
%    \begin{macro}{\count@chars}
%    This sets its first argument (which is assumed to be a count
%    register) to the number of characters in the second argument.
%    Compound characters are counted as a single character.
%    \begin{macrocode}
\def\count@chars#1#2{%
    \begingroup
        \@tempcnta\z@
        \@count@chars#2\@nil
        \edef\@tempb{#1=\the\@tempcnta\relax}%
    \@xp\endgroup
    \@tempb
}
%    \end{macrocode}
%    \end{macro}
%
%    \begin{macro}{\@count@chars}
%    \begin{macrocode}
\def\@count@chars#1{%
    \ifx #1\@nil
    \else
        \advance\@tempcnta\@ne
        \@xp\@count@chars
    \fi
}
%    \end{macrocode}
%    \end{macro}
%
% \subsection{Generating alphabetic labels}
%
% \subsubsection{The algorithm}
%
%    Like Gaul, an alphabetic label is divided into three parts.
%    \begin{enumerate}
%
%    \item The author part.  In the simplest case, this is formed by
%    extracting the first character of each word of each last name of
%    each author.  Thus, if there were two authors with last names
%    ``Vaughan Williams'' and ``Tallis'', the author part would be
%    ``VWT''.
%
%    If there are more than four authors, only the first three names
%    are used, and a superscript~``+'' is appended to represent the
%    elided names.  Similarly, if an author name is ``others'',
%    it is replaced by a superscript~``+'' and any following author
%    names (of which there shouldn't be any) are ignored.
%
%    Finally, if there is only one author and the author's last name
%    consists of a single word, the first three characters of that
%    name are used.
%
%    \item The year part.  If the \opt{y2k} option is in force, or if
%    the year is less than~1901, the entire year is used.  Otherwise
%    the last two digits of the year are used.\footnote{Years with
%    more than 4 digits are not currently handled correctly.
%    \emph{Caveat lector}.}
%
%    The combination of author part and year part will be referred to
%    as the \emph{stem}.
%
%    \item[3.] The suffix.  If two or more items have the same stems,
%    a suffix consisting of a lowercase latin letter will be appended
%    to each label to make it unique.
%
%    \end{enumerate}
%
%    \label{stem-comparison}
%    This third part is more subtle than it might first appear.
%    First, case is ignored when comparing stems, so that, for
%    example, ``Ahl1999'' and ``AHL1999'' are considered identical.
%    Second, existing practice (in English, at least), is to ignore
%    diacritics so that, for example, ``Ahl1999'' and ``\"Ahl1999''
%    are considered identical.
%
%    Note that when checking for duplicate stems, we assume that
%    bibliography items appear sorted by label, which means that all
%    items with the same stem will be adjacent.  This means we can use
%    the naive algorithm (check to see if the current item has the
%    same stem as the previous item and, if so, append a suffix) to
%    detect clashes.  This sorting will be done automatically by
%    \bst{amsxport}, but the document author is responsible for
%    ensuring the appropriate order if \bst{amsxport} is not used.
%    This is why it's an error to mix the \opt{alphabetic} and
%    \opt{citation-order} options.
%
% \subsubsection{The implementation}
%
%    \begin{macrocode}
\let\previous@stem\@empty
\let\current@stem\@empty
%    \end{macrocode}
%
%    \begin{macrocode}
\let\previous@year\@empty
\let\current@year\@empty
%    \end{macrocode}
%
%    \begin{macro}{\append@to@stem}
%    \begin{macrocode}
\def\append@to@stem{\global\@concat\current@stem}
%    \end{macrocode}
%    \end{macro}
%
%    \begin{macro}{\generate@alphalabel}
%    \begin{macrocode}
\def\generate@alphalabel{%
%    \end{macrocode}
%    If the user supplied an explicit \fld{label} field, we use it.
%    Otherwise, we generate our own.
%    \begin{macrocode}
    \ifx\bib'label\@empty
        \begingroup
%    \end{macrocode}
%    We begin by saving the previous stem and initializing the
%    current stem to the empty string.
%    \begin{macrocode}
            \global\let\previous@stem\current@stem
            \global\let\current@stem\@empty
%    \end{macrocode}
%    The list of primary contributors is available to us in
%    \cs{current@primary} in the form
%    \begin{quote}
%    \cnm{name}{$\mathrm{Last}_1, \mathrm{First}_1$}%
%    \cnm{name}{$\mathrm{Last}_2, \mathrm{First}_2$}
%     \ldots
%    \cnm{name}{$\mathrm{Last}_n, \mathrm{First}_n$}
%    \end{quote}
%    We will be executing this list multiple times with various
%    definitions of \cs{name}.  So the first thing we want to do is
%    establish a safe environment and normalize the names.
%    \begin{macrocode}
            \@apply\auto@protect\amsrefs@textsymbols
            \@apply\auto@protect\amsrefs@textaccents
            \auto@protect\name
            \auto@protect\etaltext
            \normalize@edef\@tempa\current@primary
%    \end{macrocode}
%    Now we count the number of authors in the list and invoke the
%    appropriate macro to calculate the author part of the
%    reference label.
%    \begin{macrocode}
            \get@numberof\@tempcnta\name\@tempa
            \calc@author@part
%    \end{macrocode}
%    Next append the year part.
%    \begin{macrocode}
            \append@label@year
%    \end{macrocode}
%    At this point, the \cs{current@stem} is complete and we're ready
%    to determine what (if any) suffix is needed to disambiguate
%    it from the previous label.
%    \begin{macrocode}
            \calc@alpha@suffix
%    \end{macrocode}
%    We have all the pieces now.  Arrange to end the current group and
%    then define \cs{bib@label} in the enclosing group.  (This keeps
%    \cs{bib@label} from being defined outside the group started by
%    \cs{bib@start}.  This isn't strictly necessary, but it provides a
%    bit of compartmentalization.)
%    \begin{macrocode}
            \edef\@tempa{%
                \def\@nx\bib'label{%
                    \current@stem
                    \alpha@label@suffix
                }%
            }%
        \@xp\endgroup
        \@tempa
    \fi
}
%    \end{macrocode}
%    \end{macro}
%
%    \begin{macro}{\calc@author@part}
%    \begin{macrocode}
\def\calc@author@part{%
    \ifnum \@tempcnta = 1
        \@xp\@oneauthorlabel\@xp{\@tempa}%
    \else
        \@xp\@multiauthorlabel\@xp{\@tempa}%
    \fi
}
%    \end{macrocode}
%    \end{macro}
%
%    \begin{macro}{\@firstone}
%    This extracts the first character from a properly prepared author
%    name (i.e., one in which accents are properly wrapped).
%    \begin{macrocode}
\def\@firstone#1{\@car#1\@empty\@nil}
%    \end{macrocode}
%    \end{macro}
%
%    \begin{macro}{\@firstthree}
%    And this extracts the first three characters.
%    \begin{macrocode}
\def\@firstthree#1{\@carcube#1\@empty\@empty\@empty\@nil}
%    \end{macrocode}
%    \end{macro}
%
%    \begin{macro}{\@nametoken}
%    \begin{macrocode}
\let\@nametoken\@firstone
%    \end{macrocode}
%    \end{macro}
%
%    \begin{macro}{\hyph@to@space}
%    \begin{macrocode}
\def\hyph@to@space#1-{#1 \hyph@to@space}
%    \end{macrocode}
%    \end{macro}
%
%    \begin{macro}{\@marknames}
%    Since we have a~' with funny catcode already, let's use it (being
%    able to easily put a space after the |'| makes things easier).
%    \begin{macrocode}
\def\@marknames#1{%
    \@ifnotempty{#1}{\surround@names#1 ' }%
}
%    \end{macrocode}
%    \end{macro}
%
%    \begin{macro}{\surround@names}
%    \begin{macrocode}
\def\surround@names#1 {%
    \ifx '#1%
    \else
        \@nx\@nametoken{#1}%
        \@xp\surround@names
    \fi
}
%    \end{macrocode}
%    \end{macro}
%
%    \begin{macro}{\extract@surnames}
%    \begin{macrocode}
\def\extract@surnames#1#2{%
    \get@namepart\@tempb\@nilgobble #2,\@nil
    \edef\@tempb{\@nx\@marknames{\@xp\hyph@to@space\@tempb\@gobble-}}%
    \edef#1{\@tempb}%
}
%    \end{macrocode}
%    \end{macro}
%
%    \begin{macro}{\@oneauthorlabel}
%    This is the easy case.
%    \begin{macrocode}
\newcommand{\@oneauthorlabel}[1]{%
    \def\name##1{%
        \extract@surnames\@tempa{##1}%
        \get@numberof\@tempcnta\@nametoken\@tempa
        \ifnum \@tempcnta = 1
            \let\@nametoken\@firstthree
        \fi
        \append@to@stem{\@tempa}%
    }%
    #1%
}
%    \end{macrocode}
%    \end{macro}
%
%    \begin{macro}{\@threeauthors}
%    \begin{macrocode}
\def\@threeauthors\name#1\name#2\name#3#4\@empty{%
    \name{#1}\name{#2}\name{#3}%
    \append@to@stem{\etalchar{+}}%
}
%    \end{macrocode}
%    \end{macro}
%
%    \begin{macro}{\@multiauthorlabel}
%    \begin{macrocode}
\newcommand{\@multiauthorlabel}[1]{%
    \def\name##1{%
        \ifx\etaltext ##1%
            \def\@tempa{\@nx\etalchar{+}}%
            \let\name\@gobble
        \else
            \extract@surnames\@tempa{##1}%
        \fi
        \append@to@stem{\@tempa}%
    }%
    \ifnum \@tempcnta > 4 \@xp \@threeauthors \fi
    #1\@empty
}
%    \end{macrocode}
%    \end{macro}
%
%    \begin{macro}{\etalchar}
%    \begin{macrocode}
\newcommand{\etalchar}[1]{$^{#1}$}
%    \end{macrocode}
%    \end{macro}
%
%    \begin{macro}{\year@short}
%    For alphanumeric labels, we want to extract the last 2 digits of
%    the year.  Here's a way to do that, assuming a 4-digit year.
%    \begin{macrocode}
\def\year@short#1#2#3#4\@nil{#3#4}
%    \end{macrocode}
%    \end{macro}
%
%    \begin{macro}{\append@label@year}
%    \begin{macrocode}
\def\append@label@year{%
    \safe@set\@tempcnta\bib@year
    \edef\bib@citeyear{\the\@tempcnta}%
    \append@to@stem{%
        \ifx\bib@year\@empty
        \else
            \@xp\year@short \bib@citeyear \@nil
        \fi
    }%
}
%    \end{macrocode}
%    \end{macro}
%
%    \begin{macrocode}
\let\alpha@label@suffix\@empty

\newcount\alpha@suffix
\alpha@suffix\@ne
\let\@suffix@format\@alph
%    \end{macrocode}
%
%    \begin{macro}{\calc@alpha@suffix}
%    \begin{macrocode}
\def\calc@alpha@suffix{%
    \@tempswafalse
    \compare@stems\previous@stem\current@stem
    \ifsame@stems
%    \end{macrocode}
%    Under the \opt{alphabetic} option, \cs{previous@year} and
%    \cs{current@year} will always be the same (namely, both will be
%    empty), but including the test allows this code to work with the
%    \opt{author-year} option as well.
%    \begin{macrocode}
        \ifx\previous@year\current@year
            \@tempswatrue
        \fi
    \fi
    \if@tempswa
        \global\advance\alpha@suffix\@ne
        \edef\alpha@label@suffix{\@suffix@format\alpha@suffix}%
        \ifnum\alpha@suffix=\tw@
            \immediate\write\@auxout{%
                \string\ModifyBibLabel{\prev@citekey}%
            }%
        \fi
    \else
        \let\alpha@label@suffix\@empty
        \global\alpha@suffix\@ne
        \@xp\ifx \csname b@\current@citekey @suffix\endcsname \relax
        \else
            \edef\alpha@label@suffix{\@suffix@format\alpha@suffix}%
        \fi
    \fi
}
%    \end{macrocode}
%    \end{macro}
%
%    \begin{macro}{\ifsame@stems}
%    \begin{macrocode}
\newif\ifsame@stems
%    \end{macrocode}
%    \end{macro}
%
%    \begin{macro}{\compare@stems}
%    \begin{macrocode}
\def\compare@stems#1#2{%
    \begingroup
        \purge@edef\@tempa{#1}%
        \purge@edef\@tempb{#2}%
        \lc@edef\@tempa{\@tempa}%
        \lc@edef\@tempb{\@tempb}%
        \ifx\@tempa\@tempb
            \def\@tempa{\same@stemstrue}%
        \else
            \def\@tempa{\same@stemsfalse}%
        \fi
    \@xp\endgroup
    \@tempa
}
%    \end{macrocode}
%    \end{macro}
%
%    \begin{macro}{\ModifyBibLabel}
%    \begin{macrocode}
\def\ModifyBibLabel#1{%
    \global\@xp\let\csname b@#1@suffix\endcsname\@empty
}
%    \end{macrocode}
%    \end{macro}
%
% \subsection{Generating short alphabetic labels}
%
%    This style for alphabetic labels is somewhat simpler than the
%    regular alphabetic style.  The stem consists only of an author
%    part without a year part.  The author part is formed in the same
%    way, except that even when there is only a single author with a
%    one-word last name, only the first letter of the name is used,
%    not the first three.  Finally, the suffix used to disambiguate
%    identical stems is numeric rather than alphabetic.
%
%    See section~\ref{opt:shortalphabetic} on
%    page~\pageref{opt:shortalphabetic} for the implementation.
%
% \subsection{Formatting series}
%
%    The \cn{PrintSeries} command prints a list of objects in series
%    form. The essential idea is to produce something like \qq{A, B, and
%    C} when we are given three elements ``A'', ``B'', and ``C'', with
%    suitable variations in the punctuation and other intervening
%    material depending on the number of elements.
%
%    More precisely, we can envision \cn{PrintSeries} being
%    called as
%    \begin{quote}
%    |\PrintSeries{|$S$|}{|$i_1$|}{|$i_2$|}{|$i_3$|}{|$E$|}{\do{|$T_1$|}| \dots|\do{|$T_n$|}}|
%\end{quote}
%    where $S$ and~$E$ are material to be interpolated before the
%    start and after the end of the list, respectively,
%    $i_1$, \dots, $i_3$ are material to be interpolated between the
%    elements, and the final argument is a list of indeterminate
%    length where each element consists of a macro and its
%    argument. If there are exactly two elements, $i_1$ is inserted
%    between them; otherwise, $i_2$ is inserted between each pair of
%    items except for the last pair, where $i_3$ is inserted.  Thus,
%    \begin{center}
%    \begin{tabular}{ll}
%    $n$    & output \\
%    1      & $S$ $T_1$ $E$ \\
%    2      & $S$ $T_1$ $i_1$ $T_2$ $E$ \\
%    3      & $S$ $T_1$ $i_2$ $T_2$ $i_3$ $T_3$ $E$ \\
%    4      & $S$ $T_1$ $i_2$ $T_2$ $i_2$ $T_3$ $i_3$ $T_4$ $E$
%^^A    \\$>4$   & $S$ $T_1$ $i_2$ $T_2$ $i_2$ $T_3$ \dots $T_{n-1}$ $i_3$ $T_n$ $E$
%    \end{tabular}
%    \end{center}
%    and so forth.  For example, a standard comma-separated list could
%    be formatted by
%    \begin{verbatim}
%    \PrintSeries{}{ and }{, }{, and }{}{...}
%\end{verbatim}
%
%    That is the simple case but in practice there are
%    additional complications. What if user-supplied line breaks have to
%    be supported at the boundaries between elements? What if in
%    addition to adding material between elements we also want to apply
%    some handy function to each element (e.g., \cn{textsc})? Even
%    worse, what if we want the function to be different depending on
%    the position of the element in the list? Indeed if this did not
%    happen to be the case with the current application I would not have
%    gone to the extra trouble of supporting it. But if it must be so,
%    then the output that we need from a list |\do{A}\do{B}...| is
%    \begin{verbatim}
%    f0{A}
%    f0{A} p1 i1 f1{B}
%    f0{A} p2 i2 f2{B} p3 i3 f3{B}
%\end{verbatim}
%    and so on, where
% \begin{itemize}
% \item $f_n$ is a macro taking one argument,
% \item $p_n$ is punctuation\mdash material that must precede a line
%    break if one occurs at this boundary,
% \item $i_n$ other interpolated material, as before.
% \end{itemize}
%    To reduce the number of distinct required objects we decree that
%    each element will get braces wrapped around it as a matter of
%    course; then it is possible for $f_1$, $f_2$, $f_3$ to be
%    assimilated onto the tail end of $i_1$, $i_2$, $i_3$.  Since we
%    also have to specify the macro that delimits the elements
%    of the list, we end up with the following rather formidable
%    signature:
%    \begin{quote}\raggedright
%    |\PrintSeries{\m} {|$f_0$|}|
%    |{|$p_1$|}{|$i_1 f_1$|}| |{|$p_2$|}{|$i_2 f_2$|}|
%    |{|$p_3$|}{|$i_3 f_3$|}|
%    \hspace*{2 em} |{|$S$|} {\m{|$T_1$|}|$\dots$|\m{|$T_n$|}} {|$E$|}|
%\end{quote}
%    and our comma-separated list example becomes
%    \begin{verbatim}
%    \PrintSeries{\do}{} {}{ and } {,}{ } {,}{ and } {}{...}{}
%\end{verbatim}
%
%    \begin{macro}{\series@index}
%    First we define a dedicated count register to be used in tracking
%    the ordinal number of the item currently being processed.
%    \begin{macrocode}
\newcount\series@index
%    \end{macrocode}
%    \end{macro}
%
%    \begin{macro}{\PrintSeries}
%    \begin{macrocode}
\def\PrintSeries#1#2#3#4#5#6#7#8{%
    \begingroup
        \def\series@add@a{#2}%
        \def\series@add@b{\SwapBreak{#3}#4}%
        \def\series@add@c{\SwapBreak{#5}#6}%
        \def\series@add@d{\SwapBreak{#7}#8}%
        \def\series@add@e{\SwapBreak{#7}}%
        \PrintSeries@a{#1}%
}
%    \end{macrocode}
%    \end{macro}
%
%    \begin{macro}{\PrinteSeries@a}
%    For \cs{PrintSeries@a} the first arg is the iterator function
%    present in the list which is arg 3. Args 2 and 4 are extra material
%    to be added before and after the list that may require the use of
%    \cn{Plural} or \cn{SingularPlural}.
%    \begin{macrocode}
\def\PrintSeries@a#1#2#3#4{%
        \get@numberof\@tempcnta#1{#3}%
        \chardef\series@total=\@tempcnta
        \ifnum\series@total=\@ne
            \let\SingularPlural\@firstoftwo
        \else
            \let\SingularPlural\@secondoftwo
        \fi
        \series@index=\z@
        \let#1\series@add
        #2#3#4\relax
    \endgroup
}
%    \end{macrocode}
%    \end{macro}
%
%    \begin{macro}{\series@add}
%    This is the inner function called by \cn{PrintSeries} that
%    carefully distributes all the material stored previously in
%    \cs{series@add@...} macros.
%
%    Note that the handling of ``et al.''\ cases is somewhat
%    hardcoded.  This seemed preferable to adding yet another argument
%    (or two!\@) to \cs{PrintSeries}.
%    \begin{macrocode}
\def\series@add#1{%
    \advance\series@index\@ne
    \ifx\etaltext#1\relax
        \ifnum\series@index=\tw@
            \def\@tempa{\space\SubEtal}%
        \else
            \def\@tempa{\series@add@e\space\SubEtal}%
        \fi
%    \end{macrocode}
%    We assume there are fewer than 20,000 items in the list.
%    \begin{macrocode}
        \series@index\@MM
    \else
        \ifcase\series@index
        \or
%    \end{macrocode}
%    Material before name 1:
%    \begin{macrocode}
            \let\@tempa\series@add@a
        \or
%    \end{macrocode}
%    Material before name 2:
%    \begin{macrocode}
            \ifnum\series@total<\thr@@
                \let\@tempa\series@add@b
            \else
                \let\@tempa\series@add@c
            \fi
        \else
%    \end{macrocode}
%    Material before names 3, 4, 5,\dots
%    \begin{macrocode}
            \ifnum\series@index=\series@total
                \let\@tempa\series@add@d
            \else
                \ifnum\series@index<\series@total
                    \let\@tempa\series@add@c
                \else
                    \let\@tempa\@gobble
                \fi
            \fi
        \fi
    \fi
    \@tempa{#1}%
}
%    \end{macrocode}
%    \end{macro}
%
%    \begin{macro}{\SwapBreak}
%    This takes a single argument, which should begin with a
%    punctuation character, and conditionally appends it to the
%    current horizontal list after removing any preceding whitespace.
%    If there was also a penalty at the end of the hlist (presumed to
%    be the result of a |\linebreak| at the end of a field value), it
%    moves the penalty to \emph{after} the argument.
%
%    \begin{bug}
%    \cn{SwapBreak} interferes with \TeX's kerning mechanism.  For
%    example, consider a field value that ends with a~``y'' and that
%    should have a comman automatically appended.  \pkg{amsrefs}
%    generates the equivalent of |y\SwapBreak{,}|, which results in
%    ``y\SwapBreak{,}'' (no kern before the comma) rather than~``y,''.
%    Unfortunately, fixing this would likely require a
%    disproportionate effort.  In cases where the lack of kerning is
%    unacceptable, a workaround is to add the punctuation mark to the
%    field value manually.  For example, |title={...y,}| would
%    generate the equivalent of |y,\SwapBreak{,}|, which in turn would
%    produce ``y,\SwapBreak{,}'' since \cs{SwapBreak} is careful not
%    to add duplicate punctuation.
%    \end{bug}
%    \begin{macrocode}
\def\SwapBreak#1{%
    \relax\ifvmode\leavevmode\fi
    \@tempcnta\@MM
    \toks@{#1}%
%    \end{macrocode}
%    First, remove any preceding glue.  (There usually shouldn't be
%    any of this.)
%    \begin{macrocode}
    \unskip
%    \end{macrocode}
%    There might be also be kern, typically an italic correction left
%    there by a previous \texttt{TextFontCommand} like \ncn{textit}.
%    But don't remove the special 1\,sp kern used to mark the
%    beginning of a bibliography entry.
%
%    \begin{bug}
%    Sometimes we want to keep the italic correction.
%    \end{bug}
%    \begin{macrocode}
    \ifnum \lastkern>\@ne \unkern \fi
%    \end{macrocode}
%    And now look for a penalty and stash it in a safe place.
%    \begin{macrocode}
    \ifnum\lastpenalty=\z@
    \else
        \@tempcnta\lastpenalty
        \unpenalty
    \fi
%    \end{macrocode}
%    Now we add the punctuation, \emph{unless} one of the following
%    conditions is true:
%    \begin{enumerate}
%
%    \item
%    The last item on the horizontal list was a kern of~1\,sp,
%    indicating that we're at the very beginning of a bibliography
%    item.
%
%    \item
%    The current space factor is equal to the \cs{sfcode} of the
%    puncutation mark we are adding, meaning that the mark is already
%    on the list.
%
%    \item
%    The current space factor is equal to the special value
%    |\@nopunctsfcode|, meaning that |\nopunct| was specified.
%
%    \end{enumerate}
%    This relies on distinct punctuation marks having distinct space
%    factors, as established by our definition of \cn{frenchspacing}.
%    \begin{macrocode}
    \edef\@tempa{%
        \@nx\deferredquoteslogical
        \ifnum\lastkern=\@ne
        \else
            \ifnum\spacefactor=\sfcode\@xp\@xp\@xp`\@xp\@car\string#1)\@nil
            \else
                \ifnum\spacefactor=\@nopunctsfcode
                \else
                    \the\toks@
                \fi
            \fi
        \fi
        \@nx\deferredquotes
        \ifnum\@tempcnta=\@MM \else \penalty\number\@tempcnta\space \fi
        \ifnum\lastkern=\@ne \ignorespaces \fi
    }%
    \@tempa
}
%    \end{macrocode}
%    \end{macro}
%
%    \begin{macro}{\Plural}
%    \begin{macro}{\SingularPlural}
%    \cn{Plural} takes one argument and prints it if there were two or
%    more elements in the current list. So, to get \qq{editors} instead
%    of \qq{editor} after printing a list of editor names, write
%    \verb'editor\Plural{s}'.
%
%    \cn{SingularPlural} takes two arguments and prints the first if
%    there was only one element, otherwise prints the second arg.
%    \begin{macrocode}
\newcommand{\SingularPlural}[2]{#1}
\newcommand{\Plural}{\SingularPlural{}}
%    \end{macrocode}
%    \end{macro}
%    \end{macro}
%
% \subsection{Formatting names and series of names}\label{name-style}
%
%    Now that we have a general mechanism for formatting series, we
%    can easily specialize to the common case of a comma-separated
%    list of names.  First we provide specifications for the three
%    most common name formats.
%
%    \begin{macro}{\setbib@nameLE}
%    This sets a name in standard western uninverted order, e.g.,
%    ``John Doe Jr.''  (The ``LE'' stands for little-endian.)
%    \begin{macrocode}
\BibSpec{nameLE}{
    +{}{}{given}
    +{}{\IfEmptyBibField{given}{}{ }}{surname}
    +{}{ }{jr}
}
%    \end{macrocode}
%    \end{macro}
%
%    \begin{macro}{\setbib@nameBE}
%    Big-endian order, as used for example in traditional Chinese,
%    Japanese, Vietnamese, and Hungarian names: ``Doe John''.
%    Big-endian formatting can be requested for name by setting the
%    ``inverted'' property to ``yes.''
%    \begin{macrocode}
\BibSpec{nameBE}{
    +{}{}{surname}
    +{}{ }{given}
%    \end{macrocode}
%    I don't know what should happen if there's a suffix, so I'm going
%    to just leave it out for now (although I should probably issue a
%    warning).  I suspect that either (a)~it never comes up or (b)~if
%    it does come up, there's no set standard for how it should be
%    handled.
%    \begin{macrocode}
%    +{}{ }{jr}
}
%    \end{macrocode}
%    \end{macro}
%
%    \begin{macro}{\setbib@nameinverted}
%    Inverted western-style names: ``Doe, John, Jr.''
%    \begin{macrocode}
\BibSpec{nameinverted}{
    +{} {}  {surname}
    +{,}{ } {given}
    +{,}{ } {jr}
}
%    \end{macrocode}
%    \end{macro}
%
%    Incidentally, it would probably be cleaner if names had their
%    own namespace like properties do, i.e., something like
%    \begin{verbatim}
%    \DefineSimpleKey{name}{given}
%    \DefineSimpleKey{name}{initials}
%    \DefineSimpleKey{name}{surname}
%    \DefineSimpleKey{name}{jr}
%\end{verbatim}
%    followed by
%    \begin{verbatim}
%    \NameSpec{nameLE}{...}
%\end{verbatim}
%    or
%    \begin{verbatim}
%    \BibSpec[name]{nameLE}{...}
%\end{verbatim}
%    But this seems a little extravagant at this stage, so I've
%    decided to leave things as-is for now.
%
%    \begin{macro}{\PrintNames}
%    \cn{PrintNames} is a simplified interface to \cs{PrintSeries} that
%    takes only the last three arguments:
%    \begin{quote}\raggedright
%    |\PrintNames {|$S$|} {|$E$|} {\name{|$T_1$|}|$\dots$|\name{|$T_n$|}}|
%\end{quote}
%    The order of the last two arguments is reversed to make it
%    moderately easier to use; cf.\ \cn{PrintEditorsA}, etc.
%
%    The first name in a series is treated differently than the other
%    names in the \opt{author-year} style, so we use a separate
%    formatting macro for it.
%    \begin{macrocode}
\newcommand{\PrintNames}{%
    \@ifstar{\PrintNames@a\set@othername}{\PrintNames@a\set@firstname}%
}
%    \end{macrocode}
%    \end{macro}
%
%    \begin{macro}{\PrintNames@a}
%    \begin{macrocode}
\newcommand{\PrintNames@a}[4]{%
    \PrintSeries{\name}
        {#1}
        {}{ and \set@othername}
        {,}{ \set@othername}
        {,}{ and \set@othername}
        {#2}{#4}{#3}%
}
%    \end{macrocode}
%    \end{macro}
%
%    \begin{macro}{\set@firstname}
%    By default, the first name is formatted in little-endian format.
%    The \opt{author-year} option changes this to inverted order.
%    \begin{macrocode}
\def\set@firstname#1{%
    \set@name{#1}\setbib@nameLE
}
%    \end{macrocode}
%    \end{macro}
%
%    \begin{macro}{\set@othername}
%    The rest of the names are set in little-endian format by
%    default.
%    \begin{macrocode}
\def\set@othername#1{%
    \set@name{#1}\setbib@nameLE
}
%    \end{macrocode}
%    \end{macro}
%
%    \begin{macro}{\set@name}
%    Parse the name into its components and then pass control to
%    \cs{set@name@a}, which will decide what format to use for the
%    name.
%    \begin{macrocode}
\def\set@name#1{%
    \name@split#1,,,\@nil
    \set@name@a
}
%    \end{macrocode}
%    \end{macro}
%
%    \begin{macro}{\set@namea}
%    Use the requested format unless the |order| property has been set
%    to ``inverted.''
%    \begin{macrocode}
\def\set@name@a#1{%
    \begingroup
        \get@current@properties
        \select@auxlanguage
        \def\@tempa{yes}%
        \ifx\@tempa\prop'inverted
            \setbib@nameBE
        \else
            #1%
        \fi
    \endgroup
}
%    \end{macrocode}
%    \end{macro}
%
%    \begin{macro}{\PrintPrimary}
%    \begin{macrocode}
\def\PrintPrimary{%
    \ifx\current@primary\@empty
        \EmptyPrimaryWarning
    \else
        \print@primary\current@primary
    \fi
}
%    \end{macrocode}
%    \end{macro}
%
%    \begin{macro}{\EmptyPrimaryWarning}
%    \begin{macrocode}
\def\EmptyPrimaryWarning{%
    \amsrefs@warning{No authors, editors or translators}%
}
%    \end{macrocode}
%    \end{macro}
%
%    \begin{macro}{\PrintAuthors}
%    The comparison of \cs{previous@primary} and \cs{current@primary}
%    doesn't look at auxiliary properties (see also \cs{PrintEditorsA}
%    and \cs{PrintTranslatorsA}).  This is probably ok.
%    \begin{macrocode}
\newcommand{\PrintAuthors}[1]{%
    \ifx\previous@primary\current@primary
        \sameauthors\@empty
    \else
        \def\current@bibfield{\bib'author}%
        \PrintNames{}{}{#1}%
    \fi
}
%    \end{macrocode}
%    \end{macro}
%
%    \begin{macro}{\sameauthors}
%    \begin{macrocode}
\newcommand{\sameauthors}[1]{\bysame#1}
%    \end{macrocode}
%    \end{macro}
%
%    \begin{macro}{\bysame}
%    \begin{macrocode}
\def\bysame{%
    \leavevmode\hbox to3em{\hrulefill}\thinspace
    \kern\z@
}
%    \end{macrocode}
%    \end{macro}
%
%    \begin{macro}{\PrintNameList}
%    This just prints the names without any additional information.
%    \begin{macrocode}
\newcommand{\PrintNameList}{\PrintNames{}{}}
%    \end{macrocode}
%    \end{macro}
%
%    \begin{macro}{\PrintEditorsC}
%    \begin{macrocode}
\newcommand{\PrintEditorsC}[1]{%
    \PrintNames{Edited by }{}{#1}%
}
%    \end{macrocode}
%    \end{macro}
%
%    \begin{macro}{\PrintEditorsA}
%    When we consider editor names we have to think about some further
%    complications. First, for the case of a book where editor names are
%    listed in place of author names, just copy the same style with a
%    bit of added text at the end.
%    \begin{macrocode}
\newcommand{\PrintEditorsA}[1]{%
    \ifx\previous@primary\current@primary
        \sameauthors{(ed\Plural{s}.)}%
    \else
        \def\current@bibfield{\bib'editor}%
        \PrintNames{}{ (ed\Plural{s}.)}{#1}%
    \fi
    \erase@field\bib'editor
}
%    \end{macrocode}
%    \end{macro}
%
%    \begin{macro}{\PrintEditorsB}
%    \begin{macrocode}
\newcommand{\PrintEditorsB}{%
    \PrintNames*{(}{\SwapBreak{,}~ed\Plural{s}.)}%
}
%    \end{macrocode}
%    \end{macro}
%
%    \begin{macro}{\PrintContributions}
%    \begin{macrocode}
\newcommand{\PrintContributions}[1]{%
    \PrintSeries
        {\fld@elt}
        {\print@contribution}
        {}{ and \print@contribution}
        {,}{ \print@contribution}
        {,}{ and \print@contribution}{}{#1}{}%
}
%    \end{macrocode}
%    \end{macro}
%
%    \begin{macro}{\print@contribution}
%    \begin{macrocode}
\newcommand{\print@contribution}[1]{%
    \in@={#1}%
    \ifin@
        \ifnum\series@index=\@ne with \fi
        \RestrictedSetKeys{}{bib}{%
            \bib@print@inner\setbib@contribution{\the\rsk@toks}%
        }{#1}%
    \else
        #1%
    \fi
}
%    \end{macrocode}
%    \end{macro}
%
%    \begin{macro}{\resolve@inner}
%    \begin{macrocode}
\def\resolve@inner#1#2{%
    \in@={#2}%
    \ifin@
        \RestrictedSetKeys{}{bib}{#1{\the\rsk@toks}}{#2}%
    \else
        \@ifundefined{bi@#2}{%
            \XRefWarning{#2}%
        }{%
            #1{\csname bi@#2\endcsname}%
        }%
    \fi
}
%    \end{macrocode}
%    \end{macro}
%
%    \begin{macro}{\PrintConference}
%    \begin{macrocode}
\def\PrintConference{%
    \resolve@inner{\bib@print@inner\setbib@conference}
}
%    \end{macrocode}
%    \end{macro}
%
%    \begin{macro}{\PrintConferenceDetails}
%    \begin{macrocode}
\def\PrintConferenceDetails#1{%
    \ifx\@empty\bib'address
        \ifx\@empty\bib'date
        \else
            \PrintConferenceDetails@
        \fi
    \else
        \PrintConferenceDetails@
    \fi
}
%    \end{macrocode}
%    \end{macro}
%
%    \begin{macro}{\PrintConferenceDetails@}
%    \begin{macrocode}
\def\PrintConferenceDetails@{%
    \ifnum\lastkern=\@ne\else\space\fi(\kern 1sp
    \ifx\@empty\bib'address
    \else
        \bib'address
    \fi
    \ifx\@empty\bib'date
    \else
        \SwapBreak{,}\space
        \print@date
    \fi
    )%\spacefactor\sfcode`\,%
}
%    \end{macrocode}
%    \end{macro}
%
%    \begin{macro}{\PrintBook}
%    \begin{macrocode}
\def\PrintBook{%
    \resolve@inner{\bib@print@inner\setbib@innerbook}
}
%    \end{macrocode}
%    \end{macro}
%
%    \begin{macro}{\PrintReprint}
%    \begin{macrocode}
\def\PrintReprint{%
    \resolve@inner{\bib@reprint}
}
%    \end{macrocode}
%    \end{macro}
%
%    \begin{macro}{\bib@reprint}
%    \begin{macrocode}
\def\bib@reprint#1{%
    \begingroup
        #1\relax            % execute definitions locally
        \bib@field@patches
        \bib'setup
        \IfEmptyBibField{copula}{reprinted in}{\bib'copula} \nopunct
        \let\bib'language\@empty
        \setbib@book
    \endgroup
}
%    \end{macrocode}
%    \end{macro}
%
%    \begin{macro}{\PrintTranslation}
%    \begin{macrocode}
\def\PrintTranslation{%
    \resolve@inner{\bib@translation}
}
%    \end{macrocode}
%    \end{macro}
%
%    \begin{macro}{\bib@translation}
%    \begin{macrocode}
\def\bib@translation#1{%
    \begingroup
        #1\relax            % execute definitions locally
        \bib@field@patches
        \bib'setup
        \let\PrintPrimary\@empty
        \bib@append{;}{ % keep this space!
            \IfEmptyBibField{language}{English}{\bib'language} transl.%
            \IfEmptyBibField{pages}{ in \kern\@ne sp}{, }%
        }\bib'transition
        \let\bib'language\@empty
        \setbib@@
    \endgroup
}
%    \end{macrocode}
%    \end{macro}
%
%    \begin{macro}{\PrintTranslatorsC}
%    \begin{macrocode}
\newcommand{\PrintTranslatorsC}[1]{%
    \PrintNames{translated by }{}{#1}%
}
%    \end{macrocode}
%    \end{macro}
%
%    \begin{macro}{\PrintTranslatorsA}
%    \begin{macrocode}
\newcommand{\PrintTranslatorsA}[1]{%
    \ifx\previous@primary\current@primary
        \sameauthors{(trans.)}%
    \else
        \def\current@bibfield{\bib'translator}%
        \PrintNames{}{ (trans.)}{#1}%
    \fi
    \erase@field\bib'translator
}
%    \end{macrocode}
%    \end{macro}
%
%    \begin{macro}{\PrintTranslatorsB}
%    \begin{macrocode}
\newcommand{\PrintTranslatorsB}[1]{
    \PrintNames*{(}{\SwapBreak{,}~tran\Plural{s}.)}%
}
%    \end{macrocode}
%    \end{macro}
%
%    Some special handling for ``et alii'' or ``and others''.
%    \begin{macrocode}
\DefineName{alii}{\etaltext}
\DefineName{others}{\etaltext}
%    \end{macrocode}
%
%    \begin{macro}{\etaltext}
%    \begin{macro}{\SubEtal}
%    The Chicago Manual of Style suggests that it is slightly better not
%    to italicize `et al' and some other extremely common abbreviations
%    inherited from Latin. (Compare `etc'.)
%    \begin{macrocode}
\newcommand{\etaltext}{et al.}
\newcommand{\SubEtal}[1]{\etaltext}
%    \end{macrocode}
%    \end{macro}
%    \end{macro}
%
% \subsection{The \fld{partial} field}
%
%    \begin{macro}{\print@partial}
%    \begin{macrocode}
\newcommand{\print@partial}{%
    \resolve@inner{\bib@print@inner\setbib@partial}
}
%    \end{macrocode}
%    \end{macro}
%
% \subsection{Special formatting for other fields}
%
%    \begin{macro}{\parenthesize}
%    The \cn{parenthesize} function adds parentheses around its
%    argument, calling \cn{upn} to optionally prevent italic
%    parentheses from being used.
%    \begin{macrocode}
\newcommand{\parenthesize}[1]{%
    \leavevmode\push@bracket)\upn{(}#1\pop@bracket
}
%    \end{macrocode}
%    \end{macro}
%
%    \begin{macro}{\upn}
%    By default, \cn{upn} is a no-op, meaning that this refinement
%    lies dormant unless the \pkg{upref} package or other activation
%    is done. (Probably better done via special fonts, anyway.)
%    \begin{macrocode}
\providecommand{\upn}[1]{#1}
%    \end{macrocode}
%    \end{macro}
%
%    \begin{macro}{\push@bracket}
%    \begin{macro}{\pop@bracket}
%    \begin{macrocode}
\let\bracket@stack\@empty

\def\push@bracket#1{%
    \xdef\bracket@stack{#1\bracket@stack}%
}

\def\pop@bracket{%
    \iffalse{\fi
    \@xp\pop@bracket@a\bracket@stack \@empty}%
}

\def\pop@bracket@a#1{%
    \leavevmode\/\upn{#1}%
    \xdef\bracket@stack{\iffalse}\fi
}
%    \end{macrocode}
%    \end{macro}
%    \end{macro}
%
%    \begin{macro}{\bibquotes}
%    \begin{macrocode}
\newcommand{\bibquotes}[1]{%
    \textquotedblleft#1%
    \gdef\deferredquotes{%
        \global\let\deferredquotes\@empty
        \textquotedblright
    }%
}
%    \end{macrocode}
%    \end{macro}
%
%    \begin{macro}{\mdash}
%    \begin{macro}{\ndash}
%    Cf.\@ \pkg{textcmds}, where there's also a penalty added.
%    \begin{macrocode}
\providecommand{\mdash}{\textemdash}
\providecommand{\ndash}{\textendash}
%    \end{macrocode}
%    \end{macro}
%    \end{macro}
%
%    \begin{macro}{\MR}
%    \begin{macrocode}
\def\MR#1{%
    \relax\ifhmode\unskip\spacefactor3000 \space\fi
    \def\@tempa##1:##2:##3\@nil{%
        \ifx @##2\@empty##1\else\textbf{##1:}##2\fi
    }%
    \MRhref{#1}{MR \@tempa#1:@:\@nil}%
}
%    \end{macrocode}
%    \end{macro}
%
%    \begin{macro}{\MRhref}
%    For older versions of some AMS document classes, this patch is
%    needed.
%    \begin{macrocode}
\providecommand{\MRhref}[1]{}
%    \end{macrocode}
%    \end{macro}
%
%    \begin{macro}{\PrintReviews}
%    Reviews are handled as a list to support the theoretical
%    possibility of multiple reviews.
%    \begin{macrocode}
\newcommand{\PrintReviews}[1]{%
    \PrintSeries{\fld@elt}{}{,}{ }{,}{ }{,}{ }{}{#1}{}%
}
%    \end{macrocode}
%    \end{macro}
%
%    \begin{macro}{\PrintPartials}
%    \begin{macrocode}
\newcommand{\PrintPartials}[1]{%
    \PrintSeries
        {\fld@elt}
        {\print@partial}
        {;}{ \print@partial}
        {;}{ \print@partial}
        {;}{ \print@partial}{}{#1}{}%
}
%    \end{macrocode}
%    \end{macro}
%
%    \begin{macro}{\PrintISBNs}
%    And similarly for ISBNs. There seem to be a few different
%    situations where one book might have two different ISBN numbers.
%    Here are the ones I know of so far [mjd,2002-02-18]: separate ISBN
%    numbers for hardback and paperback; separate ISBN numbers for U.S.
%    edition and European edition.
%    \begin{macrocode}
\newcommand{\PrintISBNs}[1]{%
    \PrintSeries{\fld@elt}{}{,}{ }{,}{ }{,}{ }{ISBN }{#1}{}%
}
%    \end{macrocode}
%    \end{macro}
%
%    \begin{macro}{\voltext}
%    \begin{macrocode}
\newcommand{\voltext}{\IfEmptyBibField{series}{Vol.~}{vol.~}}
%    \end{macrocode}
%    \end{macro}
%
%    \begin{macro}{\issuetext}
%    \begin{macrocode}
\newcommand{\issuetext}{no.~}
%    \end{macrocode}
%    \end{macro}
%
%    \begin{macro}{\DashPages}
%    Scan the contents of a page value to see if it is a single page.
%    Presence of \cn{ndash} or hyphen is taken to mean no. Probably
%    should test also for spaces and commas. [mjd,2000/01/24]
%    \begin{macrocode}
\newcommand{\DashPages}[1]{%
    p\pp@scan@a#1@\ndash p@\ndash{\pp@scan#1@-p@-{}\@nil}\@nil.~#1%
}

\def\pp@scan#1-#2@-#3#4\@nil{#3}

\def\pp@scan@a#1\ndash#2@\ndash#3#4\@nil{#3}
%    \end{macrocode}
%    \end{macro}
%
%    \begin{macro}{\eprintpages}
%    If we have eprint info and pages info and no journal name, the
%    pages information is presumably the number of pages in the eprint.
%    \begin{macrocode}
\newcommand{\eprintpages}[1]{%
    #1\IfEmptyBibField{eprint}{}{\IfEmptyBibField{journal}{ pp.}{}}%
}
%    \end{macrocode}
%    \end{macro}
%
%    \begin{macro}{\PrintThesisType}
%    \begin{macrocode}
\def\PrintThesisType#1{%
    \thesis@type#1?\@nil{#1}%
}

\def\thesis@type#1#2\@nil#3{%
    \ifx p#1%
        Ph.D. Thesis%
    \else
        \ifx m#1%
            Master's Thesis%
        \else
            #3%
        \fi
    \fi
}
%    \end{macrocode}
%    \end{macro}
%
%    \begin{macro}{\PrintDOI}
%    Perhaps need to add allowbreak penalties at the parentheses in a DOI.
%    Also what about prohibiting a break after the leading S?
%    \begin{macrocode}
\newcommand{\PrintDOI}[1]{%
    DOI #1%
    \IfEmptyBibField{volume}{, (to appear in print)}{}%
}
%    \end{macrocode}
%    \end{macro}
%
%    \begin{macro}{\PrintDatePV}
%    Print date in different forms depending on DOI and volume
%    information.
%    \begin{macrocode}
\newcommand{\PrintDatePV}[1]{%
    \IfEmptyBibField{doi}{%
        \let\@tempa\PrintDate
    }{%
        \IfEmptyBibField{volume}{%
            \let\@tempa\PrintDatePosted
        }{%
            \let\@tempa\PrintDate
        }%
    }%
    \@tempa{#1}%
}
%    \end{macrocode}
%    \end{macro}
%
%    \begin{macro}{\PrintDate}
%    The intent is to handle variations such as \texttt{1987},
%    \texttt{August 1987}, \texttt{1987-08}, and
%    \texttt{1987-08-14}. If the month is present, print August or
%    Aug. or 08 or nothing, at the behest of the bib style.
%
%    We've taken some special care to parse out the date info ahead of
%    time, so this function just discards arg 1 and uses the
%    already-parsed value.
%    \begin{macrocode}
\newcommand{\PrintDate}[1]{(\print@date)}
%    \end{macrocode}
%    \end{macro}
%
%    \begin{macro}{\PrintDateB}
%    The same, but without the parentheses.
%    \begin{macrocode}
\newcommand{\PrintDateB}[1]{\print@date}
%    \end{macrocode}
%    \end{macro}
%
%    \begin{macro}{\print@date}
%    \begin{macrocode}
\def\print@date{%
    \ifx\bib@month\@empty
    \else
        \print@month@day
    \fi
    \bib@year
}
%    \end{macrocode}
%    \end{macro}
%
%    \begin{macro}{\print@month@day}
%    \begin{macrocode}
\def\print@month@day{%
    \bib@monthname
    \ifx\@empty\bib@day \else \nobreakspace\number 0\bib@day,\fi
    \space
}
%    \end{macrocode}
%    \end{macro}
%
%    \begin{macro}{\bib@monthname}
%    With the Babel package, month names for a given language are
%    typically available in a macro \cs{month@}\emph{language}:
%    \begin{verbatim}
%    \def\month@german{\ifcase\month\or
%      Januar\or Februar\or M\"arz\or April\or Mai\or Juni\or
%      Juli\or August\or September\or Oktober\or November\or Dezember\fi}
%\end{verbatim}
%    However this is not true for English.
%    \begin{macrocode}
\newcommand{\bib@monthname}{%
    \ifcase 0\bib@month
    \or January\or February\or March\or April\or May\or June\or
      July\or August\or September\or October\or November\or December\or
      Winter\or Spring\or Summer\or Fall\else Unknown Month%
    \fi
}
%    \end{macrocode}
%    \end{macro}
%
%    \begin{macro}{\PrintYear}
%    You can use \cn{PrintYear} if you want to suppress month/day even
%    when supplied in the data.
%    \begin{macrocode}
\newcommand{\PrintYear}[1]{\bib@year}
%    \end{macrocode}
%    \end{macro}
%
%    \begin{macro}{\PrintDatePosted}
%    This one is special for AMS use.
%    \begin{macrocode}
\newcommand{\PrintDatePosted}[1]{\unskip,  posted on \print@date}
%    \end{macrocode}
%    \end{macro}
%
%    \begin{macro}{\PrintEdition}
%    \begin{macrocode}
\newcommand{\PrintEdition}[1]{%
    \afterassignment\print@edition
    \count@ 0#1\relax\@nil
}
%    \end{macrocode}
%    \end{macro}
%
%    \begin{macro}{\print@edition}
%    If the number assignment swept up all the contents, produce a
%    cardinal number from \cs{count@}.
%    \begin{macrocode}
\def\print@edition#1#2\@nil{%
    \ifx\relax#1\relax
        \ifnum\count@>\z@
            \CardinalNumeric\count@
        \else
            ??th%
        \fi
        \ \editiontext
    \else
        \ifnum \count@>\z@ \number\count@ \fi
        #1#2\relax
    \fi
}
%    \end{macrocode}
%    \end{macro}
%
%    \begin{macro}{\editiontext}
%    \begin{macrocode}
\newcommand{\editiontext}{ed.}
%    \end{macrocode}
%    \end{macro}
%
%    \begin{macro}{\CardinalNumber}
%    \begin{macrocode}
\newcommand{\CardinalNumeric}[1]{%
    \number#1\relax
    \if
        \ifnum#1<14
            \ifnum#1>\thr@@ T\else F\fi
        \else
            F%
        \fi
        T%
            th%
    \else
        \@xp\keep@last@digit\@xp#1\number#1\relax
        \ifcase#1th\or st\or nd\or rd\else th\fi
    \fi
}
%    \end{macrocode}
%    \end{macro}
%
%    \begin{macro}{\keep@last@digit}
%    \begin{macrocode}
\def\keep@last@digit#1#2{%
    \ifx\relax#2%
        \@xp\@gobbletwo
    \else
        #1=#2\relax
    \fi
    \keep@last@digit#1%
}
%    \end{macrocode}
%    \end{macro}
%
%    \begin{macro}{\SentenceSpace}
%    Note how careful we are here to preserve \cn{frenchspacing}.
%    \begin{macrocode}
\newcommand{\SentenceSpace}{\relax\ifhmode\spacefactor`\. \fi}
%    \end{macrocode}
%    \end{macro}
%
%    \begin{macro}{\eprint}
%    For now, this does nothing. Could do a url/hyperlink or something.
%    \begin{macrocode}
\newcommand{\eprint}[1]{\url{#1}}
%    \end{macrocode}
%    \end{macro}
%    The \url{www.arXiv.org} recommendations for citing their eprints are
%    found at \url{http://xxx.lanl.gov/help/faq/references}, including
%    these examples:
%    \begin{verbatim}
%    arXiv:hep-th/9910001
%    arXiv:math.AT/9910001
%    arXiv:physics.acc-ph/9911027
%\end{verbatim}
%
% \subsection{\BibTeX\ support}
%
%    \begin{macro}{\bibliographystyle}
%    Disable \cn{bibliographystyle} since we're going to handle that
%    behind the scenes.
%    \begin{macrocode}
\let\bibliographystyle\@gobble
%    \end{macrocode}
%    \end{macro}
%
%    \begin{macro}{\bibtex@style}
%    \begin{macrocode}
\def\bibtex@style{amsrn}
%    \end{macrocode}
%    \end{macro}
%
%    \begin{macrocode}
\AtBeginDocument{
    \if@filesw
        \immediate\write\@auxout{\string\bibstyle{\bibtex@style}}%
    \fi
}
%    \end{macrocode}
%
% \subsection{Implementing package options}
%
% \subsubsection{The \opt{alphabetic} option}
%
%    \begin{macrocode}
\IfOption{alphabetic}{%
    \def\bibtex@style{amsra}%
    \def\alpha@label{%
        \ifx\@empty\bib'label
            \def\thebib{\CurrentBib}%
        \else
            \let\thebib\bib'label
        \fi
    }%
    \let\generate@label\generate@alphalabel
    \let\process@citelist\process@citelist@unsorted
    \def\numeric@refs{01}%
}{}
%    \end{macrocode}
%
% \subsubsection{The \opt{shortalphabetic} option}
% \label{opt:shortalphabetic}
%
%    \begin{macrocode}
\IfOption{shortalphabetic}{%
    \def\bibtex@style{amsrs}%
    \def\alpha@label{%
        \ifx\@empty\bib'label
            \def\thebib{\CurrentBib}%
        \else
            \let\thebib\bib'label
        \fi
    }%
    \let\@suffix@format\@arabic
    \def\calc@author@part{%
        \@xp\@multiauthorlabel\@xp{\@tempa}%
    }%
    \let\append@label@year\@empty
    \let\generate@label\generate@alphalabel
    \let\process@citelist\process@citelist@unsorted
    \def\numeric@refs{01}%
}{}
%    \end{macrocode}
%
% \subsubsection{The \opt{backrefs} option}
%
%    \begin{macrocode}
\IfOption{backrefs}{%
    \let\PrintBackRefs\print@backrefs
    \@ifundefined{Hy@backout}{%
        \amsrefs@warning{backref option requires hyperref package}%
    }{%
        \let\BackCite\back@cite
        \AtBeginDocument{\@starttoc{brf}{}}%
    }%
}{%
}
%    \end{macrocode}
%
% \subsubsection{The \opt{citation-order} option}
%
%    \begin{macrocode}
\IfOption{citation-order}{%
    \IfOption{alphabetic}{%
        \amsrefs@warning@nl{%
            The 'citation-order' and 'alphabetic' options are
            incompatible%
        }%
    }{
        \def\bibtex@style{amsru}%
    }
}{}
%    \end{macrocode}
%
% \subsubsection{The \opt{initials} option}
%
%    \begin{macrocode}
\IfOption{initials}{% TRUE:
    \BibSpec{nameLE}{
        +{}{}{initials}
        +{}{\IfEmptyBibField{initials}{}{ }}{surname}
        +{}{ }{jr}
    }

    \BibSpec{nameBE}{
        +{}{}{surname}
        +{}{ }{initials}
    %    +{}{ }{jr}
    }

    \BibSpec{nameinverted}{
        +{} {}  {surname}
        +{,}{ } {initials}
        +{,}{ } {jr}
    }
}{% initials? FALSE:
%    \let\extract@initials\@gobble
} % end conditional code for initials option
%    \end{macrocode}
%
% \subsubsection{The \opt{jpa} option}
%
%    \begin{macrocode}
\IfOption{jpa}{%
    \amsrefs@warning{The 'jpa' option is obsolete}%
    \typeout{Trying \string\usepackage{amsjpa} instead ...}%
    \RequirePackage{amsjpa}[2000/02/02]
}{}
%    \end{macrocode}
%
% \subsubsection{The \opt{logical-quotes} option}
%
%    \begin{macro}{\deferredquotes}
%    \begin{macrocode}
\let\deferredquotes\@empty
%    \end{macrocode}
%    \end{macro}
%
%    \begin{macro}{\deferredquoteslogical}
%    \begin{macrocode}
\IfOption{logical-quotes}{%
    \def\deferredquoteslogical{\deferredquotes}%
}{%
    \let\deferredquoteslogical\relax
}
%    \end{macrocode}
%    \end{macro}
%
% \subsubsection{The \opt{non-compressed-cites} option}
%
%    \begin{macrocode}
\IfOption{non-compressed-cites}{%
    \let\cite@compress\cite@print
}{}
%    \end{macrocode}
%
% \subsubsection{The \opt{non-sorted-cites} option}
%
%    \begin{macrocode}
\IfOption{non-sorted-cites}{%
    \let\process@citelist\process@citelist@unsorted
}{}
%    \end{macrocode}
%
% \subsubsection{The \opt{short-journals} option}
%
%    \begin{macrocode}
\IfOption{short-journals}{%
    \renewcommand{\DefineJournal}[4]{%
        \bib*{#1}{periodical}{
            issn={#2},
            journal={#3},
        }%
    }
}{}
%    \end{macrocode}
%
% \subsubsection{The \opt{short-publishers} option}
%
%    \begin{macrocode}
\IfOption{short-publishers}{%
    \renewcommand{\DefinePublisher}[4]{%
        \bib*{#1}{publisher}{%
            publisher={#2},%
%    \end{macrocode}
%    Maybe \opt{short-publishers} should suppress the \fld{address}?  Or
%    is that a separate option?  I sense a combinatorial explosion
%    coming on.\ldots
%    \begin{macrocode}
            address={#4},
        }%
    }%
}{}
%    \end{macrocode}
%
% \subsubsection{The \opt{short-months} option}
%
%    \begin{macrocode}
\IfOption{short-months}{%
    \renewcommand{\bib@monthname}{%
        \ifcase 0\bib@month
        \or Jan.\or Feb.\or Mar.\or Apr.\or May\or June\or
          July\or Aug.\or Sep.\or Oct.\or Nov.\or Dec.\or
          Winter\or Spring\or Summer\or Fall\else Unknown Month%
        \fi
    }%
}{}
%    \end{macrocode}
%
% \subsubsection{The \opt{y2k} option}
%
%    \begin{macrocode}
\IfOption{y2k}{%
    \IfOption{alphabetic}{%
        \def\year@short#1\@nil{#1}%
        \def\bibtex@style{amsry}%
    }{%
        \amsrefs@warning@nl{%
            The 'y2k' option can only be used with the^^J%
            'alphabetic' option%
        }%
}
}{}
%    \end{macrocode}
%
% \subsubsection{The \opt{bibtex-style} option}
%
%    \begin{macrocode}
\IfOption{bibtex-style}{%
    \RequirePackage{amsbst}
}{}
%    \end{macrocode}
%
% \subsubsection{The \opt{author-year} option}
%
%    Here ends the \pkg{amsrefs} package, unless the \opt{author-year}
%    option is in effect; then we want to use some different
%    bibspecs.
%    \begin{macrocode}
\IfOption{author-year}{}{\PopCatcodes \endinput}
%    \end{macrocode}
%
%    \begin{macro}{\generate@label}
%    \begin{macrocode}
\def\generate@label{%
%    \end{macrocode}
%    If the user supplied an explicit \fld{label} field, we use it.
%    Otherwise, we generate our own.
%    \begin{macrocode}
    \ifx\bib'label\@empty
        \begingroup
%    \end{macrocode}
%    We begin by saving the previous stem and initializing the
%    current stem to the empty string.
%    \begin{macrocode}
            \global\let\previous@stem\current@stem
            \global\let\current@stem\@empty
            \global\let\previous@year\current@year
            \global\let\current@year\bib@year
%    \end{macrocode}
%    The list of primary contributors is available to us in
%    \cs{current@primary} in the form
%    \begin{quote}
%    \cnm{name}{$\mathrm{Last}_1, \mathrm{First}_1$}%
%    \cnm{name}{$\mathrm{Last}_2, \mathrm{First}_2$}
%     \ldots
%    \cnm{name}{$\mathrm{Last}_n, \mathrm{First}_n$}
%    \end{quote}
%    We will be executing this list multiple times with various
%    definitions of \cs{name}.  So the first thing we want to do is
%    establish a safe environment and normalize the names.
%    \begin{macrocode}
            \@apply\auto@protect\amsrefs@textsymbols
            \@apply\auto@protect\amsrefs@textaccents
            \def\name##1{\@nx\name{\lnscan@a##1,\@nil}}%
            \auto@protect\etaltext
            \normalize@edef\current@stem{\current@primary}%
            \xdef\current@stem{\current@stem}%
%    \end{macrocode}
%    At this point, the \cs{current@stem} is complete and we're ready
%    to determine what (if any) suffix is needed to disambiguate
%    it from the previous label.
%    \begin{macrocode}
            \calc@alpha@suffix
%    \end{macrocode}
%    We have all the pieces now.  Arrange to end the current group and
%    then define \cs{bib@label} in the enclosing group.  (This keeps
%    \cs{bib@label} from being defined outside the group started by
%    \cs{bib@start}.  This isn't strictly necessary, but it provides a
%    bit of compartmentalization.)
%    \begin{macrocode}
            \edef\@tempa{%
                \def\@nx\cite@label{\current@stem}%
                \def\@nx\bib@label@year{%
                    \current@year
                    \alpha@label@suffix
                }%
            }
        \@xp\endgroup
        \@tempa
    \fi
}
%    \end{macrocode}
%    \end{macro}
%
%    \begin{macro}{\lnscan@a}
%    \begin{macrocode}
\def\lnscan@a#1,#2\@nil{#1}
%    \end{macrocode}
%    \end{macro}
%
%    \begin{macro}{\citesel@author}
%    \begin{macrocode}
\def\citesel@author#1#2#3#4#5{\PrintCiteNames{#3}}
%    \end{macrocode}
%    \end{macro}
%
%    \begin{macro}{\citesel@authoryear}
%    \begin{macrocode}
\def\citesel@authoryear#1#2#3#4#5{\PrintCNY{#3}{#4}}
%    \end{macrocode}
%    \end{macro}
%
%    \begin{macro}{\citesel@object}
%    \begin{macrocode}
\def\citesel@object#1#2#3#4#5{\PrintCiteNames{#3} \citeleft#4}
%    \end{macrocode}
%    \end{macro}
%
%    \begin{macro}{\citesel}
%    \begin{macrocode}
\let\citesel\citesel@authoryear
%    \end{macrocode}
%    \end{macro}
%
%    \begin{macro}{\numeric@refs}
%    \begin{macrocode}
\def\numeric@refs{01}%
%    \end{macrocode}
%    \end{macro}
%
%    \begin{macro}{\citeleft}
%    \begin{macrocode}
\def\citeleft{(}%
%    \end{macrocode}
%    \end{macro}
%
%    \begin{macro}{\citeright}
%    \begin{macrocode}
\def\citeright{)}%
%    \end{macrocode}
%    \end{macro}
%
%    \begin{macro}{\@citeleft}
%    \begin{macrocode}
\def\@citeleft{\ifx\citesel\citesel@object\else\citeleft\fi}%
%    \end{macrocode}
%    \end{macro}
%
%    \begin{macro}{\citepunct}
%    \begin{macrocode}
\def\citepunct{; }
%    \end{macrocode}
%    \end{macro}
%
%    \begin{macro}{\BibLabel}
%    \begin{macrocode}
\let\BibLabel\@empty
%    \end{macrocode}
%    \end{macro}
%
%    \begin{macro}{\process@citelist}
%    \begin{macrocode}
\let\process@citelist\process@citelist@unsorted
%    \end{macrocode}
%    \end{macro}
%
%    \begin{macro}{\ycite}
%    \begin{macrocode}
\DeclareRobustCommand{\ycite}[1]{%
    \star@{\cite@a\citesel@year{#1}}{}%
}
%    \end{macrocode}
%    \end{macro}
%
%    \begin{macro}{\ycites}
%    \begin{macrocode}
\DeclareRobustCommand{\ycites}[1]{%
    \begingroup
        \def\citepunct{, }%
        \let\citesel\citesel@year
        \cites{#1}%
    \endgroup
}
%    \end{macrocode}
%    \end{macro}
%
%    \begin{macro}{\ocite}
%    \begin{macrocode}
\DeclareRobustCommand{\ocite}[1]{%
    \star@{\cite@a\citesel@object{#1}}{}%
}
%    \end{macrocode}
%    \end{macro}
%
%    \begin{macro}{\ocites}
%    \begin{macrocode}
\DeclareRobustCommand{\ocites}[1]{%
    \begingroup
        \let\@citelist\@ocitelist
        \cites{#1}%
    \endgroup
}
%    \end{macrocode}
%    \end{macro}
%
%    \begin{macro}{\ocitelist}
%    \begin{macrocode}
\def\@ocitelist#1{%
    \PrintSeries{\InnerCite}%
        {\ocite}%
        {}{ and \ocite}%
%    \end{macrocode}
%    For three or more names: print `et al' instead of the last name.
%    Have to putz around with the space factor a bit or the comma
%    between name and year will not be applied.
%    \begin{macrocode}
        {,}{ \ocite}%
        {,}{ and \ocite}%
        {}%
        {#1}%
        {}%
}
%    \end{macrocode}
%    \end{macro}
%
%    \begin{macro}{\citeauthor}
%    \begin{macrocode}
\DeclareRobustCommand{\citeauthor}[1]{%
    \star@{\cite@a\citesel@author{#1}}{}%
}
%    \end{macrocode}
%    \end{macro}
%
%    \begin{macro}{\citeauthory}
%    \begin{macrocode}
\DeclareRobustCommand{\citeauthory}[1]{%
    \citeauthor{#1} \ycite{#1}%
}
%    \end{macrocode}
%    \end{macro}
%
%    \begin{macro}{\fullcite}
%    \begin{macrocode}
\DeclareRobustCommand{\fullcite}[1]{%
    \begingroup
        \let\print@citenames\CiteNamesFull
        \star@{\cite@a\citesel@authoryear{#1}}{}%
    \endgroup
}
%    \end{macrocode}
%    \end{macro}
%
%    \begin{macro}{\fullocite}
%    \begin{macrocode}
\DeclareRobustCommand{\fullocite}[1]{%
    \begingroup
        \let\print@citenames\CiteNamesFull
        \star@{\cite@a\citesel@object{#1}}{}%
    \endgroup
}
%    \end{macrocode}
%    \end{macro}
%
%    Invert the first author's name.
%    \begin{macrocode}
\def\set@firstname#1{%
    \set@name{#1}\setbib@nameinverted
}
%    \end{macrocode}
%
%    \begin{macro}{\PrintCNY}
%    \begin{macrocode}
\def\PrintCNY#1#2{%
    \PrintCiteNames{#1}%
    \@ifnotempty{#2}{\@addpunct{,} #2}%
}
%    \end{macrocode}
%    \end{macro}
%
%    \begin{macro}{\PrintCiteNames}
%    \begin{macrocode}
\def\PrintCiteNames#1{%
    \leavevmode
    \def\@tempa{#1}%
    \ifx\@tempa\prev@names
    \else
        \gdef\prev@names{#1}%
        \@xp\ifx\@car#1.\@nil\CitePrintUndefined
            #1\relax
        \else
            \print@citenames{#1}%
        \fi
    \fi
}
%    \end{macrocode}
%    \end{macro}
%
%    \begin{macro}{\CiteNames}
%    \begin{macrocode}
\newcommand{\CiteNames}[1]{%
    \PrintSeries{\name}%
        {}%
        {}{ and }%
%    \end{macrocode}
%    For three or more names: print `et al' instead of the last name.
%    Have to putz around with the space factor a bit or the comma
%    between name and year will not be applied.
%    \begin{macrocode}
        {}{\@gobble}%
        {}{ \etaltext\@\@gobble}%
        {}%
        {#1}%
        {}%
}
%    \end{macrocode}
%    \end{macro}
%
%    \begin{macro}{\print@citenames}
%    \begin{macrocode}
\let\print@citenames\CiteNames
%    \end{macrocode}
%    \end{macro}
%
%    \begin{macro}{\CiteNamesFull}
%    \begin{macrocode}
\newcommand{\CiteNamesFull}[1]{%
    \PrintSeries{\name}%
        {}%
        {}{ and }%
%    \end{macrocode}
%    For three or more names: print `et al' instead of the last name.
%    Have to putz around with the space factor a bit or the comma
%    between name and year will not be applied.
%    \begin{macrocode}
        {,}{ }%
        {,}{ and }%
        {}%
        {#1}%
        {}%
}
%    \end{macrocode}
%    \end{macro}
%
%    \begin{macro}{\PrintDate}
%    No parentheses around the year.
%    \begin{macrocode}
\renewcommand{\PrintDate}[1]{\bib@label@year}
%    \end{macrocode}
%    \end{macro}
%
%    \begin{macro}{\print@date}
%    Only print the year, not the month or day.
%    \begin{macrocode}
\def\print@date{%
    \IfEmptyBibField{date}{%
        \IfEmptyBibField{year}{\BibField{status}}{\bib@year}%
    }{%
        \bib@year
    }%
}
%    \end{macrocode}
%    \end{macro}
%
%    \begin{macrocode}
\BibSpec{article}{%
    +{}  {\PrintAuthors}                {author}
    +{.} { \PrintDate}                  {date}
    +{.} { \textit}                     {title}
    +{.} { }                            {part}
    +{:} { \textit}                     {subtitle}
    +{,} { \PrintContributions}         {contribution}
    +{.} { \PrintPartials}              {partial}
    +{,} { }                            {journal}
    +{}  { \textbf}                     {volume}
    +{,} { \issuetext}                  {number}
    +{,} { \eprintpages}                {pages}
    +{,} { }                            {status}
    +{,} { \PrintDOI}                   {doi}
    +{,} { available at \eprint}        {eprint}
    +{}  { \parenthesize}               {language}
    +{}  { \PrintTranslation}           {translation}
    +{;} { \PrintReprint}               {reprint}
    +{.} { }                            {note}
    +{.} {}                             {transition}
    +{}  {\SentenceSpace \PrintReviews} {review}
}

\BibSpec{book}{%
    +{}  {\PrintPrimary}                {transition}
    +{.} { \PrintDate}                  {date}
    +{.} { \textit}                     {title}
    +{.} { }                            {part}
    +{:} { \textit}                     {subtitle}
    +{,} { \PrintEdition}               {edition}
    +{}  { \PrintEditorsB}              {editor}
    +{,} { \PrintTranslatorsC}          {translator}
    +{,} { \PrintContributions}         {contribution}
    +{,} { }                            {series}
    +{,} { \voltext}                    {volume}
    +{,} { }                            {publisher}
    +{,} { }                            {organization}
    +{,} { }                            {address}
    +{,} { }                            {status}
    +{}  { \parenthesize}               {language}
    +{}  { \PrintTranslation}           {translation}
    +{;} { \PrintReprint}               {reprint}
    +{.} { }                            {note}
    +{.} {}                             {transition}
    +{}  {\SentenceSpace \PrintReviews} {review}
}

\BibSpec{collection.article}{%
    +{}  {\PrintAuthors}                {author}
    +{.} { \PrintDate}                  {date}
    +{.} { \textit}                     {title}
    +{.} { }                            {part}
    +{:} { \textit}                     {subtitle}
    +{,} { \PrintContributions}         {contribution}
    +{,} { \PrintConference}            {conference}
    +{}  {\PrintBook}                   {book}
    +{,} { }                            {booktitle}
    +{,} { pp.~}                        {pages}
    +{,} { }                            {status}
    +{,} { \PrintDOI}                   {doi}
    +{,} { available at \eprint}        {eprint}
    +{}  { \parenthesize}               {language}
    +{}  { \PrintTranslation}           {translation}
    +{;} { \PrintReprint}               {reprint}
    +{.} { }                            {note}
    +{.} {}                             {transition}
    +{}  {\SentenceSpace \PrintReviews} {review}
}

\BibSpec{report}{%
    +{}  {\PrintPrimary}                {transition}
    +{.} { \PrintDate}                  {date}
    +{.} { \textit}                     {title}
    +{.} { }                            {part}
    +{:} { \textit}                     {subtitle}
    +{,} { \PrintEdition}               {edition}
    +{,} { \PrintContributions}         {contribution}
    +{,} { Technical Report }           {number}
    +{,} { }                            {series}
    +{,} { }                            {organization}
    +{,} { }                            {address}
    +{,} { \eprint}                     {eprint}
    +{,} { }                            {status}
    +{}  { \parenthesize}               {language}
    +{}  { \PrintTranslation}           {translation}
    +{;} { \PrintReprint}               {reprint}
    +{.} { }                            {note}
    +{.} {}                             {transition}
    +{}  {\SentenceSpace \PrintReviews} {review}
}

\BibSpec{thesis}{%
    +{}  {\PrintAuthors}                {author}
    +{.} { \PrintDate}                  {date}
    +{.} { \textit}                     {title}
    +{:} { \textit}                     {subtitle}
    +{,} { \PrintThesisType}            {type}
    +{,} { }                            {organization}
    +{,} { }                            {address}
    +{,} { \eprint}                     {eprint}
    +{,} { }                            {status}
    +{}  { \parenthesize}               {language}
    +{}  { \PrintTranslation}           {translation}
    +{;} { \PrintReprint}               {reprint}
    +{.} { }                            {note}
    +{.} {}                             {transition}
    +{}  {\SentenceSpace \PrintReviews} {review}
}
%    \end{macrocode}
%
%    \begin{macrocode}
\PopCatcodes
%</pkg>
%    \end{macrocode}
%
% \subsection{The \pkg{amsbst} package}
%
%    \begin{macrocode}
%<*bst>
%    \end{macrocode}
%
%    \begin{macrocode}
\NeedsTeXFormat{LaTeX2e}[1995/12/01]
\ProvidesPackage{amsbst}[2004/03/29 v1.68]
%\RequirePackage{amsrefs}[2004/03/29]
%    \end{macrocode}
%
%    \begin{macrocode}
\BibSpec{article}{%
    +{}  {\PrintAuthors}                {author}
    +{.} { }                            {title}
    +{.} { }                            {part}
    +{:} { }                            {subtitle}
    +{.} { \PrintContributions}         {contribution}
    +{.} { \PrintPartials}              {partial}
    +{.} { \emph}                       {journal}
    +{}  { }                            {volume}
    +{}  { \parenthesize}               {number}
    +{:} {}                             {pages}
    +{,} { \PrintDateB}                 {date}
    +{,} { }                            {status}
    +{.} { \PrintTranslation}           {translation}
    +{.} { Reprinted in \PrintReprint}  {reprint}
    +{.} { }                            {note}
    +{.} {}                             {transition}
}

\BibSpec{partial}{%
    +{}  {}                             {part}
    +{:} { }                            {subtitle}
    +{.} { \PrintContributions}         {contribution}
    +{.} { \emph}                       {journal}
    +{}  { }                            {volume}
    +{}  { \parenthesize}               {number}
    +{:} {}                             {pages}
    +{,} { \PrintDateB}                 {date}
}

\BibSpec{book}{%
    +{}  {\PrintPrimary}                {transition}
    +{.} { \emph}                       {title}
    +{.} { }                            {part}
    +{:} { \emph}                       {subtitle}
    +{.} { }                            {series}
    +{,} { \voltext}                    {volume}
    +{.} { Edited by \PrintNameList}    {editor}
    +{.} { Translated by \PrintNameList}{translator}
    +{.} { \PrintContributions}         {contribution}
    +{.} { }                            {publisher}
    +{.} { }                            {organization}
    +{,} { }                            {address}
    +{,} { \PrintEdition}               {edition}
    +{,} { \PrintDateB}                 {date}
    +{.} { }                            {note}
    +{.} {}                             {transition}
    +{.} { \PrintTranslation}           {translation}
    +{.} { Reprinted in \PrintReprint}  {reprint}
    +{.} {}                             {transition}
}

\BibSpec{collection.article}{%
    +{}  {\PrintAuthors}                {author}
    +{.} { }                            {title}
    +{.} { }                            {part}
    +{:} { }                            {subtitle}
    +{.} { \PrintContributions}         {contribution}
    +{.} { \PrintConference}            {conference}
    +{.} { \PrintBook}                  {book}
    +{.} { In }                         {booktitle}
    +{,} { pages~}                      {pages}
    +{.} { \PrintDateB}                 {date}
    +{.} { \PrintTranslation}           {translation}
    +{.} { Reprinted in \PrintReprint}  {reprint}
    +{.} { }                            {note}
    +{.} {}                             {transition}
}

\BibSpec{conference}{%
    +{}  {}                        {title}
    +{}  {\PrintConferenceDetails} {transition}
}

\BibSpec{innerbook}{%
    +{.} { \emph}                       {title}
    +{.} { }                            {part}
    +{:} { \emph}                       {subtitle}
    +{.} { }                            {series}
    +{,} { \voltext}                    {volume}
    +{.} { Edited by \PrintNameList}    {editor}
    +{.} { Translated by \PrintNameList}{translator}
    +{.} { \PrintContributions}         {contribution}
    +{.} { }                            {publisher}
    +{.} { }                            {organization}
    +{,} { }                            {address}
    +{,} { \PrintEdition}               {edition}
    +{,} { \PrintDateB}                 {date}
    +{.} { }                            {note}
    +{.} {}                             {transition}
}

\BibSpec{report}{%
    +{}  {\PrintPrimary}                {transition}
    +{.} { \emph}                       {title}
    +{.} { }                            {part}
    +{:} { \emph}                       {subtitle}
    +{.} { \PrintContributions}         {contribution}
    +{.} { Technical Report }           {number}
    +{,} { }                            {series}
    +{.} { }                            {organization}
    +{,} { }                            {address}
    +{,} { \PrintDateB}                 {date}
    +{.} { \PrintTranslation}           {translation}
    +{.} { Reprinted in \PrintReprint}  {reprint}
    +{.} { }                            {note}
    +{.} {}                             {transition}
}

\BibSpec{thesis}{%
    +{}  {\PrintAuthors}                {author}
    +{,} { \emph}                       {title}
    +{:} { \emph}                       {subtitle}
    +{.} { \PrintThesisType}            {type}
    +{.} { }                            {organization}
    +{,} { }                            {address}
    +{,} { \PrintDateB}                 {date}
    +{.} { \PrintTranslation}           {translation}
    +{.} { Reprinted in \PrintReprint}  {reprint}
    +{.} { }                            {note}
    +{.} {}                             {transition}
}
%    \end{macrocode}
%
%    \begin{macro}{\PrintEditorsA}
%    When we consider editor names we have to think about some further
%    complications. First, for the case of a book where editor names are
%    listed in place of author names, just copy the same style with a
%    bit of added text at the end.
%    \begin{macrocode}
\renewcommand{\PrintEditorsA}[1]{%
    \def\current@bibfield{\bib'editor}%
    \PrintNames{}{, editor\Plural{s})}{#1}%
    \erase@field\bib'editor
}
%    \end{macrocode}
%    \end{macro}
%
%    \begin{macro}{\PrintTranslatorsA}
%    \begin{macrocode}
\renewcommand{\PrintTranslatorsA}[1]{%
    \def\current@bibfield{\bib'translator}%
    \PrintNames{}{, translator\Plural{s}}{#1}%
    \erase@field\bib'translator
}
%    \end{macrocode}
%    \end{macro}
%
%    \begin{macrocode}
%</bst>
%    \end{macrocode}
%
%    The usual \cs{endinput} to ensure that random garbage at the end of
%    the file doesn't get copied by \fn{docstrip}.
%    \begin{macrocode}
\endinput
%    \end{macrocode}
%
% \CheckSum{5005}
% \Finale
